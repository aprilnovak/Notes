\documentclass[10pt]{article}
\usepackage[letterpaper]{geometry}
\geometry{verbose,tmargin=1in,bmargin=1in,lmargin=1in,rmargin=1in}
\usepackage{setspace}
\usepackage{ragged2e}
\usepackage{color}
\usepackage{titlesec}
\usepackage{graphicx}
\usepackage{float}
\usepackage{mathtools}
\usepackage{amsmath}
\usepackage[font=small,labelfont=bf,labelsep=period]{caption}
\usepackage[english]{babel}
\usepackage{indentfirst}
\usepackage{array}
\usepackage{makecell}
\usepackage[usenames,dvipsnames]{xcolor}
\usepackage{multirow}
\usepackage{tabularx}
\usepackage{arydshln}
\usepackage{caption}
\usepackage{subcaption}
\usepackage{xfrac}
\usepackage{etoolbox}
\usepackage{cite}
\usepackage{url}
\usepackage{dcolumn}
\usepackage{hyperref}
\usepackage{courier}
\usepackage{url}
\usepackage{esvect}
\usepackage{commath}
\usepackage{verbatim} % for block comments
\usepackage{enumitem}
\usepackage{hyperref} % for clickable table of contents
\usepackage{braket}
\usepackage{titlesec}
\usepackage{booktabs}
\usepackage{gensymb}
\usepackage{longtable}
\usepackage{soul} % for striking out text
\usepackage{tcolorbox} % for colored boxes
\tcbuselibrary{breakable} % to allow colored boxed to extend over multiple pages
\usepackage[makeroom]{cancel}	% to cancel out text
\usepackage{breqn}
\usepackage{listings}
\usepackage[mathscr]{euscript}

% for circled numbers
\usepackage{tikz}
\newcommand*\circled[1]{\tikz[baseline=(char.base)]{
            \node[shape=circle,draw,inner sep=2pt] (char) {#1};}}


\titleclass{\subsubsubsection}{straight}[\subsection]

% define new command for triple sub sections
\newcounter{subsubsubsection}[subsubsection]
\renewcommand\thesubsubsubsection{\thesubsubsection.\arabic{subsubsubsection}}
\renewcommand\theparagraph{\thesubsubsubsection.\arabic{paragraph}} % optional; useful if paragraphs are to be numbered

\titleformat{\subsubsubsection}
  {\normalfont\normalsize\bfseries}{\thesubsubsubsection}{1em}{}
\titlespacing*{\subsubsubsection}
{0pt}{3.25ex plus 1ex minus .2ex}{1.5ex plus .2ex}

\makeatletter
\renewcommand\paragraph{\@startsection{paragraph}{5}{\z@}%
  {3.25ex \@plus1ex \@minus.2ex}%
  {-1em}%
  {\normalfont\normalsize\bfseries}}
\renewcommand\subparagraph{\@startsection{subparagraph}{6}{\parindent}%
  {3.25ex \@plus1ex \@minus .2ex}%
  {-1em}%
  {\normalfont\normalsize\bfseries}}
\def\toclevel@subsubsubsection{4}
\def\toclevel@paragraph{5}
\def\toclevel@paragraph{6}
\def\l@subsubsubsection{\@dottedtocline{4}{7em}{4em}}
\def\l@paragraph{\@dottedtocline{5}{10em}{5em}}
\def\l@subparagraph{\@dottedtocline{6}{14em}{6em}}
\makeatother

\newcommand{\volume}{\mathop{\ooalign{\hfil$V$\hfil\cr\kern0.08em--\hfil\cr}}\nolimits}

\setcounter{secnumdepth}{4}
\setcounter{tocdepth}{4}

\usepackage[utf8]{inputenc}
 
\usepackage{listings}
\usepackage{color}
 
\definecolor{codegreen}{rgb}{0,0.6,0}
\definecolor{codegray}{rgb}{0.5,0.5,0.5}
\definecolor{codepurple}{rgb}{0.58,0,0.82}
\definecolor{backcolour}{rgb}{0.95,0.95,0.92}
 
\lstdefinestyle{mystyle}{
    backgroundcolor=\color{backcolour},   
    commentstyle=\color{codegreen},
    keywordstyle=\color{magenta},
    numberstyle=\tiny\color{codegray},
    stringstyle=\color{codepurple},
    basicstyle=\footnotesize,
    breakatwhitespace=false,         
    breaklines=true,                 
    captionpos=b,                    
    keepspaces=true,                 
    numbers=left,                    
    numbersep=5pt,                  
    showspaces=false,                
    showstringspaces=false,
    showtabs=false,                  
    tabsize=2
}
 
\lstset{style=mystyle}

\begin{document}

\tableofcontents
\clearpage

Exnihilo is a neutron transport framework that consists of Denovo and Shift, and the capabilities to perform hybrid calculations to accelerate Shift calculations. The Omnibus front-end interface is used for both Denovo and Shift, but also by other codes such as ADVANTAG to create, execute, and postprocess Denovo problems. Exnihilo supports a variety of input formats, and can read input files developed for other codes. A geometry transportable insid Shift can be discretized for Denovo using a ray tracer (?). Similarly, material definitions can be converted to multigroup cross sections using SCALE libraries. This allows easy use of both Denovo and Shift. Exnihilo also contains Insilico, which is a front-end that couples Denovo and Shift with cross-section processing, depletion, and thermal-hydraulics feedback for reactor analysis. Insilico is integrated into VERA.

Exnihilo must be build alongside SCALE, Trilinos (similar to PETSc, except that this is developed at Sandia), and TriBITS (software utilities). Trilinos is more focused towards solving FEM problems, while PETSc towards sparse linear algebra problems. Most of the software in Exnihilo is written in C++. 

This document is intended to provide a better explanation (for me) of Omnibus ASCII inputs and how to run Omnibus. This document is structured to describe in detail an Omnibus input file, where each section below describes the [DATABASES] in the Omnibus input file. A database consists of containers of parameters, sub-databases, and sublists. A sublist is a list of congruent databases. Parameters are simple elements with a name and a range of allowed values. Omnibus also permits the use of commands in the input file for creating or modifying input parameters and for pre- and post-processing. 

\section{Running Omnibus}
Omnibus drive Denovo and Shift from an XML input file. Omnibus can be run in two different ways - using the \texttt{omnibus-run} script, which performs the execution of an input file from start to finish, or using a variety of separate scripts provided to customize each aspect of the input file execution. Both methods require an ASCII file with extension \texttt{.omn} as the input file. To run using the first method:\newline

\texttt{omnibus-run inputfile.omn}\newline

The \texttt{omnibus-run} script is meant to be run on the head node of a cluster instead of on a compute node (but be aware that preprocessing some large MCNP inputs can be expensive - in those cases, generate the run tape files separately). Using job submission scripts, it will submit the job to other nodes and monitor this process. The actions performed when using the \texttt{omnibus-run} front end are described in detail when discussing the second execution method. To run using the second method requires several steps:\newline

\begin{tabular}{l}
\texttt{omnibus-pre inputfile.omn}\\
\texttt{mpirun -np 16 omnibus inputfile.inp.xml}\\
\texttt{omnibus-post inputfile.out.xml}\\
\end{tabular}\newline

The \texttt{omnibus-pre} script generates the Omnibus XML input file \texttt{inputfile.inp.xml} from the ASCII input, provided that all the input is valid. The ASCII input is unfolded into hierarchical databases, then converted to XML for the Omnibus executable. YAML and JSON hierarchical databases are also supported as input. This preprocessing is typically performed on a login node, thus allowing the input to be prepared without requiring the user to wait in a job queue and potentially waste time due to an invalid input. Another script that is available to assist with preprocessing is the \texttt{vacuum\_omnibus\_input} script, which will read the Omnibus input file and reformat it and rewrite it. Comparing this file with the original file will reveal any syntax errors in terms of how the input will eventually be interpreted.

In addition, because the Omnibus front end is simply a Python script, Python files that modify an existing ASCII Omnibus input file can be incorporated into the preprocessing. A local variable \texttt{db} contains the Omnibus input definition hierarchy can be modified or extended. 

This hierarchical XML parameter list will be read in by the \texttt{omnibus} binary executable for actual execution. If this preprocessing is successful, several output files will be created. Additional files that may be created are MCNP run tape files and PBS job submission scripts.\newline

\begin{tabular}{l l}
\texttt{inputfile.pp.json} & (\texttt{omnibus-run} only) fully-processed version of the problem input, with all\\
& default parameters filled in. This is used later for post-processing.\\
\texttt{inputfile.inp.omn} & (\texttt{omnibus-run} only) fully-processed and reformatted version of the \\
& problem input, with all default parameters filled in\\
\texttt{inputfile.inp.xml} & XML file that is read in by the \texttt{omnibus} executable\\
\end{tabular}\newline

Then, the XML input file is run using the Omnibus driver (binary \texttt{omnibus} executable). This parameter list is very complex, and should not be written from scratch. The parameter tree consists of three general elements: 1) parameters, which are simple elements with a name and allowed range of values, 2) databases, which are containers of parameters, sub-databases, and sublists, and 3) sublists, which are a list of congruent databases. 

This creates several output files:\newline

\begin{tabular}{l l}
\texttt{inputfile.out.xml} & default output file name\\
\texttt{inputfile.description.rst} & \\
\texttt{inputfile.html} & list of created files and their descriptions\\
\texttt{omnibus.out} & standard out\\
\texttt{omnibus.err} & standard error\\
\texttt{inputfile.out.xml} & \\
\texttt{tallies.h5} & \\
\end{tabular}\newline

Using the \texttt{omnibus-post} script, the XML output is converted into a more human-readable format. Several files are produced during post-processing:\newline

\begin{tabular}{l l}
\texttt{inputfile.out.rst} & \\
\texttt{inputfile.out.html} & \\
\end{tabular}\newline

Additional scripts that can be used include:\newline

\begin{tabular}{l l}
\texttt{vacuum\_omnibus\_input inputfile.omn} & checks input file for syntax errors\\
\end{tabular}\newline

\subsection{Python ASCII Input Editing}
An inputfile can be run using Python by importing from the omnibus module. The first argument to the run attribute is a list, because multiple files can be passed to the input.  

\begin{lstlisting}[language=Python]
from omnibus.scripts import omnibus_run
omnibus_run.run(["inputfile.omn"])
\end{lstlisting}

In addition to the input file name, a Python function can also be passed into the run attribute, where this function can be used to modify the hierarchical database. In addition, multiple Omnibus runs can be performed, where parameters can be modified for each separate run.

\begin{lstlisting}[language=Python]
from omnibus.scripts import omnibus_run

# function to change the name parameter in the [PROBLEM] database in the Omnibus input
def change_problem_name(db):
	db['problem']['name'] = "Alternative name"

omnibus_run.run(["inputfile.omn", change_problem_name])
\end{lstlisting}

The databases in the Omnibus input are unfolded into hierarchical databases, and then converted into a hierarchical XML parameter list for the Omnibus executable. A Python script that edits \texttt{db} could be integrated into an Omnibus run file by executing:\newline

\texttt{omnibus-run inputfile.omn python\_edits.py}\newline

\subsection{Omnibus Format}
The ASCII input consists of blocks of input data, each of which represents a database, and cards, which consist of parameters and commands. Commands generate parameters or perform other functions. Blocks have the following formats, where the options with the \texttt{name} parameter are shorthand notation for including the parameter \texttt{name} as a field under the block.\newline

[CLASS]\newline
[CLASS=type]\newline
[CLASS name]\newline
[PARENT][CLASS=type name]\newline

The \texttt{name} refers to the particular instance of each class. Each card starts on a new line, and an indentation of four or more spaces is treated as a continuation of the previous card. Spaces can be used to separate words in a parameter, but if you would like to backslash escape any characters, you'll need to enclose the entire parameter in quotes. Syntax highlighting is provided in the \texttt{Exnihilo/environment} directory to help catch Omnibus input errors.

\section{[COMP]}
This database contains composition options and definitions. By default, it is an empty database. This block controls output and management of compositions and allows them to be defined manually if they're not already defined in the geometry or if the user wants to override those values. The material ID numbers in each block must match those in the problem geometry.

\subsection{compgen \textit{type}}
This parameter defines the type of composition input - either \texttt{model}, \texttt{hdf5}, or \texttt{inline}.

\subsection{force\_scl \textit{boolean}}
This parameter forces the SCALE standard composition library to load early. Exnihilo uses a global database that contains nuclide data with names, atomic numbers, number densities, etc. that are used to construct compositions. This database is also used to generate cross sections in Fulcrum, SCALE's multigroup cross section builder. By default, the SCALE standard composition library (SCL) is loaded before libraries associated with other codes such as Geant4. Using the libraries from other codes will only result in very minor changes to composition number densities.

\subsection{input \textit{file}\quad\quad\quad(\texttt{compgen hdf5})}
This parameter contains the name of the HDF5 file (extension .h5) containing the compositions. 

\subsection{output \textit{boolean}}
By default, this parameter saves compositions to an output file.

\subsection{[COMP][COMPOUND]}
This sublist, by default empty, defines elemental compounds. 

\subsubsection{czaid \texttt{MZZZAAA}}
This parameter defines the compound ZAID. For elemental compounds, set AAA=0. 

\subsubsection{name \texttt{name}}

\subsubsection{wtfrac \texttt{number}}
Weight fraction of each constituent. 

\subsubsection{zaid \texttt{MZZZAAA}}
This defines the constituent nuclide IDs. The last three digits are the atomic mass (A), and the digits before that the atomic number (Z), so that carbon-12 would be written as 6012 (MZZZAAA). What is the difference between czaid and zaid? 

\subsection{[COMP][MATERIAL]\quad\quad\quad(\texttt{compgen inline})}
This sublist defines composition definitions.

\subsubsection{color \textit{color}}
This parameter defines colors for the material for visualization. The colors can be X11 or HTML. 

\subsubsection{deplete \textit{boolean}}
By default, materials are not depletable. 

\subsubsection{fission \textit{boolean}}
By default, materials are not fissionable. 

\subsubsection{matid \textit{number}}
Internal material ID number, which is an internal numbering system beginning at 0 and ending at \(N-1\), where \(N\) is the number of materials. These numbers typically differ from the material names used in the problem physics input file. The only way to guarantee that this corresponds to the correct material in the geometry input is to use an input that specifies material IDs explicitly, such as RTK or mesh geometry. However, the Omnibus postprocessed output will show the mapping between inline material IDs with those used in the SCALE or MCNP input geometry, and it can be determined how the IDs correspond. 

\subsubsection{name \textit{name}}

\subsubsection{nd \textit{[atoms/b-cm]}}
This parameter lists the number densities of each nuclide. 

\subsubsection{temperature \textit{[K]}}
This parameter lists the material temperature. 

\subsubsection{zaid \textit{MZZZAAA}}
This parameter lists the element/nuclide IDs in this material.

\subsection{[COMP][NUCLIDES]}
This database defines nuclides.

\subsubsection{name \textit{names}}
List of names of nuclides.

\subsubsection{zaid \textit{MZZZAAA}}
Nuclide IDs. 

\section{[DENOVO]}
This database contains Denovo solver options. Denovo is the deterministic solver within the Exnihilo framework that solves the steady-state Boltzmann transport equation. Denovo uses either discrete ordinates or spherical harmonics for its angular discretization. 

High levels of accuracy are not necessary when simply using Denovo to accelerate a Monte Carlo calculation, and just roughly-correct answers will lead to significantly higher figures of merit. The default settings in Denovo are satisfactory for a wide range of problems. 

\subsection{dimension \textit{2 or 3}}
The spatial dimension of the problem.

\subsection{do\_transport \textit{boolean}}
If false, this parameter prevents the actual deterministic solve. 

\subsection{first\_group \textit{group}}
The first energy group to solve.

\subsection{last\_group \textit{grou[}}
The last energy group to solve.

\subsection{method \textit{method}}
This parameter specifies the solution method or spatial discretization. The diamond difference and trilinear discontinuous discretizations are primarily included for comparison with other transport solvers, and should not be used for mot calculations.\newline

\begin{tabular}{l l}
spn & simplified \(P_N\) finite volume\\
sc & \(S_N\) with step characteristics spatial discretization (sc_2d for 2-D problems)\\
bld\_2d & \(S_N\) with bilinear discontinuous discretization in 2-D\\
ld, tld & linear or trilinear discontinuous discretization\\
twd, wdd, wdd\_ff & theta-weighted, weighted, and weight with flux fix-up diamond differences\\
\end{tabular}

Step characteristics (SC) guarantees positive solutions and offers good accuracy for a wide range of transport problems, so this should be the preferred method when the mesh is relatively coarse such as is required for some shielding calculations. However, linear discontinuous (LD) methods are often more accurate than SC Methods, but these do not guarantee positive solutions. The mesh must adequately capture the physics in order for LD to be useful. This method is most appropriate for problems that rarely give negative solutions, such as reactor applications with very large flux values. The \(SP_N\) method is best for fast and accurate solutions for light water reactor eigenvalue problems, since these reactors are very diffusive.

\subsection{physics \textit{database}}
This parameter contains the associated MG physics database.

\subsection{spn\_order \textit{order}}
By default, the \(SP_N\) order is 1. 

\subsection{x \textit{coordinate}}
Mesh coordinates along the \(x\)-axis. At least two values are required. Similarly, coordinates are given in the \(y\) and \(z\) direction. This is applicable when the model type is not \texttt{mesh}. 

\subsection{[DENOVO][BOUNDARY=vacuum]}
This database specifies the boundary conditions, which by default are vacuum conditions. 

\subsection{[DENOVO][BOUNDARY=reflect]}
The \texttt{reflect} boundary condition allows a mix of reflecting and vacuum boundaries - the \(\pm x\) coordinates are assigned a value of 0 if vacuum, and a value of 1 if reflecting.

\subsection{[DENOVO][BOUNDARY=isotropic]}
This parameter defines one or more isotropic incident boundaries. Flux values for each energy group are given on the six \(\pm x, y, z\) faces.

\subsubsection{minus\_x \textit{flux}}
Contains a list of the flux values for each energy group on the \(-x\) bounding surface. 

\subsubsection{plus\_x \textit{flux}}
Contains a list of the flux values for each energy group on the \(+x\) bounding surface. 


\subsection{[DENOVO][DECOMPOSITION]}
This database specifies the space-energy decomposition.

\subsubsection{energy\_sets \textit{number}}
This parameter gives the number of energy groups, which by default is 1.

\subsubsection{x\_blocks \textit{number}}
This parameter gives the number of spatial partitions along the \(x\)-axis. Similar parameters are defined for the \(y\) and \(z\) direction. Discretization in \(z\) is not permitted when using the \(SP_N\) solver?

\subsection{[DENOVO][OUTPUT]}
This database specifies HDF5 output options. This database contains parameters that accept boolean values that determine whether or not to write the following variables. The default value is given next to the parameter name.\newline

\begin{tabular}{l l l}
angular\_flux & full angular flux & false\\
block & KBA domain for each cell & true\\
current & first angular moment & false\\
flux & scalar flux & true\\
mat & materials and mix tables & false\\
source & energy-dependent source term & true\\
uncflux & uncollided flux & true\\
\end{tabular}

\subsection{[DENOVO][QUADRATURE]}
This database contains the discrete ordinates quadrature set used. This is applicable for all solve types except the \(SP_N\) solver.

\subsubsection{construction \textit{type}}
This parameter specifies the type of quadrature rule. The quadrature rule can be:

\begin{enumerate}
\item \texttt{levelsym} - level-symmetric quadrature set whose angular positions and weights are rotationally invariant (same in each 1/8 of a unit sphere). All the weights will be positive up until \(S_{20}\). In three dimensions, there are a total of \(N(N+2)/8\) angles per octant, for a total number of angles at each spatial position of \(N(N+2)\). These quadrature rules must have an even number of points. Level-symmetric quadrature tends to display the most ray effects. By default, if level-symmetric quadrature is used, the \(S_{10}\) quadrature is used.
\item \texttt{product} - product set with a specified number of azimuthal angles per polar angle, and a specified number of polar angles. This quadrature type can be used to specify Gauss-Legendre quadrature, were the quadrature points are specified by taking the combination of uniformly distributed azimuthal angles and 1-D Gauss-Legendre quadrature points in the polar angle. You could also formulate Quadruple range (QR) rules, which exactly integrate the highest-possible order products of sines and cosines of the polar and azimuthal angles. QR quadrature tends to perform very well, and display the smallest amount of ray effects in most situations. Finally, linear-discontinuous finite element (LDFE) quadratures are defined by requiring that the integration of the basis functions is equal to the surface area of a unit sphere, which gives \(4^(N+1)\) angles per octant, with weights always positive.
\item \texttt{product\_vec} - the azimuthal angles can be different at each polar angle. 
\end{enumerate}

\subsubsection{input \textit{file}\quad\quad(\texttt{quadrature userdefined})}
Quadrature set file. Each line in this file contains the three angles of each quadrature point and the corresponding weight. 

\subsection{[DENOVO][RAYTRACE]}
This database contains problem discretization options. This is applicable so long as the model type is not \texttt{mesh}.

\subsection{[DENOVO][SILO]}
This database contains Silo output options. This database contains similar boolean parameters as those for [DENOVO][OUTPUT].

\subsection{[DENOVO][SOLVER=fixed]}
This database contains the fixed-source solve options, which should be used for forward, adjoint, or hybrid modes. The withingroup and upscatter solves can be controlled from this database. 

You can specify which groups are allowed to upscatter (custom or only the thermal groups), or upscattering can be performed for all groups. The \(SP_N\) method always treats all groups as groups that can upscatter, and no customization can be done with this type of solver. In addition, if Rayleigh Quotient iteration is used, all groups must upscatter.

\subsection{[DENOVO][SOLVER=eigenvalue]}
This database contains the eigenvalue solve options, and should be used when the problem is a criticality problem. Rebalance eigenvalue acceleration methods are available. Arnoldi, power iteration, Ralyeigh Quotient iteration, and the Davidson methods are all available to solve for the eigenvalue. By default, the Arnoldi method is used for \(S_N\) solves and Davidson for \(SP_N\). 

\subsubsection{[DENOVO][SOLVER][UPSCATTER]}

\subsection{[DENOVO][SOURCE]}
This database contains Shift source discretization options. This is applicable when the problem mode is forward, adjoint, or hybrid. 

\subsubsection{max\_samples \textit{numer}}
Maximum number of samples to take per source. This is only applicable when using a separable or MCNP source. By default, a minimum of 1,000,000 samples are taken per source.

\subsubsection{mc\_num\_particles \textit{number}}
Number of uncollided source particles to sample.

\subsubsection{source\_extents \textit{\(\pm x,y,z\)}}
Bounding box for MCNP source. To permit the MCNP source to be discretized in parallel, the spatial mesh that overlaps the source must be replicated - for large problems, it is best to manually input an axis-aligned bounding box that surrounds the source region, since the default is to simply sample the entire spatial mesh. 

\subsubsection{uncflux \textit{type}}
The uncollided flux treatment can reduce ray effects from point or other small sources. The default is to treat all sources by discretizing them onto the Denovo grid as \(S_N\) sources. The default treatment is \texttt{analytic}, which treats point sources as analytic uncollided flux sources. The \texttt{mc} option is only valid when a single source is present, and if selected, the source is converted into a Monte Carlo uncollided flux source.



\section{[DEPLETION]}
This database contains ORIGEN depletion options. Each depletion step can be performed on each depletable region in parallel. Reaction rates are computed by Shift using an ultra-fine-group flux in the depletion regions. This fine flux is then used to collapse the CE cross sections into reaction rates that are sent to Origen to perform depletion. The default depletion solver is CRAM. By default, an ORIGEN JEFF multigroup file is used. 

\subsection{always\_transport \textit{boolean}}
By default, transport is run for each time step, even during decay periods. 

\subsection{burn\_length \textit{days}}
Burnup lengths for each burnup step. 

\subsection{calculate\_depletion\_energy \textit{boolean}}
By default, the energy released during depletion is calculated to give more accurate burnups.

\subsection{constant\_flux\_per\_step \textit{n/cm^2}}
You can specify a constant flux to be applied for each burnup step (not necessary).

\subsection{corrector\_substeps \textit{number}}
Number of depletion substeps on the corrector, if applicable. The default value of this parameter is specified based on the coupling method and solver. 

\subsection{coupling\_method \textit{method}}
This is the only parameter that should be specified for most applications. This parameter specifies how to solve coupled depletion-transport problems, specifically in how the cross sections to be used over each depletion step are selected (if assumed constant) or what interpolation procedure is done to make an estimate of their values over the step.

The default coupling method, \texttt{fully\_explicit} (also referred to as \texttt{ce}), results in for each time step, the beginning-of-step cross sections being used for the entire depletion step. More accurate methods can be used to interpolate cross sections based on previous time steps, to solve transport at the beginning and end of each time step, perform quadratic interpolation, etc. The interpolation methods tend to be more accurate, but are sensitive to large time steps. Pure decay steps are always solved with the fully explicit method, since this is the simplest method and is exact in the absence of neutron flux (which is assumed for decay steps).

\subsection{cram\_internal\_substeps \textit{number}\quad\quad\quad(\texttt{depletion\_solver cram})}
Number of internal substeps in the CRAM depletion solver. This is the minimum number of substeps that the CRAM depletion solver should use on each depletion step. Internal substeps are different from regular substeps in that they do not calculate different cross sections or renormalizations, and hence they are much faster than regular substeps. By default, two internal substeps are performed.

\subsection{cram\_order \textit{number}\quad\quad\quad(\texttt{depletion\_solver cram})}
By default, the order of the CRAM depletion solver is 16. 

\subsection{decay\_length \textit{days}}
Decay lengths for each input step.

\subsection{deplete\_cells \textit{cell names}}
List cell labels for depletion instead of depleting all fissionable cells, or use an asterisk to indicate that all cells should be depleted. By default, all depletable cells are tracked and depleted (depletable cells are defined as those containing fissionable materials, or if the [COMP] block is used, then those materials flagged as depletable). This parameter can override these default specifications.

\subsection{deplete\_nuclides \textit{nuclides}}
List nuclides to deplete, and then all cells containing these nuclides will be depleted. 

\subsection{depletion\_solver \textit{type}}
The depletion solver is either \texttt{cram} or \texttt{matrex}. 

\subsection{group\_bounds \textit{boundaries}}
Energy bin boundaries, in eV, for depletion tallies.

\subsection{jeff\_library \textit{path}}
Path to an ORIGEN JEFF multigroup file. 

\subsection{kappa\_library \textit{path}}
Path to an HDF5 file containing kappa values. 

\subsection{max\_burn\_substep\_size}
The maximum substep size for an ORIGEN time step (with non-zero flux) is by default 40 days. 

\subsection{max\_decay\_substep\_size}
The maximum substep size for an ORIGEN time step (with zero flux) is by default 75 days (longer than the case with non-zero flux).

\subsection{max\_step \textit{days}}
Maximum time step before automatically increasing the number of steps, which is by default 400 days.

\subsection{nuclide\_filter\_threshold \textit{threshold}}
Threshold at which nuclides below the threshold are removed from transport. 

\subsection{nuclide\_filter\_type \textit{type}}
Type of filter to be used to remove nuclides from transport. By default, no nuclides are removed, but they can be removed based on number density, absorption cross section, or total cross section.

\subsection{num\_burn\_steps \textit{number}}
Number of steps (constant-flux calculations) to take for each burn length entry. The flux is changed for each burn length (?). More transport calculations are required the more burn steps are requested.

\subsection{num\_decay\_steps \textit{number}}
Number of steps to take for each decay length entry. 

\subsection{origen\_library \textit{path}}
File path to an ORIGEN library file.

\subsection{power \textit{MW}}
Constant power to be applied during each burnup step. 

\subsection{predictor\_normalization \textit{boolean}}
By default, if Polaris coupling is used, then renormalization is performed on the predictor.

\subsection{predictor\_substeps \textit{number}}
This parameter specifies the number of depletion substeps to be used on the predictor. Substeps are always equidistant. The flux is renormalized at each substep using information for that substep. The number of substeps is directly related to depletion accuracy. The default value of this parameter depends on the depletion solver used.

\subsection{renormalization\_method \textit{method}}
This parameter describes how to renormalize the flux at each substep. The default is to use the energy method (fairly accurate) if using CRAM and moss if using MATREX.

\begin{tabular}{l l}
none & no normalization\\
boss & use beginning-of-substep cross sections and compositions (inaccurate!)\\
moss & use middle-of-substep cross sections and compositions\\
energy & renormalize based on energy released during depletion\\
origen & the power distribution is assumed to remain constant through the step\\
\end{tabular}

\subsection{reset\_inactive\_cycles \textit{number}}
Number of inactive cycles to run for all transport calculations except the initial calculation. By default, this parameter has a value of \(-1\).

\subsection{tracking\_nuclides \textit{list}}
Nuclides that are tracked for depletion. You can quickly add large numbers of nuclides with the \texttt{tracking\_set} parameter, and can individually add additional nuclides using this parameter. 

\subsection{tracking\_set \textit{set}}
This command appends a set of TRITON nuclides. The default is to add no extra nuclides to track, but setting \texttt{tracking\_set addnux1} adds some common and important lanthanides and actinides for tracking. Additional sets exist to quickly add other sets of nuclides. 

\subsection{yield\_library \textit{path}}
File path to an ORIGEN fission yields library file.

\subsection{[DEPLETION][MOVE \textit{name}]}
This sublist performs time-dependent geometry movements.

\subsubsection{delta \textit{distance}}
Distance to move surfaces at the beginning of each step. The number of movement steps should match depletion steps. Negative numbers indicate that something is move downwards, while positive numbers indicate it is moved upwards.

\subsubsection{name \textit{name}}
Name of the surface group to be moved.

\section{[HYBRID]}
This database is used for Monte Carlo acceleration methods.

\section{[MODEL=\textit{type}]}
This database contains the geometry and material representations. Exnihilo provides adapters for multiple models, which are problem definitions that include geometry, compositions, sources, and tallies. For most models, Shift allows cell volumes to be manually input with the \texttt{volumes} and \texttt{volume\_cells} keywords.

The only one of these that is not enabled is the type=trk? All of these require a \texttt{input ``filename.i''} line to indicate the path to the input file.\newline

\begin{tabular}{l l}
mcnp & MCNP5 model using Lava library\\
scale & KENO input file\\
gg & General Geometry XML specification\\
rtk & RTK XML specification\\
mesh & explicitly-meshed problem in HDF5 format\\
geant & GEANT4 GDML model input\\
sword & SWORD 6.0 input\\
dagmc & DAGMC CAD geometry definition (surfaces must be faceted)\\
\end{tabular}\newline

A GDML file is the Geometry Description Markup Language, which is an application-independent and flexible geometry format that is intended to interchange geometry between different applications. GDML is designed as an application of XML.  

XML is the Extensible Markup Language, which is used to create common information formats. This language contains markup symbols to describe the contents of a page or file. This is similar to HTML, except that HTML is primarily used for web applications, and describes how the content of a web page is to be displayed and interacted with, but not the actual content. XML is used to describe the actual content. XML is considered extensible, while HTML is not, because the markup symbols and unlimited and self-defining. XML is a subset of the Standard Generalized Markup Language (SGML) for how to create a document structure. 

All of the geometry implementations support redefining the volumes of cells except for Sword and Mesh.

\subsection{[MODEL=dagmc]}
Direct Accelerated Geometry Monte Carlo (DAGMC) is a MOAB package that has been integrated into Exnihilo to support CAD-based geometries. The model must be preprocessed so that the geometry is a faceted HDF5 file. Material compositions can be defined in the HDF5 geometry file or as a separate HDF5 file.

\subsection{[MODEL=geant]}
Exnihilo can read both the material and geometry from Geant4 inputs stored as GDML files. Tracking particles in Shift is not supported, and Geant4 can only be discretized in Denovo or ray-traced in Python.

\subsubsection{input \textit{path}}
Path to GEANT4 GDML input file.

\subsection{[MODEL=gg]}
General Geometry (GG) is the designation of the geometry engine underpinning the new SCALE geometry implementation. Geometry can be constructed explicitly in an Omnibus-style input which is translated to an XML file. This only supports geometry definitions - material IDs must be matched in the [COMP] block. 

\subsubsection{input}
This command generates a GG XML representation from a \texttt{.gg.omn} input file. 

\subsubsection{simplify\_max\_surfaces \textit{integer}}
This is the threshold for making a complex cell considered simple. A simple cell is a cell in which crossing a boundary always causes a particle to leave a cell. Some cells, however, have internal boundaries that can be crossed while still remaining in the cell. The tracking for simple and complex cells differs, and more effort is required for tracking in complex cells. A cell is assumed complex if it contains a positive shape in its definition. This definition then erroneously leads some simple geometries to be considered complex.

\subsubsection{xml\_path \textit{path}}
Path to the GG XML file. 

\subsection{[MODEL=mesh]}
Explicitly meshed problem in HDF5 format.

\subsubsection{input \textit{path}}
File path to hdf5 file.

\subsection{[MODEL=mcnp]}
Exnihilo can read MCNP input files through ORNL's Lava library, which was built to power ADVANTG. Lava is a C interface to MCNP routines, and it requires MCNP to generate a run tape file from an input file. The Omnibus front end automatically executes MCNP and generates the run tape. Any MCNP5 geometry is supported. In depletion calculations, groups of axis-aligned planes can be moved together at each time step without restart files. 

Tallies cannot be read into Shift, and MCNP sources cannot be used in domain-decomposed Shift, CADIS, or FW-CADIS modes.

\subsubsection{cell\_raytrace \textit{boolean}}
By default, cell labels are not turned into material IDs for raytracing.

\subsubsection{extents \textit{\(\pm x, \pm y, \pm z\)}}
This parameter specifies the bounding box for the active region of the geometry. Setting this model boundary enables certain Exnihilo features that require global boundaries, such as global mesh tallies and the global initial fission source.

\subsubsection{input}
This command generates an MCNP run tape file and sets the path to this file.

\subsubsection{mat\_names \textit{names}}
You can override names for the materials in the geometry. These customized names will be reflected in plots.

\subsubsection{runtpe\_path \textit{path}}
Path to the MCNP run tape file.

\subsection{[MODEL][MOVABLE=surface]\quad\quad\quad(\texttt{model = mcnp})}
In depletion calculations, groups of axis-aligned planes can be moved at each time step without restarting the calculation.

\subsection{[MODEL=rtk]}
RTK is an internal geometry engine used for PWR geometries within the VERA framework. Omnibus will read a geometry from an XML input file. The Insilico front-end should be used for creating reactors with RTK geometries.

\subsubsection{input \textit{path}}
File path to the RTK XML geometry file.

\subsection{[MODEL=scale]}
The geometry and composition definitions from a KENO-Va or KENO-VI file may be input to Exnihilo. The composition and geometry data are extracted directly from within the SCALE input. 

\subsubsection{input \textit{path}}

\subsection{[MODEL=sword]}

\section{[PHYSICS=\textit{type}]}
This sublist describes the physics treatments used. Omnibus supports two physics packages - multigroup (MG) and continuous energy (CE). 

\begin{tabular}{ll}
mg & multigroup physics (smg is an alias)\\
ce & continuous energy (solver is Shift) (sce is an alias)\\
void & all materials are replaced with void\\
\end{tabular}

The void option creates a one-group transport problem that is typically used in conjunction with visualization modes. 

\subsection{[PHYSICS=mg]}
The SCALE MG package as implemented in Exnihilo can use various multigroup cross section formats, including GIP and ANISN. The MG package by default uses SCALE to calculate infinite homogeneous medium cross sections for all materials. No self-shielding is used, so the results generally are poor, but the default behavior is mostly intended for generating deterministic solutions for hybrid problems. The MG physics allows users to manually input macroscopic cross sections for each material. The path to the SCALE MG cross sections must be provided. 

Like the CE case, problems can be run with neutrons only, photons only, or coupled neutron-photon problems.

MG depletion is not yet supported, and multigroup Shift is also not permitted. 

\subsubsection{mg\_lib \textit{library}}
The multigroup library must be given. 

\subsubsection{mode \textit{mode}}
The mode indicates the particles to transport, the default being \texttt{n} for neutrons with \texttt{mode kcode}.

\begin{tabular}{l l}
n, neutron & neutrons\\
p, photon & photons\\
np & \\
pn & \\
\end{tabular}

\subsection{[PHYSICS=ce]}
The CE (sometimes abbreviated as SCE to refer to SCALE CE) implementation in Shift is driven by AMPX-processed cross sections and collision data. Problems can be solved using neutrons only, photons only, or coupled neutron-photon problems. However, the photon cross sections do not include Bremsstrahlung reactions, so low-energy physics will not be modeled correctly.

The maximum neutron energy cutoff for cross sections is by default 20 MeV, while the minimum if 1e-5 eV. The default cutoff values for photons are 25 MeV and 10 keV. By default, cross section calculations for problems with fission are accelerated by pre-calculating the total and fission cross sections  in a material using a log-spaced energy grid. By default, no temperature correction is performed for the cross-section data. 

Only continuous-energy depletion is implemented, so depletion calculations cannot be performed with Denovo.

\subsubsection{ce\_lib \textit{library}}

Sets the CE library.

\subsubsection{ce\_lib\_path \textit{path}}

\subsubsection{mode \textit{mode}}

The mode indicates the particles to transport, the default being \texttt{n} for neutrons with \texttt{mode kcode}.

\begin{tabular}{l l}
n, neutron & neutrons\\
p, photon & photons\\
np & \\
pn & \\
\end{tabular}

\section{[POST]}
This database contains postprocessing options.

\section{[PROBLEM]}
This database specifies high-level information about the problem being run.

\subsection{name ``\textit{name}''}
The problem name should be a descriptive name, though this is not used in any file names.

\subsection{output \textit{path}}
Output path for the HDF5 file.

\subsection{mode \textit{mode}}
The \textit{mode} can be one of several values, depending on the type of calculation to be run. The fixed-source problem is referred to as a ``forward'' mode because the adjoint problem is often referred to as the ``backward'' transport equation.\newline

\begin{tabular}{l l}
kcode & solve the \(k\)-eigenvalue problem\\
forward & solve a fixed-source problem (for shielding)\\
adjoint & solve an adjoint fixed-source problem with the [SOURCE] or [TALLY] blocks interpreted as adjoint sources\\
raytrace & use Denovo ray tracer to generate voxelized materials for the problem. No transport is performed.\\
hybrid & run a forward transport problem in Shift using deterministic acceleration\\
\end{tabular}

\subsection{seed \textit{2272013}}
The seed is an integer that controls the random number generator, and is used in Shift for particle sourcing, but also within Denovo for various purposes, such as point source sampling. By default, the same seed is used for all problems unless changed.g

\section{[RESPONSE]}
This sublist contains tally responses. This is applicable when the solver is Shift or acceleration methods are being used (problem mode is adjoint or hybrid).

\section{[RUN]}
This database contains execution parameters. The Omnibus input file is run with \texttt{omnibus-run file.omn}, and parameters in this database control how that is performed. 

\subsection{[RUN=none]}
This doe not run the input, but only performs preprocessing. 

\subsection{[RUN=serial]}
Run on a single CPU core on the local machine. 

\subsection{[RUN=mpi]}
Run on multiple cores by directly calling MPI. This runs the input as an MPI process on the local machine. 

\subsubsection{np \textit{processors}}
Number of processors to run.

\subsection{[RUN=pbs]}
Run by submitting a PBS job. This creates a PBS run file. If you cancel the \texttt{omnibus-run} command, you still need to delete the PBS job using \texttt{qdel}. 

\subsubsection{account \textit{number}}
Account number to charge for time. 
\subsubsection{cpp \textit{cores}}
Number of cores per process. This is based on the request ppn and nodes. 
\subsubsection{nodes \textit{number}}
Number of nodes to use.
\subsubsection{pmem \textit{amount}}
Amount of memory per processor.
\subsubsection{ppn \textit{number}}
Number of processes per node. 
\subsubsection{qdel \textit{command}}
PBS deletion command (default is qdel).
\subsubsection{qsub \textit{command}}
PBS submission command. 
\subsubsection{walltime \textit{time}}
Wall time limit. 

\section{[SHIFT]}
This database contains Shift solver options. 

\subsection{do\_transport \textit{boolean}}
By default, the actual solve is performed. By setting this parameter to false, problem integrity can be verified without beginning a potentially very large solve. 

\subsection{num\_histories \textit{number}\quad\quad\quad(\texttt{mode forward})}

\subsection{num\_dd\_samples \textit{number}}
Number of test samples to determine initial particle balance. When using domain decomposition, an initial guess of the particle balance across blocks is performed before beginning the transport solve. Each domain randomly samples the source for this number of times, and the number of source particles found within the non-overlapping components of the domains is used to determine the number of particles emitted from the source in that domain. By default, this number is the number of histories divided by 10. 

\subsection{num\_histories \textit{number}}
Number of particle histories to run. This is applicable so long as the mode is not \texttt{kcode}.

\subsection{num\_response\_samples \textit{number}}
This is the number of samples to take when calculating source response. When calculating an approximate response for biasing sources, Shift samples a fixed number of points per source. The default is 100,000.

\subsection{[SHIFT][DECOMPOSITION=\textit{type}}
This database contains domain decomposition options that can be used for very large problems. If this database is not present, then the problem's spatial extents are used for the default partitioning. For no domain decomposition, the type is \texttt{none}. Otherwise, a boundary mesh is used for domain decomposition, and the type is \texttt{bmesh}. All the following parameters only apply when \textit{type}=\texttt{bmesh}. You should ensure that the Shift spatial decomposition is compatible with the number of processors if using a [RUN] block.

\subsubsection{boundary\_condition \textit{list}}
List the boundary conditions for each side of the boundary mesh. By default, all six sides are vacuum conditions, but reflecting, rotating, and periodic conditions are also accepted.

\subsubsection{overlap \textit{number}}
Fraction of domain decomposition domain overlap. By default, this is zero.

\subsubsection{x\_partition \textit{list}}
List of x-values (at least two) denoting the partitioning along the x-axis. Similarly, the boundary mesh in the y and z directions is permitted.

\subsection{[SHIFT][KCODE]\quad\quad\quad(\texttt{mode kcode})}
This database contains eigenvalue solution options. Using a Monte Carlo code to solve for the eigenvalue involves iteratively sampling and transporting generations of fission neutrons. 

\subsubsection{convergence\_method \textit{type}}
Shift supports different convergence criteria for switching from inactive to active cycles and for completion of the eigenvalue solve. The inactive cycles are intended to allow the fission source distribution to converge before tallying begins. The number of inactive cycles is specified by the user. If the type of convergence method is set to \texttt{count}, the default, then the user also specifies the number of active cycles. This should be used if a certain number of particles need to be run before finishing the problem. Alternatively, the eigenvalue solve can be stopped after reaching a particular standard deviation in the eigenvalue.

\subsubsection{initial\_keff \textit{value}}
Initial value for \(k\), which is by default 1.0.

\subsubsection{num\_cycles \textit{number}}
Number of active and inactive kcode cycles. 

\subsubsection{num\_histories\_per\_cycle \textit{number}}

\subsubsection{num\_inactive\_cycles \textit{number}}
Number of inactive cycles.

\subsubsection{x\_entropy \textit{list}}
Boundaries in \(x\) for the Shannon entropy mesh. Similarly, boundaries are defined in the y and z directions.

\subsubsection{[SHIFT][KCODE][ACCELERATION]}
Source acceleration method. Either no acceleration can be performed, or the Kernel Density Estimator acceleration of fission sites can be used (\texttt{type=kde}).

\subsection{[SHIFT][TRANSPORTER]}
This database contains transport communication options.

\subsubsection{check\_frequency \textit{number}\quad\quad\quad(\texttt{method async})}
Check frequency for domain decomposition completion. 

\subsubsection{max\_lost\_count \textit{number}}
Maximum number of particles that can be lost on a single decomposed domain.

\subsubsection{method \textit{type}}
Domain decomposition transport method, which is either \texttt{sync} or \texttt{async}.

\subsubsection{particle\_buffer\_size \textit{number}\quad\quad\quad(\texttt{method async})}
Size of particle buffer for DD problem, which by default is 1000. 

\subsection{[SHIFT][TRANSPORTER][GENERATIONS]]\quad\quad\quad(\texttt{method sync})}
Synchronous DD control.

\subsection{[SHIFT][VR]}
This database contains variance reduction options.

\subsubsection{method \textit{type}}
Weight windows (\texttt{ww}) are used when the mode is hybrid, and \texttt{roulette} when otherwise. To be fully explicit, another method is technically \texttt{analog}.

\subsubsection{weight\_cutoff \textit{weight}}
Particle weight cutoff for rouletting, which by default is 0.25. This is applicable when using rouletting or weight windows.

\subsubsection{weight\_survival \textit{weight}}
Particle weight survival for rouletting, which by default is 0.5. This is applicable when using rouletting.

\subsubsection{ww\_lower\_factor \textit{number}}
Lower weight window ratio, which by default is 0.5. Likewise, the upper weight window ratio, default of 2.5, can be specified.

\section{[SOURCE=\textit{type}]}
This database defines the source particle distribution for Shift, a fixed-source problem, or the starting source for an eigenvalue problem. This is applicable when using Shift or running a fixed-source problem. The total strength of all sources is used as a global multiplier for all tallies in fixed-source mode (but because criticality calculations are eigenvalue problems, only the relative strengths of the sources are used in \texttt{kcode}). 

By default, all sources use the Watt energy spectrum when in \texttt{kcode} mode. Energy and particles can be defined to only be emitted in a particular region or material.\newline

\begin{tabular}{l l}
fission-mesh & fixed fission source from a Shift tally or mesh source\\
mesh & discretized source from the mesh model HDF5 file\\
mcnp & source from MCNP SDEF cards\\
sword & SWORD source/spectra definitions\\
material & volumetric material composition emission\\
surface & emit particles from a pre-computed surface source\\
sourcerer & use Denovo solution as a fission source (for \texttt{kcode} mode only)\\
\end{tabular}

\subsection{[SOURCE=separable \textit{name}]}
This source is separable in space, energy, and angle.

\subsubsection{allow\_biasing \textit{boolean}}
By default, source biasing for hybrid modes is not enabled.

\subsubsection{description ``\textit{description}''}

\subsubsection{fissionable\_only \textit{boolean}}
By default, particles are emitted only in fissionable regions for \texttt{kcode} mode. 

\subsubsection{material\_only \textit{materials}}
Emit particles only in a given material name.

\subsubsection{name \textit{name}}

\subsubsection{num\_rejection\_samples \textit{number}} 
By default, 1000 samples are attempted before declaring that a particle is lost. This is only applicable when \texttt{fissionable\_only} is true (the default) or if \texttt{material\_only} is defined. 

\subsubsection{strength \textit{particles/s}} 
Strength of the source, by default 1 particle/second.

\subsection{[SOURCE][ANGLE=isotropic]}
By default, all sources are isotropic. For Denovo calculations, the only allowable angular discretization is isotropic. 

\subsection{[SOURCE][ANGLE=mono]}
Monodirectional source.

\subsubsection{direction \textit{length-3 vector}}

\subsubsection{[SOURCE][ENERGY=histogram]}

\subsubsubsection{energy \textit{bins}}
List the energy bin bounds in eV to define a custom histogram energy source.

\subsubsubsection{particle\_type \textit{type}}
A source can emit neutrons or photons. 

\subsubsubsection{probability \textit{probability bins}}
Probability of each bin being selected.

\subsubsection{[SOURCE][ENERGY=mono]}
Monoenergetic line distribution.

\subsubsubsection{energy \textit{source energy}}
Source energy.

\subsubsubsection{particle\_type \textit{type}}
A source can emit neutrons or photons. 

\subsubsection{[SOURCE][ENERGY=lines]}
Multiple line energy distribution.

\subsubsubsection{energy \textit{energies}}
Individual line energies. 

\subsubsubsection{particle\_type \textit{type}}
A source can emit neutrons or photons. 

\subsubsubsection{probability \textit{probabilities}}
Probability of each line being selected.

\subsubsection{[SOURCE][ENERGY=watt]}
The Watt fission energy spectrum uses an exponential-hyperbolic sine function to approximate the fission neutron energy spectrum, modified by a normalization constant. The user can enter their own constants for the correlation in order to tailer for difference nuclides, but the default spectrum is for U-235. This energy spectrum is the default when the problem mode is \texttt{kcode}.

\subsubsubsection{a}
\subsubsubsection{b}
\(a\) and \(b\) are parameters in the Watt equation.

\subsubsubsection{nuclide \textit{nuclide}}
Alternatively to custom-definitions, providing a nuclide name will give the \(a\), \(b\) that correspond to that nuclide.

\subsubsection{[SOURCE][ENERGY=origen]}
This option reads photon spectra out of an Origen master library to provide decay gamma spectrum.

\subsubsubsection{filename \textit{file}}
File containing the ORIGEN-generated spectrum.

\subsection{[SOURCE][SHAPE=\textit{type}]}
This option allows the definition of a uniform spatial distribution, defined over a particular volume. The global source is a computationally expensive method to sample particles inside the geometry. Points are sampled uniformly within the global bounding box.\newline

\begin{tabular}{l l}
box & axis-aligned cuboid\\
cyl & axis-aligned cylinder\\
cylshell & cylinder shell shape\\
sphere & sphere\\
sphereshell & spherical shell shape\\
point & single point\\
multipoint & multiple points\\
global & box covering the entire geometry\\
\end{tabular}



\subsection{[SOURCE=fissionmesh]}

This option uses a fixed fission source from a Shift tally or mesh source. This allows the fission neutron distribution from a mesh tally or from a previous run to be used as a source. This source can either be a starting source in a \texttt{kcode} calculation to reduce the number of inactive cycles required, or a fixed source so that a combined neutron-gamma problem can be run separately from the criticality calculation. 

\subsubsection{allow\_biasing}

By default, source biasing for hybrid problem modes is not used.


\section{[TALLY]}
This database defines tallies. This is applicable when the solver is Shift or acceleration methods are being used (problem mode is adjoint or hybrid). Unlike most other Monte Carlo codes, the energy boundaries in Exnihilo are truly boundaries, and the bounds given bound the energy range (the energy is not implied to be defined by an upper bound and all energies greater than that energy, for instance). 

Cell and mesh tallies produce volume-averaged quantities if the volumes are available for the cells being tallied. All reaction rates are output as reaction rate densities. 

\subsection{[TALLY][CELL]}
Cell path length tally.

\subsubsection{cells \textit{cells}}
Generate \texttt{union\_cells} and \texttt{union\_lengths} from colon-separated unions. 

\subsubsection{cycles \textit{active/inactive}}
By default, tallies are only tallied during the active portion of a \texttt{kcode} problem, though you can also tally during the inactive portion. 

\subsubsection{description ``\textit{string}''}

\subsubsection{name ``\textit{name}''}

\subsubsection{neutron\_bins \textit{bins}}
Neutron energy bin boundaries, in eV, listed in decreasing order.

\subsubsection{normalization \textit{number}}
Constant multiplicative factor to apply to the tally results. 

\subsubsection{photon\_bins \textit{bins}}
Photon energy bin boundaries, in eV, listed in decreasing order.

\subsubsection{reactions \textit{type}}
Reactions to calculate for this tally. By default, the flux is calculation, but the total, absorption, scattering, fission, \(\nu\)-fission, kappa-sigma, or kerma can also be calculated.

\subsubsection{responses \textit{strings}}
Responses for this tally. This must agree with the [RESPONSES] block. 

\subsubsection{[TALLY][CELL][FILTERS]}
Tally filtering options.

\subsubsection{cycles \textit{active/inactive}}
By default, tallies are only tallied during the active portion of a \texttt{kcode} problem, though you can also tally during the inactive portion. 

\subsubsection{description ``\textit{string}''}

\subsubsection{name ``\textit{name}''}

\subsubsection{neutron\_bins \textit{bins}}
Neutron energy bin boundaries, in eV, listed in decreasing order.

\subsubsection{normalization \textit{number}}
Constant multiplicative factor to apply to the tally results. 

\subsubsection{photon\_bins \textit{bins}}
Photon energy bin boundaries, in eV, listed in decreasing order.

\subsubsection{r \textit{coordinates}}
Radial mesh coordinates, in cm. List a monotonically increasing list with at least two values. 

\subsubsection{reactions \textit{type}}
Reactions to calculate for this tally. By default, the flux is calculation, but the total, absorption, scattering, fission, \(\nu\)-fission, kappa-sigma, or kerma can also be calculated.

\subsubsection{responses \textit{strings}}
Responses for this tally. This must agree with the [RESPONSES] block. 

\subsubsection{rotate \textit{rotation matrix}}
Length-9 row-major rotation matrix. By default, this is the identity matrix and no rotation is performed.

\subsubsection{theta \textit{coordinates}}
Theta mesh coordinates, in units of revolutions. By default, the entire range of \(\theta\) is tallied. List as a monotonically increasing list with at least two values.

\subsubsection{translate \textit{vector}}
Translation vector, to be applied after rotation. 

\subsubsection{z \textit{coordinates}}
Mesh coordinate along the \(z\)-axis, in cm. List as a monotonically increasing list with at least two values. 

\subsection{[TALLY][CYLMESH]}
Cylindrical mesh tally. 

\subsection{[TALLY][DIAGNOSTIC]}
Singleton tallies used for transport diagnostic purposes. This is only applicable when the solver is Shift.

\subsection{[TALLY][MESH]}
Cartesian mesh tally. 

\subsection{[TALLY][SENSITIVITY]}
Sensitivity/uncertainty tally. This is applicable when the CE physics is used. 

\subsection{[TALLY][SHADOW]}
Path length cell tallies inside an overlaid but non-interacting geometry definition. 

\subsection{[TALLY][SWORD]}
Tallies from SWORD input. 









\end{document}