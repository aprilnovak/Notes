\section{Reactivity Control}
\label{sec:Reactivity}

The ability to exercise control over the reactivity of a nuclear system is essential for safe operation of the system. This section describes how the neutron transport solution depends on the \gls{th} state; this inherent dependence is usually characterized with ``reactivity coefficients,'' described in Section \ref{sec:ReactivityCoefficients}. External effects, such as movement of control rods, is usually characterized according to ``worth.''

\subsection{Reactivity Coefficients}
\label{sec:ReactivityCoefficients}

A reactivity coefficient \(\alpha\) is defined based on the relative change in reactivity given a change in some input parameter \(x\):

\beqa
\alpha_x\equiv&\frac{\partial\rho}{\partial x}\\
=&\frac{\partial}{\partial x}\left(\frac{k-1}{k}\right)\\
=&\frac{1}{k^2}\frac{\partial k}{\partial x}
\eeqa

By the chain rule, each reactivity coefficient can also be expressed in terms of the temperature reactivity coefficients:

\beq
\alpha_x=\sum_i\frac{\partial\rho}{\partial T_i}\frac{\partial T_i}{\partial x}
\eeq

where \(i\) is the number of regions considered in the definition of the reactivity coefficient. A reactivity ``defect'' is the cumulative reactivity required to move between some initial state and some final state, accounting for the fact that the reactivity coefficients are not constant. The required excess reactivity at \gls{hfp} is frequently defined in terms of the reactivity needed to sustain a significant drop in power, which would initially reduce burnout of Xe-135 to the point that some excess positive reactivity is required \cite{fratoni}. Power defects are on the order of 1000-3000 pcm in \glspl{lwr} \cite{duderstadt}.

It is generally desirable to have entirely negative reactivity coefficients, but it may not be conservative to have extremely negative coefficients. For example, if the coolant reactivity coefficients are very negative, then an overcooling transient could induce a very rapid increase in power. The following sections describe the most important reactivity coefficients in more detail and give typical values for reactor systems.

\subsubsection{Power Reactivity Coefficient}
Of all the definitions for reactivity coefficients, the only coefficient that {\it must} be negative to ensure safe power operation is the power reactivity coefficient. If this coefficient were positive, then the slightest increase in power would result in a runaway power excursion. Despite the importance of this coefficient, because the power depends on thermal considerations such as the flowrate, determination of the power reactivity coefficient based solely on a neutronics calculation is nearly impossible. Therefore, other reactivity coefficients comprise the bulk of the reactivity control investigations performed. Power coefficients are typically on the order of 30-60 pcm/\% power.

\subsubsection{Fuel Temperature Coefficient}
The fuel temperature coefficient, also referred to as ``Doppler coefficient,'' typically has the smallest magnitude of the important reactivity feedback mechanisms in reactor systems. However, the fuel temperature is the first mechanism to respond to a \gls{ria}, and having a negative coefficient is especially important to reactor control in these fast-acting transients and when the prompt neutron lifetime is very small, such as in fast spectrum systems.

In a spatially self-shielded system, the fuel is separated from the moderator. Combined with resonance cross sections, neutrons are more likely to be absorbed in resonances near the outer periphery of the fuel, which results in a energy-dependent flux depression within the central region of the fuel. This effect is known as ``energy self-shielding.'' An increase in fuel temperature increases Doppler broadening, which reduces the effectiveness of energy self-shielding. The energy-dependent flux corresponding to a resonance increases within the central region of the fuel upon Doppler broadening - in essence, the energy-dependent flux penetrates further into the fuel, resulting in an increase in reaction rates at that energy in the central region of the fuel. If the energy spectrum is relatively thermal such that the spectrum only spans capture resonances in fertile materials, a reduction in energy self-shielding results in a greater fraction of the fuel being exposed to epithermal neutrons, and hence an increase in parasitic absorption by fertile isotopes such as U-238 and Pu-240 and some fission products. This yields a negative fuel temperature coefficient. %For reactors with very soft spectra, perturbation theory predicts that the Doppler coefficient is inversely proportional to the square root of the absolute temperature.

On the other hand, if the energy spectrum is relatively fast such that the spectrum spans capture resonances in both fertile and fissile materials, a reduction in energy self-shielding results in an increase in both parasitic capture in fertile materials and fission in fissile materials. For fast systems, the fuel temperature coefficient could therefore be positive if the relative increase in fission is greater than the relative increase in parasitic capture. This increased fission rate (for fast spectrum systems) is most prevalent in Pu-239, so the greater the concentration of Pu-239, the more positive the fuel temperature coefficient. In heterogeneous fast spectrum systems, the fertile material is typically located in the lower flux region to breed fissile material, and thus the flux experienced by the fissile material is higher than that experienced by the fertile material, producing a positive Doppler coefficient. Homogeneous systems more intimately mix the fissile and fertile materials, helping to achieve a negative Doppler coefficient by exposing both the fertile and fissile isotopes to nearly the same flux, allowing cancellation of positive and negative reactivity insertion associated with changes in fission and parasitic absorption rates. Such mixing can be effectively achieved using \gls{mox} fuels. 

Techniques for obtaining a negative Doppler coefficient in fast spectrum systems attempt to soften the spectrum such that increases in the epithermal flux in the fuel results in a greater increase in parasitic capture than fission. This is sometimes achieved through the use of a BeO moderator or designing for larger-diameter cores to reduce leakage. The Doppler coefficient can me made more negative in a number of ways for thermal spectrum systems:

\begin{itemize}
\item Harden the spectrum (for very thermal systems) such that a greater fraction of the energy-dependent flux is in the epithermal range where Doppler broadening leads to a net increase in parasitic absorption. This can be achieved by reducing the moderator to fuel ratio \cite{fratoni}.
\item Reduce the effects of spatial self-shielding so that a reduction in energy self-shielding accompanying Doppler broadening results in full exposure of the fuel to epithermal flux. Spatial self-shielding effects can be reduced by using smaller fuel dimensions with larger surface to volume ratios \cite{fratoni} or other changes such as increasing the \gls{mfp} \cite{duderstadt}.
\end{itemize}

%Other changes affecting the average spectrum, such as a loss of sodium in an \gls{sfr}, will affect the Doppler coefficient as well.

The fuel temperature coefficient is sensitive to the fuel composition, and must be evaluated as a function of burnup. In low-enriched thermal systems, Pu-240 has a wider resonance than U-238, and despite depletion of U-238, the Doppler coefficient becomes more negative with burnup as build in of Pu-240 occurs \cite{duderstadt,fratoni}. For all systems, the Doppler coefficient decreases with temperature as the material experiences increased lattice vibrations that reduces the possibility of achieving very large relative velocities between the neutron and the nucleus, essentially reducing the effective broadening experience by the neutron. Therefore, some test reactors may choose to operate at lower temperatures to obtain a more negative Doppler coefficient.

The Doppler coefficient is insensitive to the control rod pattern in the core provided the control rods are designed to absorb neutrons in the thermal spectrum, as Doppler broadening impacts epithermal neutron absorption. Typical fuel temperature coefficients are on the order of -4 to -1 pcm/K for \glspl{lwr}, -7 pcm/K for \glspl{htgr}, and -2.5 to -0.6 pcm/K for \glspl{lmfbr} \cite{duderstadt}. The fuel temperature coefficient has been estimated at -3.8 pcm/K for the \gls{pbfhr} \cite{xin_wang_thesis}.

\subsubsection{Moderator Temperature Coefficient}
An increase in moderator temperature, neglecting the corresponding density change, results in spectrum hardening. For fast systems, an increase in the energy spectrum results in an increase in \(\eta\) and a positive coefficient, though the effect is more complicated for thermal systems due to the resonant-like structure in the epithermal energy range. The spectrum change in thermal systems typically results in a decreased \(\eta\) due to increased absorption in low-lying resonances \cite{duderstadt}. However, the sign of this effect depends on the fuel composition, since cross sections fall off fastest for U-235, then U-238, and then Pu-239. If the reactor contains a significant amount of Pu-239, a hardened spectrum results in a greater fraction of the fission occurring in Pu-239, which has a higher number of neutrons released per fission event, which can lead to a slightly positive moderator temperature coefficient in systems with significant quantities of Pu-239. 

%The sign of the moderator temperature coefficient also depends on the size of the core; if the systems has significant leakage, a hardened spectrum may still produce a negative feedback, though this feedback could be positive for large cores.

%In addition, cross sections for fertile materials often display thresholds that, with a hardened spectrum, may result in increased fertile absorption. 

%For graphite-moderated systems, however, a hardened spectrum leads to a net increase in \(\eta\), producing a slightly positive moderator temperature coefficient. 

Typical moderator temperature coefficients are on the order of -50 to -8 pcm/K for \glspl{lwr} and 1 pcm/K for \glspl{htgr} \cite{duderstadt}. 

\subsubsection{Coolant Density Coefficient}
\label{sec:CoolantDensityCoeff}
The moderator density coefficient represents changes in reactivity due to reduced density of the moderator. The sign of the moderator density coefficient for infinite, thermal systems depends on whether the system is over- or under-moderated. For example, in a system containing only fuel, no moderation occurs such that \(p=0\), while in a system containing only moderator, the fuel utilization factor is zero such that no neutrons are absorbed in fuel. When these effects are combined, a range in available operating states exists, where the reactor is classified as ``under-moderated'' in the range where the resonance escape probability response is stronger than the thermal utilization response, and as ``over-moderated'' in the reverse case. A reduction in coolant density in an under-moderated core results in a decrease in reactivity. A desire to have a negative moderator density coefficient results in limits on the soluble boron concentration in \glspl{pwr} and the Li-6 concentration in the \gls{pbfhr}. 

The sign of the coefficient may depend on location within the system once leakage is considered. For example, in the center of large \glspl{lmfbr}, the void coefficient may be positive in the center as the resonance escape probability decreases - a positive insertion in fast spectrum systems. However, the coefficient may be negative at the edge of the core where the leakage increases. 

%The coolant in \glspl{bwr} is typically at saturated conditions, the MTC is relatively unimportant in normal operation, but has the greatest impact during startup and shutdown processes when there are no voids in the coolant. In normal operation, any power increases are curbed much more by the void coefficient, and hence in BWRs it is acceptable to have a positive MTC.

As fuel depletes, the reactor becomes closer to an over-moderated core, so the magnitude of the moderator density coefficient typically decreases with burnup (provided the moderator to fuel ratio is not held constant through the process of online-refueling). In addition, later in life control rods are typically removed to compensate for depletion, which allows coolant to fill the vacated space, resulting in a less negative coefficient.

\subsubsection{Coolant Void Coefficient}

Under 100\% loss of coolant, the reactivity response is characterized by the coolant void coefficient. This coefficient is essentially an extension to the coolant density coefficient described in Section \ref{sec:CoolantDensityCoeff}, and follows similar trends. The void coefficient is especially large in \glspl{bwr}, being on the order of -200 to -100 pcm/K \cite{duderstadt}. 

The coolant void coefficient is in some cases also used to refer to voiding of the coolant, or bubble formation due to phase change or entrained gas. In \glspl{bwr}, it is possible to observe a power increase upon removal of a shallow control rod; local boiling due to increased reactivity produces bubbles that advect to higher-importance regions of the core at higher elevations, inserting sufficient negative reactivity to compensate for the positive insertion associated with removal of the control rod.