\section{The Continuous-Energy Neutron Transport Equation}
\label{sec:CE_NTE}

The \gls{nte} is a statement of the conservation of neutron density \(n\seat\), the {\it expected} number of neutrons at a point in space per unit time \(t\), energy \(E\), and (unit) direction of motion \(\hO\). The {\it true} number of neutrons at a point in space cannot be known, since statistical processes govern the exact distribution of neutrons, and we will be limited to predicting mean or average neutron populations. The neutron density is a function of a seven-dimensional space - spatial (three), energy (one), direction of motion described as two angles within and relative to one coordinate plane, and time (one). The energy of the neutron correspond to the kinetic energy:

\beq
E\equiv\frac{1}{2}mv^2
\eeq

where \(v\equiv\|\vv{v}\|_2\) is used throughout. Because neutron kinetic energies are typically very high, units of electron volts for energy are assumed throughout, rather than the more conventional unit of Joules. The unit direction of motion \(\hO\) is a unit vector in the same direction as the neutron velocity \(\vv{v}\):

\beq
\label{eq:OmegaDef}
\hO\equiv \frac{\vv{v}}{v}
\eeq

The direction of motion of a {\it fluid} particle is typically expressed using ``velocity,'' with one vector component corresponding to each spatial dimension. However, this can equivalently be expressed in terms of the kinetic energy \(E\) and two angles \(0\leq\phi\leq2\pi\) and \(0\leq\theta\leq\pi\), where \(\phi\) is the angle in the \(x\)-\(y\) plane from the positive \(x\)-axis and \(\theta\) is the angle between the flight direction and the \(z\)-axis. Like velocity, \(\hO\) has three components corresponding to the three spatial dimensions, where each component is only a function of the two angles \(\phi\) and \(\theta\). Note that using \(\hO\) instead of velocity does not reduce the number of unknowns - instead of working with three velocity components, we work with one energy component and two angular components. This decomposition is useful for tabulating scattering cross sections independently as a function of energy and incoming and outgoing directions, rather than jointly in terms of velocity.

\begin{tcolorbox}[breakable]
In Cartesian coordinates, \(\hO\) is given in terms of unit vectors \(\hat{i}\), \(\hat{j}\), and \(\hat{k}\) corresponding to the \(x\), \(y\), and \(z\) axes, respectively, as:

\beqa
\label{eq:OmegaCartesian}
\hO  = \sin{(\theta)}\cos{(\phi)}\hat{i} + \sin{(\theta)}\sin{(\phi)}\hat{j} + \cos{(\theta)}\hat{k}\\
\eeqa

Each of these components are often given their own names: 

\beq
\label{eq:xiDef}
 \xi \equiv \sin{(\theta)}\cos{(\phi)}
\eeq

\beq
\label{eq:etaDef}
 \eta \equiv \sin{(\theta)}\sin{(\phi)}
\eeq

\beq
\label{eq:muDef}
 \mu \equiv \cos{(\theta)}
 \eeq
 
 Using the \(\sin^2{(\theta)} + \cos^2{(\theta)} = 1\) identity, Eq. \eqref{eq:OmegaCartesian} can also be written as:
 
\beq
\hO=\sqrt{1-\mu^2}\textrm{cos}(\phi)\hat{i} + \sqrt{1-\mu^2}\textrm{sin}(\phi)\hat{j} + \mu\hat{k}\\
\eeq
 
Due to the very simple form of the component of \(\hO\) along the \(z\)-axis, one-dimensional Cartesian problems are frequently reoriented such that \(z\) is the independent variable.\newline

In a spherical coordinate system, an integral over the surface of the unit sphere is given as \(dA=r^2\sin{(\theta)}d\theta d\phi\), so an integral over the solid angle is determined by dividing by \(r^2\) since solid angle represents the dimensionless area on the unit sphere:

\beqa
\label{eq:DifferentialOmega}
d\hO  \equiv&\sin{(\theta)}d\theta d\phi\\
\equiv&d\mu d\phi
\eeqa

From the definition of \(\mu\) in Eq. \eqref{eq:muDef}, \(d\mu=d(\cos{(\theta)})=\sin{(\theta)}d\theta\), giving the second equivalent form in Eq. \eqref{eq:DifferentialOmega}. An integral over solid angle is represented shorthand by \(\int_{4\pi}^{ } d\hO\), and is equivalent to the integral of Eq. \eqref{eq:DifferentialOmega}:

\beqa
\label{eq:SolidAngleIntegration}
\int_{4\pi}d\hO\equiv&\ \int_{-1}^1d\mu\int_0^{2\pi}d\phi\\
=&\ 4\pi
\eeqa

The integral of any single component of \(\hO\) defined in Eq. \eqref{eq:OmegaCartesian} over solid angle is zero:

\beq
\label{eq:OmegaCartesianIntegration}
\int_{4\pi}d\hO\hO=0
\eeq

The integral of any two components of \(\hO\) defined in Eq. \eqref{eq:OmegaCartesian} over solid angle gives the following orthogonality condition:

\beq
\label{eq:4PiOmegaOmega}
\int_{4\pi}^{ } d\hO\ \Om_i\Om_j = \frac{4\pi}{3}\delta_{ij}
\eeq

The integral of any three components of \(\hO\) defined in Eq. \eqref{eq:OmegaCartesian} over solid angle is zero:

\beq
\label{eq:4PiOmegaOmegaOmega}
\int_{4\pi}d\hO \Om_i\Om_j\Om_k=0
\eeq
 
\end{tcolorbox}

 Because the neutron population is conserved, the rate of change of the neutron population within an arbitrary convex control volume \(\volume(t)\) consisting must equal the sum of any gains and losses of neutrons:

\beq
\label{eq:neutron_conservation}
\left\lbrack\frac{\partial}{\partial t}\int d\volume n\seat\right\rbrack dEd\hO=\text{gains}-\text{losses}
\eeq

For those familiar with fluid mechanics, this derivation immediately begins from the control volume perspective, rather than the system perspective, eliminating the need to use the Reynolds Transport Theorem to convert a system perspective to a control volume perspective. 

The volume must be convex to eliminate any boundary sources of neutrons that arise due to leakage from one portion of the volume to another across the boundary. Because the neutron density is in general a function of \(E\) and \(\hO\), a correct statement of conservation requires all volume integrals to be multiplied by the differential \(dEd\hO\) to represent integration over all energies and angles (though integration will not be shown, but implicitly assumed to simplify notation). If the volume is assumed to be independent of time, the time integration can be brought inside the integral such that the time rate of change term becomes:

\beq
\left\lbrack\int d\volume \frac{\partial n\seat}{\partial t}\right\rbrack dEd\hO
\eeq

Sources of neutrons exist due to external sources \(S\) independent of the neutron density; scattering from other energies and angles; and fission. The external source contribution to Eq. \eqref{eq:neutron_conservation} is:

\beq
\label{eq:ConservationExternalSource}
\left\lbrack\int d\volume \source \right\rbrack dEd\hO  
\eeq

The scattering of neutrons from other energies and angles to \(E\) and \(\hO\), also known as the in-scattering source, is an integral of the scattering reaction rate density over all neutron energies \(E'\) and angles \(\hO'\):

\beq
\label{eq:ConservationInscatteringSource}
\left\lbrack\int d\volume\inscatteringsource v(E')n\seatelse\right\rbrack dEd\hO  
\eeq

Note that Eq. \eqref{eq:ConservationInscatteringSource} includes ``sources'' of neutrons due to scattering from \(E\rightarrow E\) and \(\hO\rightarrow\hO\), which should not be represented as either a source or a loss. This double-counting will be cancelled later.

The fission source is expressed as the sum of a prompt and delayed source of neutrons. The neutrons born nearly instantaneously following fission from the fragmentation of the compound nucleus are considered ``prompt'' neutrons. Delayed by up to several hundred seconds is another source of neutrons from the decay of fission products. Rather than consider the build up and decay of fission products with the Batemann equations on time scales of seconds to capture this delayed source of neutrons, these ``delayed'' neutrons are treated as a time-lagged fission source of neutrons. The fraction of fission neutrons born delayed is \(\beta\), the delayed neutron fraction:

\beq
\label{eq:BetaDef}
\beta\equiv\frac{\text{fission neutrons born delayed}}{\text{fission neutrons born promptly and delayed}}
\eeq

While it is not possible to change the source of prompt neutrons by removing compound nuclei from the control volume before decay (such time scales are much smaller than nanoseconds), it is possible to remove the fission products whose decay produces neutrons. Writing the total fission source as the sum of a prompt and delayed source permits flexibility in transient analysis of nuclear systems. The prompt fission source is written as an integral over all neutron energies and angles of the fission reaction rate density:

\beq
\label{eq:ConservationFissionPrompt}
\left\lbrack\int d\volume\promptfissionsource v(E')n\seatelse\right\rbrack dEd\hO
\eeq

where \(\chi_p(E,\hO)\) is the prompt fission neutron energy and angle spectrum and \(\nu\) is the average number of neutrons born in fission, considering both prompt and delayed neutrons. If prompt neutrons are born nearly isotropically (which is frequently a good approximation), angle integration of the prompt fission neutron energy and angle spectrum yields only an energy dependence:

\beq
\label{eq:prompt_isotropic}
\int_{4\pi}d\hO \chi_p(E,\hO)=\frac{\chi_p(E)}{4\pi}
\eeq

\(\chi_p(E,\hO)\) is frequently replaced by \(\chi_p(E)/4\pi\) in Eq. \eqref{eq:ConservationFissionPrompt} using this assumption. The delayed neutron source cannot be written simply as an integral of a reaction rate density because interaction probabilities for the production and decay of fission product nuclides that result in neutron production are not simply functions of physical parameters. The movement of fission products impacts the production of delayed neutrons within the control volume. The delayed fission source is frequently approximated as an integral over a finite sum of \(J\) grouped fission product precursor concentrations:

\beq
\label{eq:ConservationFissionDelayed}
\left\lbrack\int d\volume \delayedfissionsource\right\rbrack dEd\hO  
\eeq

where \(\chi_d(E,\hO)\) is the delayed neutron energy and angle spectrum, \(C\) is the delayed neutron precursor concentration, and \(\lambda\) is the delayed neutron precursor decay constant. If \(J\) is taken equal to the total number of fission products whose decay results in a neutron, Eq. \eqref{eq:ConservationFissionDelayed} is exact. However, \(J\) is frequently approximated as 6. Similar to prompt fission neutrons, delayed neutrons are frequently approximated as being born isotropically, and angle integration of the delayed neutron energy and angled spectrum yields only an energy dependence:

\beq
\label{eq:delayed_isotropic}
\int_{4\pi}d\hO \chi_d(E,\hO)=\frac{\chi_d(E)}{4\pi}
\eeq

\(\chi_d(E,\hO)\) is frequently replaced by \(\chi_d(E)/4\pi\) in Eq. \eqref{eq:ConservationFissionDelayed} using this assumption. Losses of neutrons exist due to absorption and scattering to other energy groups. The absorption loss is represented by an integral in space of the absorption reaction rate density:

\beq
\label{eq:AbsorptionLoss}
\left\lbrack\int d\volume \Sigma_a\seat v(E)n\seat\right\rbrack dEd\hO
\eeq

The scattering to other energy groups, referred to as ``out-scattering,'' is represented as an integral of the scattering reaction rate density:

\beq
\label{eq:ScatteringLoss}
\left\lbrack\int d\volume \Sigma_s\seatout v(E)n\seat\right\rbrack dEd\hO
\eeq

Note that Eq. \eqref{eq:ScatteringLoss} includes scattering from \(E\rightarrow E\), which should not be represented as a loss term. This double-counting is balanced by the same double-counting in the in-scattering term in Eq. \eqref{eq:ConservationInscatteringSource}, giving the correct conservation statement. Eqs. \eqref{eq:AbsorptionLoss} and \eqref{eq:ScatteringLoss} are frequently combined by defining the total cross section \(\Sigma_t\) as:

\beq
\label{eq:TotalSigmaDef}
\Sigma_t\seatout\equiv\Sigma_a\seat+\Sigma_s\seatout
\eeq

Finally, streaming of neutrons is considered both a source and a loss - when neutrons stream into the control volume, streaming represents a source, and vice versa. The net rate at which neutrons leave the control volume is given as the surface integral of the neutron density multiplying the dot product of its velocity \(v\hO\) and the unit outward normal \(\hat{n}\) for the surface \(dS\):

\beq
\label{eq:streaming1}
\left\lbrack\int dS n\seat v(E)\hO\cdot\hat{n}\right\rbrack dEd\hO
\eeq

All source and loss terms derived thus far have been represented as volume integrals; in order to permit the entire \gls{nte} to be represented as a volume integral, convert Eq. \eqref{eq:streaming1} to a volume integral using the divergence rule:

\beqa
\label{eq:streaming2}
\left\lbrack\int dS n\seat v(E)\hO\cdot\hat{n}\right\rbrack dEd\hO\equiv&\left\lbrack\int d\volume\nabla\cdot\left(n\seat v(E)\hO\right)\right\rbrack dEd\hO\\
=&\left\lbrack\int d\volume\hO\cdot \nabla\left(v(E)n\seat\right)\right\rbrack dEd\hO
\eeqa

where \(\nabla\cdot\hO=0\) has been used because \(\hO\) is not a function of the spatial coordinate. Summing all of the source and loss terms in Eqs. \eqref{eq:ConservationExternalSource}, \eqref{eq:ConservationInscatteringSource}, \eqref{eq:ConservationFissionPrompt}, \eqref{eq:ConservationFissionDelayed}, \eqref{eq:AbsorptionLoss}, \eqref{eq:ScatteringLoss}, and \eqref{eq:streaming2} and inserting into Eq. \eqref{eq:neutron_conservation}, canceling the common \(dEd\hO\) differential, and recognizing that because the volume is arbitrary the integrand must be zero, gives:

\beqa
\label{eq:nte}
&\frac{\partial n\seat}{\partial t}+\hO\cdot\nabla(v(E)n\seat)+\Sigma_t\seat v(E)n\seat=\\
&\hspace{1cm}\inscatteringsource v(E')n\seatprime \ +\\
&\hspace{2cm}\promptfissionsource v(E')n\seat \ +\\
&\hspace{3cm}\delayedfissionsource + \source
\eeqa

Eq. \eqref{eq:nte} is the \gls{nte} written in terms of neutron density. As can be seen, the product \(vn\) appears in almost every term, which suggests a convenient definition of the angular flux \(\psi\):

\beq
\label{eq:AngularFlux}
\psi\spa \equiv v(E)n\spa
\eeq

The angular flux is the number of neutrons per unit area and time with energy \(E\) and direction of motion \(\hO\). Eq. \eqref{eq:nte} can be written in terms of the angular flux using Eq. \eqref{eq:AngularFlux} as:

\beqa
\label{eq:nte1}
&\frac{\partial}{\partial t}\left(\frac{\psi\seat}{v(E)}\right)+\hO\cdot\nabla\psi\seat+\Sigma_t\seat \psi\seat=\\
&\hspace{1cm}\inscatteringsource\psi\seatprime\\
&\hspace{2cm}\promptfissionsource\psi\seatprime +\\
&\hspace{3cm}\delayedfissionsource + \source
\eeqa

The \gls{nte} in Eq. \eqref{eq:nte1} is an integro-differential equation because it contains both integrals and derivatives. The initial condition requires specification of the initial angular flux:

\beq
\psi(\vv{r},E,\hO,0)\equiv\psi_0(\vv{r},E,\hO)
\eeq

The \gls{bc} for Eq. \eqref{eq:nte1} is a condition on the incoming flux, and consists of the sum of a generic incoming flux, any specular reflective flux (i.e. the incoming angle is a reflection of the outgoing angle), and any diffusive reflective flux (i.e. the incoming angle is averaged over all outgoing angles):

\beqa
\label{eq:NTEBCs}
\psi\seat=&\ \psi_{inc}\seat+\alpha_s\seat\psi(\vv{r},E,\hO_r,t)+\\
&\hspace{1cm}\alpha_d\seat\frac{\int_{\hO'\cdot\hat{n}>0}d\hO'\|\hO'\cdot\hat{n}\psi(\vv{r},E,\hO',t)}{\int_{\hO'\cdot\hat{n}>0}d\hO'\|\hO'\cdot\hat{n}}
\eeqa

on \(\hO\cdot\hat{n}<0\), where \(\alpha_s\) is the specular reflectivity, \(\alpha_d\) is the diffusive reflectivity, and \(\hO_r\) is the reflected direction corresponding to \(\hO\):

\beq
\label{eq:hOrDef}
\hO_r\equiv\hO-2\left(\hO\cdot\hat{n}\right)\hat{n}
\eeq

At at interface \(\vv{r_s}\) of two differing cross sections \(1\) and \(2\), the angular flux must be continuous to properly conserve neutrons:

\beq
\label{eq:NTE_interface}
\psi_1(\vv{r}_s,E,\hO,t)=\psi_2(\vv{r}_s,E,\hO,t)
\eeq

The scattering and fission terms couple all energies and angles together. Eq. \eqref{eq:nte1} can be solved analytically for general problems for purely absorbing media, but the presence of the integral forms for fissioning and scattering media greatly complicates analytic solution, and in most cases precludes it. 

Several approximations exist in Eq. \eqref{eq:nte1} -

\begin{itemize}
\item Neutron-neutron interactions are negligible because the density of neutrons is much smaller than typical atomic densities. With this approximation, the \gls{nte} is equivalent to the linear Boltzmann equation.
\item Neutron decay is negligible because the mean neutron lifetimes from birth to absorption are significantly smaller than neutron decay half lives on the order of 15 minutes.
\item The neutron density is only a function of \(\vv{r}\), \(E\), \(\hO\), and \(t\) - other parameters such as neutron spin are neglected.
\end{itemize}

The \gls{nte} is sometimes written in a form known as the ``eigenvalue'' form, in which there is no external source and all time dependence is removed. Using such a form to find an eigenvalue solution attempts to determine the largest eigenvalue of the system; this eigenvalue is defined as the largest of the eigenvalues \(1/k\) preceding the prompt and delayed fission source term, and hence \(k\) represents the ratio of fission sources to losses for a steady-state system with no external sources.

The remainder of this section presents alternative forms of the \gls{nte}, and is organized according to each individual term appearing in Eq. \eqref{eq:nte1}.

\subsection{The Fission Source Term}

For steady-state calculations, the time delay of the delayed neutrons becomes unimportant, and it is common to combine the prompt and delayed fission source terms into a single term with combined energy and angle spectrum \(\chi(E,\hO)\):

\beqa
\promptfissionsource\psi\seatprime+\delayedfissionsource\rightarrow\\
\totalfissionsource\psi\seatprime
\eeqa

If both the prompt and delayed neutrons are assumed to be born isentropic, then Eqs. \eqref{eq:prompt_isotropic} and \eqref{eq:delayed_isotropic} show that the total energy and angle fission neutron spectrum can be written solely in terms of energy dependence:

\beq
\int_{4\pi}d\hO\chi(E,\hO)=\frac{\chi(E)}{4\pi}
\eeq

\subsection{The Inscattering Source Term}

It is frequently assumed that the angular dependence of a collision is rotationally symmetric, meaning that scattering cross sections do not depend uniquely on both \(\hO\) and \(\hO'\), but only on the change between the incident and outgoing directions. It is extremely rare for the scattering cross section to depend on the incident neutron direction of motion unless the target nuclei are aligned such as in ferromagnetic materials. This rare case does not occur in most reactor analysis problems, and hence rotational symmetry is frequently implicitly assumed.

Assuming rotational symmetry, \(\hO\rightarrow\hO'\) can be represented as the dot product between \(\hO\) and \(\hO'\), which by the definition of a dot product and the presence of two unit vectors is simply equal to \(\mu\), defined in Eq. \eqref{eq:muDef}:

\beqa
\label{eq:OmegaDotOmega}
\hO  \cdot\hO  ' \equiv& |\hO  ||\hO  '| \cos{(\theta)} \\
=&\ \mu
\eeqa

Rotational symmetry implies that we are ignoring any type of anisotropic grain structure, which is sufficient since on a macroscopic scale, grains are sufficiently randomly oriented. Because the scattering cross sections can now easily be expressed as a function of \(\mu\) instead of \(\hO\) and \(\hO'\), the scattering cross section can be expanded in a series of orthogonal polynomials, where only the coefficients of the expansion need to be stored. Orthogonal polynomials commonly selected for this purpose are the Legendre polynomials and the spherical harmonics functions. The remainder of this section describes two such orthogonal functions that are commonly used to rewrite the scattering term. If the scattering is assumed isotropic, then a significant simplification is possible:

\beq
\frac{\Sigma_s(\vv{r},E'\rightarrow E,t)}{4\pi}=\int_{4\pi}d\hO\Sigma_s\seatout
\eeq

While scattering is typically nearly isotropic in the center of mass frame, this assumption is typically a poor approximation of reality, especially for low mass number nuclei that have significantly peaked forward scattering.

\subsubsection{Legendre Polynomial Expansion}
\label{sec:LegendrePolynomialExpansion}

The scattering cross section can be expanded as a sum of Legendre polynomials multiplying coefficients \(\Sigma_{s,l}\), also known as the scattering moments:

\beq
\label{eq:ScatteringLegendre}
\Sigma_s\seatout \equiv\sum_{l=0}^L\frac{2l+1}{4\pi}\Sigma_{s,l}(\vv{r},E'\rightarrow E,t)P_l(\mu)
\eeq

\(L\) is the order of scattering anisotropy. If \(L\) is infinite, the above expansion is exact; however, in practice a finite sum is used, where \(L\) is typically chosen to match the order of the angular approximation made in the flux. For neutron and photon transport, using only a few terms is typically sufficient, though this is not the case for electron scattering. When \(L=0\), the angular dependence of a cross section is entirely neglected, which therefore represents isotropic scattering. \(L=1\) is often referred to being ``linearly anisotropic,'' since the functional form of the series expansion is \(C_0+C_1\mu\).

The factor of \((2l+1)/4\pi\) appears to obtain the desired normalization of \(2\pi\) in the definition of the scattering moment to account for integration over \(0\leq\phi\leq2\pi\). Multiplying both sides by \(P_{l'}(\mu)\), integrating over \(\mu\), and using the orthogonality property of Legendre polynomials in Eq. \eqref{eqn:LegendrePolynomialsOrthogonality} gives the following definition for the scattering moments:

\beqa
\label{eq:ScatteringMomentsLegendre}
\int_{-1}^1d\mu\Sigma_s\seatout P_{l'}(\mu)=&\int_{-1}^1d\mu\Sigma_{s,l}(\vv{r},E'\rightarrow E,t)P_l(\mu)P_{l'}(\mu)\\
2\pi\int_{-1}^1d\mu\Sigma_s\seatout P_{l}(\mu)=&\Sigma_{s,l}(\vv{r},E'\rightarrow E,t)\\
\eeqa

Expanding the scattering cross section in Legendre polynomials is known as the \(P_L\) approximation. Legendre polynomials are also frequently used to express the angular dependence of the angular flux in one-dimensional problems where \(\hO=\cos{(\theta)}\hat{z}\) has a single independent variable \(\mu\) given by Eq. \eqref{eq:muDef}.

\subsubsection{Spherical Harmonics Expansion}

Similar to the Legendre polynomials, any quantity \(f(\hO)\) can be expanded as a sum of spherical harmonics functions multiplying coefficients \(f_{lm}\), known as moments:

\beq
\label{eq:SphericalHarmonicsGeneralExpansion}
f(\hO)\equiv\sum_{l=0}^{N}\sum_{m=-l}^{l}Y_{lm}(\hO)f_{lm}
\eeq

Multiplying both sides of Eq. \eqref{eq:SphericalHarmonicsGeneralExpansion} by \(Y_{lm}^*\), integrating over solid angle, and using the orthogonality property of spherical harmonics in Eq. \eqref{eq:SHOrthogonality} gives the following definition for the moments:

\beq
\label{eq:SHGeneralMoments}
f_{lm}\equiv\int_{4\pi}^{}f(\hO)Y_{lm}^{*}(\hO)d\hO  
\eeq

After expanding the scattering cross section in the Legendre polynomials as shown in Section \ref{sec:LegendrePolynomialExpansion}, the spherical harmonics addition theorem in Eq. \eqref{eq:SHAdditionTheorem} may be used to substitute spherical harmonics functions for Legendre functions in Eq. \eqref{eq:ScatteringLegendre}:

\beq
\label{eq:ScatteringSH}
\Sigma_s\seatout \equiv\sum_{l=0}^L\Sigma_{s,l}(\vv{r},E'\rightarrow E,t) \sum_{m=-l}^{l} Y_{lm}^{*} (\hO  ') Y_{lm}(\hO  )
\eeq

When Eq. \eqref{eq:ScatteringSH} is used in the \gls{nte}, the inscattering source term can be written as follows:

\beq
\label{eq:SHInscattering}
\int_0^\infty dE\sum_{l=0}^L\Sigma_{s,l}(\vv{r},E'\rightarrow E,t) \sum_{m=-l}^{l}  Y_{lm}(\hO  )\phi_{lm}(\vv{r},E,t)
\eeq

 where the definition of the flux moments \(\phi_{lm}\) from Eq. \eqref{eq:SHGeneralMoments} has been used. Because the diffusion approximation is equivalent to assuming isotropic scattering, \(\phi_{00}\) is equivalent to the scalar flux.
