\section{The Continuous-Energy Neutron Transport Equation}
\label{sec:CE_NTE}

The \gls{nte} is a statement of the conservation of neutron density \(n\seat\), the {\it expected} number of neutrons at a point in space per unit time \(t\), energy \(E\), and (unit) direction of motion \(\hO\). The {\it true} number of neutrons at a point in space cannot be known, since statistical processes govern the exact distribution of neutrons, and we will be limited to predicting mean or average neutron populations. The neutron density is a function of a seven-dimensional space - space (three), energy (one), direction of motion described as angles (two) within and relative to one coordinate plane, and time (one). The energy of the neutron refers to kinetic energy:

\beq
\label{eq:KineticEnergyDef}
E\equiv\frac{1}{2}mv^2
\eeq

where \(v\equiv\|\vv{v}\|_2\) is used throughout. Because neutron kinetic energies are typically very high, units of electron volts for energy are typically used, rather than the more conventional unit of Joules. The unit direction of motion \(\hO\) is a unit vector in the same direction as the neutron velocity \(\vv{v}\):

\beq
\label{eq:OmegaDef}
\hO\equiv \frac{\vv{v}}{v}
\eeq

The direction of motion of a {\it fluid} particle is typically expressed using ``velocity,'' with one vector component corresponding to each spatial dimension. However, this can equivalently be expressed in terms of the kinetic energy \(E\) and two angles \(0\leq\phi\leq2\pi\) and \(0\leq\theta\leq\pi\), where \(\phi\) is the angle in the \(x\)-\(y\) plane from the positive \(x\)-axis and \(\theta\) is the angle between the flight direction and the \(z\)-axis. Like velocity, \(\hO\) has three components corresponding to the three spatial dimensions, where each component is only a function of the two angles \(\phi\) and \(\theta\). Note that using \(\hO\) instead of velocity does not reduce the number of unknowns - instead of working with three velocity components, we work with one energy component and two angular components. This decomposition is useful for tabulating scattering cross sections independently as a function of energy and incoming and outgoing directions, rather than jointly in terms of velocity.

\begin{tcolorbox}[breakable]
In Cartesian coordinates, \(\hO\) is given in terms of unit vectors \(\hat{i}\), \(\hat{j}\), and \(\hat{k}\) corresponding to the \(x\), \(y\), and \(z\) axes, respectively, as:

\beqa
\label{eq:OmegaCartesian}
\hO  = \sin{(\theta)}\cos{(\phi)}\hat{i} + \sin{(\theta)}\sin{(\phi)}\hat{j} + \cos{(\theta)}\hat{k}\\
\eeqa

Each of these components are often given their own symbols: 

\beq
\label{eq:xiDef}
 \xi \equiv \sin{(\theta)}\cos{(\phi)}
\eeq

\beq
\label{eq:etaDef}
 \eta \equiv \sin{(\theta)}\sin{(\phi)}
\eeq

\beq
\label{eq:muDef}
 \mu \equiv \cos{(\theta)}
 \eeq
 
 Using the \(\sin^2{(\theta)} + \cos^2{(\theta)} = 1\) identity, Eq. \eqref{eq:OmegaCartesian} can also be written as:
 
\beq
\hO=\sqrt{1-\mu^2}\textrm{cos}(\phi)\hat{i} + \sqrt{1-\mu^2}\textrm{sin}(\phi)\hat{j} + \mu\hat{k}\\
\eeq
 
Due to the very simple form of the component of \(\hO\) along the \(z\)-axis, one-dimensional Cartesian problems are frequently reoriented such that \(z\) is the independent variable.\newline

In a spherical coordinate system, an integral over the surface of the unit sphere is given as \(dA=r^2\sin{(\theta)}d\theta d\phi\), so an integral over the solid angle is determined by dividing by \(r^2\) since solid angle represents the dimensionless area on the unit sphere:

\beqa
\label{eq:DifferentialOmega}
d\hO  \equiv&\ \sin{(\theta)}d\theta d\phi\\
=&\ d\mu d\phi
\eeqa

From the definition of \(\mu\) in Eq. \eqref{eq:muDef}, \(d\mu=d(\cos{(\theta)})=\sin{(\theta)}d\theta\), giving the second equivalent form in Eq. \eqref{eq:DifferentialOmega}. An integral over solid angle is represented shorthand by \(\int_{4\pi}^{ } d\hO\), and is equivalent to the integral of Eq. \eqref{eq:DifferentialOmega}:

\beqa
\label{eq:SolidAngleIntegration}
\int_{4\pi}d\hO\equiv&\ \int_{-1}^1d\mu\int_0^{2\pi}d\phi\\
=&\ 4\pi
\eeqa

The integral of any single component of \(\hO\) defined in Eq. \eqref{eq:OmegaCartesian} over solid angle is zero:

\beq
\label{eq:OmegaCartesianIntegration}
\int_{4\pi}d\hO\Omega_i=0
\eeq

The integral of any two components of \(\hO\) defined in Eq. \eqref{eq:OmegaCartesian} over solid angle gives the following orthogonality condition:

\beq
\label{eq:4PiOmegaOmega}
\int_{4\pi}^{ } d\hO\ \Om_i\Om_j = \frac{4\pi}{3}\delta_{ij}
\eeq

The integral of any three components of \(\hO\) defined in Eq. \eqref{eq:OmegaCartesian} over solid angle is zero:

\beq
\label{eq:4PiOmegaOmegaOmega}
\int_{4\pi}d\hO \Om_i\Om_j\Om_k=0
\eeq
 
\end{tcolorbox}

 Because the neutron population is conserved, the rate of change of the neutron population within an arbitrary convex control volume \(\volume(t)\) consisting must equal the sum of any gains and losses of neutrons:

\beq
\label{eq:neutron_conservation}
\left\lbrack\frac{\partial}{\partial t}\int d\volume n\seat\right\rbrack dEd\hO=\text{gains}-\text{losses}
\eeq

For those familiar with fluid mechanics, this derivation immediately begins from the control volume perspective, rather than the system perspective, eliminating the need to use the Reynolds Transport Theorem to convert a system perspective to a control volume perspective. 

The volume must be convex to eliminate any boundary sources of neutrons that arise due to leakage from one portion of the volume to another across the boundary. Because the neutron density is in general a function of \(E\) and \(\hO\), a correct statement of conservation requires all volume integrals to be multiplied by the differential \(dEd\hO\) to represent integration over all energies and angles (though integration will not be shown, but implicitly assumed to simplify notation). If the volume is assumed to be independent of time, the time integration can be brought inside the integral such that the time rate of change term becomes:

\beq
\left\lbrack\int d\volume \frac{\partial n\seat}{\partial t}\right\rbrack dEd\hO
\eeq

Sources of neutrons exist due to external sources \(S\) independent of the neutron density; scattering from other energies and angles; and fission. The external source contribution to Eq. \eqref{eq:neutron_conservation} is:

\beq
\label{eq:ConservationExternalSource}
\left\lbrack\int d\volume \source \right\rbrack dEd\hO  
\eeq

The scattering of neutrons from other energies and angles to \(E\) and \(\hO\), also known as the in-scattering source, is an integral of the scattering reaction rate density over all neutron energies \(E'\) and angles \(\hO'\):

\beq
\label{eq:ConservationInscatteringSource}
\left\lbrack\int d\volume\inscatteringsource v(E')n\seatelse\right\rbrack dEd\hO  
\eeq

Note that Eq. \eqref{eq:ConservationInscatteringSource} includes ``sources'' of neutrons due to scattering from \(E\rightarrow E\) and \(\hO\rightarrow\hO\), which should not be represented as either a source or a loss. This double-counting will be cancelled later.

The fission source is expressed as the sum of a prompt and delayed source of neutrons. The prompt fission source is written as an integral over all neutron energies and angles of the fission reaction rate density:

\beq
\label{eq:ConservationFissionPrompt}
\left\lbrack\int d\volume\promptfissionsource v(E')n\seatelse\right\rbrack dEd\hO
\eeq

If prompt neutrons are born nearly isotropically (which is frequently a good approximation), angle integration of the prompt fission neutron energy and angle spectrum yields only an energy dependence:

\beq
\label{eq:prompt_isotropic}
\int_{4\pi}d\hO \chi_p(E,\hO)=\frac{\chi_p(E)}{4\pi}
\eeq

\(\chi_p(E,\hO)\) is frequently replaced by \(\chi_p(E)/4\pi\) in Eq. \eqref{eq:ConservationFissionPrompt} using this assumption. The delayed neutron source cannot be written simply as an integral of a reaction rate density because interaction probabilities for the production and decay of fission product nuclides that result in neutron production are not simply functions of physical parameters. The movement of fission products impacts the production of delayed neutrons within the control volume. The delayed fission source is frequently approximated as an integral over a finite sum of \(J\) grouped fission product precursor concentrations:

\beq
\label{eq:ConservationFissionDelayed}
\left\lbrack\int d\volume \delayedfissionsource\right\rbrack dEd\hO  
\eeq

where \(C\) is the delayed neutron precursor concentration, and \(\lambda\) is the delayed neutron precursor decay constant. Similar to prompt fission neutrons, delayed neutrons are frequently approximated as being born isotropically, and angle integration of the delayed neutron energy and angled spectrum yields only an energy dependence:

\beq
\label{eq:delayed_isotropic}
\int_{4\pi}d\hO \chi_d(E,\hO)=\frac{\chi_d(E)}{4\pi}
\eeq

\(\chi_d(E,\hO)\) is frequently replaced by \(\chi_d(E)/4\pi\) in Eq. \eqref{eq:ConservationFissionDelayed} using this assumption. Losses of neutrons exist due to absorption and scattering to other energy groups. The absorption loss is represented by an integral in space of the absorption reaction rate density:

\beq
\label{eq:AbsorptionLoss}
\left\lbrack\int d\volume \Sigma_a\seat v(E)n\seat\right\rbrack dEd\hO
\eeq

The scattering to other energy groups, referred to as ``out-scattering,'' is represented as an integral of the scattering reaction rate density:

\beq
\label{eq:ScatteringLoss}
\left\lbrack\int d\volume \Sigma_s(\vv{r},E\rightarrow E',\hO\rightarrow\hO',t) v(E)n\seat\right\rbrack dEd\hO
\eeq

Note that Eq. \eqref{eq:ScatteringLoss} includes scattering from \(E\rightarrow E\), which should not be represented as a loss term. This double-counting is balanced by the same double-counting in the in-scattering term in Eq. \eqref{eq:ConservationInscatteringSource}, giving the correct conservation statement. Eqs. \eqref{eq:AbsorptionLoss} and \eqref{eq:ScatteringLoss} are frequently combined by defining the total cross section \(\Sigma_t\) as:

\beq
\label{eq:TotalSigmaDef}
\Sigma_t\seatout\equiv\Sigma_a\seat+\Sigma_s\seatout
\eeq

Finally, streaming of neutrons is considered both a source and a loss - when neutrons stream into the control volume, streaming represents a source, and vice versa. The net rate at which neutrons leave the control volume is given as the surface integral of the neutron density multiplying the dot product of its velocity \(v\hO\) and the unit outward normal \(\hat{n}\) for the surface \(dS\):

\beq
\label{eq:streaming1}
\left\lbrack\int dS n\seat v(E)\hO\cdot\hat{n}\right\rbrack dEd\hO
\eeq

All source and loss terms derived thus far have been represented as volume integrals; in order to permit the entire \gls{nte} to be represented as a volume integral, convert Eq. \eqref{eq:streaming1} to a volume integral using the divergence rule:

\beqa
\label{eq:streaming2}
\left\lbrack\int dS n\seat v(E)\hO\cdot\hat{n}\right\rbrack dEd\hO\equiv&\left\lbrack\int d\volume\nabla\cdot\left(n\seat v(E)\hO\right)\right\rbrack dEd\hO\\
=&\left\lbrack\int d\volume\hO\cdot \nabla\left(v(E)n\seat\right)\right\rbrack dEd\hO
\eeqa

where \(\nabla\cdot\hO=0\) has been used because \(\hO\) is not a function of the spatial coordinate. Summing all of the source and loss terms in Eqs. \eqref{eq:ConservationExternalSource}, \eqref{eq:ConservationInscatteringSource}, \eqref{eq:ConservationFissionPrompt}, \eqref{eq:ConservationFissionDelayed}, \eqref{eq:AbsorptionLoss}, \eqref{eq:ScatteringLoss}, and \eqref{eq:streaming2} and inserting into Eq. \eqref{eq:neutron_conservation}, canceling the common \(dEd\hO\) differential, and recognizing that because the volume is arbitrary the integrand must be zero, gives the \gls{nte} in terms of neutron density:

\beqa
\label{eq:nte}
&\frac{\partial n\seat}{\partial t}+\hO\cdot\nabla(v(E)n\seat)+\Sigma_t\seat v(E)n\seat=\\
&\hspace{1cm}\inscatteringsource v(E')n\seatprime \ +\\
&\hspace{2cm}\promptfissionsource v(E')n\seat \ +\\
&\hspace{3cm}\delayedfissionsource + \source
\eeqa

The product \(vn\) appears in almost every term in Eq. \eqref{eq:nte}, which suggests a convenient definition of the angular flux \(\psi\):

\beq
\label{eq:AngularFlux}
\psi\spa \equiv v(E)n\spa
\eeq

The angular flux is the number of neutrons per unit area and time with energy \(E\) and direction of motion \(\hO\). Eq. \eqref{eq:nte} can be written in terms of the angular flux using Eq. \eqref{eq:AngularFlux} as:

\beqa
\label{eq:nte1}
&\frac{\partial}{\partial t}\left(\frac{\psi\seat}{v(E)}\right)+\hO\cdot\nabla\psi\seat+\Sigma_t\seat \psi\seat=\\
&\hspace{1cm}\inscatteringsource\psi\seatprime\\
&\hspace{2cm}\promptfissionsource\psi\seatprime +\\
&\hspace{3cm}\delayedfissionsource + \source
\eeqa

The \gls{nte} is coupled to \(J\) delayed neutron precursor concentration equations; sources of delayed neutrons occur due to fission at all energies, since \(\beta_j\) represents the average fraction of neutrons born into delayed group \(j\):

\beq
\label{eq:DelayedNeutrons}
\frac{\partial C_j(\vv{r},t)}{\partial t}=\beta_j\int_{0}^\infty dE'\int_{4\pi}\nu(E')\Sigma_f(E')\psi(\vv{r},E',\hO',t)-\lambda_jC_j(\vv{r},t)
\eeq

The \gls{nte} in Eq. \eqref{eq:nte1} is an integro-differential equation because it contains both integrals and derivatives. The initial condition requires specification of the initial angular flux:

\beq
\psi(\vv{r},E,\hO,0)\equiv\psi_0(\vv{r},E,\hO)
\eeq

The \gls{bc} for Eq. \eqref{eq:nte1} is a condition on the incoming flux, and consists of the sum of a generic incoming flux, any specular reflective flux (i.e. the incoming angle is a reflection of the outgoing angle), and any diffusive reflective flux (i.e. the incoming angle is averaged over all outgoing angles):

\beqa
\label{eq:NTEBCs}
\psi\seat=&\ \psi_{inc}\seat+\alpha_s\seat\psi(\vv{r},E,\hO_r,t)+\\
&\hspace{1cm}\alpha_d\seat\frac{\int_{\hO'\cdot\hat{n}>0}d\hO'\|\hO'\cdot\hat{n}\psi(\vv{r},E,\hO',t)}{\int_{\hO'\cdot\hat{n}>0}d\hO'\|\hO'\cdot\hat{n}}
\eeqa

on \(\hO\cdot\hat{n}<0\), where \(\alpha_s\) is the specular reflectivity, \(\alpha_d\) is the diffusive reflectivity, and \(\hO_r\) is the reflected direction corresponding to \(\hO\):

\beq
\label{eq:hOrDef}
\hO_r\equiv\hO-2\left(\hO\cdot\hat{n}\right)\hat{n}
\eeq

At at interface \(\vv{r_s}\) of two differing cross sections \(1\) and \(2\), the angular flux must be continuous to properly conserve neutrons:

\beq
\label{eq:NTE_interface}
\psi_1(\vv{r}_s,E,\hO,t)=\psi_2(\vv{r}_s,E,\hO,t)
\eeq

The scattering and fission terms couple all energies and angles together. Eq. \eqref{eq:nte1} can be solved analytically for general problems for purely absorbing media, but the presence of the integral forms for fissioning and scattering media greatly complicates analytic solution, and in most cases precludes it. Several approximations exist in Eq. \eqref{eq:nte1} -

\begin{itemize}
\item Neutron-neutron interactions are negligible because the density of neutrons is much smaller than typical atomic densities. With this approximation, the \gls{nte} is equivalent to the linear Boltzmann equation.
\item Neutron decay is negligible because the mean neutron lifetimes from birth to absorption are significantly smaller than neutron decay half lives on the order of 15 minutes.
\item The neutron density is only a function of \(\vv{r}\), \(E\), \(\hO\), and \(t\) - other parameters such as neutron spin are neglected.
\end{itemize}

The remainder of this section presents alternative forms of the \gls{nte}, and is organized according to each individual term appearing in Eq. \eqref{eq:nte1} or to general rewritings such as eigenvalue form.

\subsection{The Fission Source Term}

For steady-state calculations, the time delay of the delayed neutrons becomes unimportant, and it is common to combine the prompt and delayed fission source terms into a single term with combined energy and angle spectrum \(\chi(E,\hO)\):

\beqa
\promptfissionsource\psi\seatprime+\delayedfissionsource\rightarrow\\
\totalfissionsource\psi\seatprime
\eeqa

If both the prompt and delayed neutrons are assumed to be born isentropic, then Eqs. \eqref{eq:prompt_isotropic} and \eqref{eq:delayed_isotropic} show that the total energy and angle fission neutron spectrum can be written solely in terms of energy dependence:

\beq
\int_{4\pi}d\hO\chi(E,\hO)=\frac{\chi(E)}{4\pi}
\eeq

\subsection{The Inscattering Source Term}

It is frequently assumed that the angular dependence of a collision is rotationally symmetric, meaning that scattering cross sections do not depend uniquely on both \(\hO\) and \(\hO'\), but only on the change between the incident and outgoing directions. It is extremely rare for the scattering cross section to depend on the incident neutron direction of motion unless the target nuclei are aligned such as in ferromagnetic materials. This rare case does not occur in most reactor analysis problems, and hence rotational symmetry is frequently implicitly assumed.

Assuming rotational symmetry, \(\hO\rightarrow\hO'\) can be represented as the dot product between \(\hO\) and \(\hO'\), which by the definition of a dot product and the presence of two unit vectors is simply equal to \(\mu\), defined in Eq. \eqref{eq:muDef}:

\beqa
\label{eq:OmegaDotOmega}
\hO  \cdot\hO  ' \equiv& |\hO  ||\hO  '| \cos{(\theta)} \\
=&\ \mu
\eeqa

Rotational symmetry implies that we are ignoring any type of anisotropic grain structure, which is sufficient since on a macroscopic scale, grains are sufficiently randomly oriented. Because the scattering cross sections can now easily be expressed as a function of \(\mu\) instead of \(\hO\) and \(\hO'\), the scattering cross section can be expanded in a series of orthogonal polynomials, where only the coefficients of the expansion need to be stored. Orthogonal polynomials commonly selected for this purpose are the Legendre polynomials and the spherical harmonics functions. The remainder of this section describes two such orthogonal functions that are commonly used to rewrite the scattering term. If the scattering is assumed isotropic, then a significant simplification is possible:

\beq
\frac{\Sigma_s(\vv{r},E'\rightarrow E,t)}{4\pi}=\int_{4\pi}d\hO\Sigma_s\seatout
\eeq

While scattering is typically nearly isotropic in the center of mass frame, this assumption is typically a poor approximation of reality, especially for low mass number nuclei that have significantly peaked forward scattering.

\subsubsection{Legendre Polynomial Expansion}
\label{sec:LegendrePolynomialExpansion}

The scattering cross section can be expanded as a sum of Legendre polynomials multiplying coefficients \(\Sigma_{s,l}\), also known as the scattering moments:

\beq
\label{eq:ScatteringLegendre}
\Sigma_s\seatout \equiv\sum_{l=0}^L\frac{2l+1}{4\pi}\Sigma_{s,l}(\vv{r},E'\rightarrow E,t)P_l(\mu)
\eeq

\(L\) is the order of scattering anisotropy. If \(L\) is infinite, the above expansion is exact; however, in practice a finite sum is used, where \(L\) is typically chosen to match the order of the angular approximation made in the flux. For neutron and photon transport, using only a few terms is typically sufficient, though this is not the case for electron scattering. When \(L=0\), the angular dependence of a cross section is entirely neglected, which therefore represents isotropic scattering. \(L=1\) is often referred to being ``linearly anisotropic,'' since the functional form of the series expansion is \(C_0+C_1\mu\).

The factor of \((2l+1)/4\pi\) appears to obtain the desired normalization of \(2\pi\) in the definition of the scattering moment to account for integration over \(0\leq\phi\leq2\pi\). Multiplying both sides by \(P_{l'}(\mu)\), integrating over \(\mu\), and using the orthogonality property of Legendre polynomials in Eq. \eqref{eqn:LegendrePolynomialsOrthogonality} gives the following definition for the scattering moments:

\beqa
\label{eq:ScatteringMomentsLegendre}
\int_{-1}^1d\mu\Sigma_s\seatout P_{l'}(\mu)=&\int_{-1}^1d\mu\Sigma_{s,l}(\vv{r},E'\rightarrow E,t)P_l(\mu)P_{l'}(\mu)\\
2\pi\int_{-1}^1d\mu\Sigma_s\seatout P_{l}(\mu)=&\Sigma_{s,l}(\vv{r},E'\rightarrow E,t)\\
\eeqa

Expanding the scattering cross section in Legendre polynomials is known as the \(P_L\) approximation. Legendre polynomials are also frequently used to express the angular dependence of the angular flux in one-dimensional problems where \(\hO=\cos{(\theta)}\hat{z}\) has a single independent variable \(\mu\) given by Eq. \eqref{eq:muDef}.

\subsubsection{Spherical Harmonics Expansion}

Similar to the Legendre polynomials, any quantity \(f(\hO)\) can be expanded as a sum of spherical harmonics functions multiplying coefficients \(f_{lm}\), known as moments:

\beq
\label{eq:SphericalHarmonicsGeneralExpansion}
f(\hO)\equiv\sum_{l=0}^{N}\sum_{m=-l}^{l}Y_{lm}(\hO)f_{lm}
\eeq

Multiplying both sides of Eq. \eqref{eq:SphericalHarmonicsGeneralExpansion} by \(Y_{lm}^*\), integrating over solid angle, and using the orthogonality property of spherical harmonics in Eq. \eqref{eq:SHOrthogonality} gives the following definition for the moments:

\beq
\label{eq:SHGeneralMoments}
f_{lm}\equiv\int_{4\pi}^{}f(\hO)Y_{lm}^{*}(\hO)d\hO  
\eeq

After expanding the scattering cross section in the Legendre polynomials as shown in Section \ref{sec:LegendrePolynomialExpansion}, the spherical harmonics addition theorem in Eq. \eqref{eq:SHAdditionTheorem} may be used to substitute spherical harmonics functions for Legendre functions in Eq. \eqref{eq:ScatteringLegendre}:

\beq
\label{eq:ScatteringSH}
\Sigma_s\seatout \equiv\sum_{l=0}^L\Sigma_{s,l}(\vv{r},E'\rightarrow E,t) \sum_{m=-l}^{l} Y_{lm}^{*} (\hO  ') Y_{lm}(\hO  )
\eeq

When Eq. \eqref{eq:ScatteringSH} is used in the \gls{nte}, the inscattering source term can be written as follows:

\beq
\label{eq:SHInscattering}
\int_0^\infty dE\sum_{l=0}^L\Sigma_{s,l}(\vv{r},E'\rightarrow E,t) \sum_{m=-l}^{l}  Y_{lm}(\hO  )\phi_{lm}(\vv{r},E,t)
\eeq

 where the definition of the flux moments \(\phi_{lm}\) from Eq. \eqref{eq:SHGeneralMoments} has been used. Because the diffusion approximation is equivalent to assuming isotropic scattering, \(\phi_{00}\) is equivalent to the scalar flux.
 
\subsection{Eigenvalue Form}
\label{sec:EigenvalueForm}
By setting the external source to zero and neglecting time dependence, the \gls{nte} can be written in eigenvalue form as:

\beqa
\label{eq:EigenvalueNTE}
\hO\cdot\nabla\psi(\vv{r},E,\hO)+\Sigma_t(\vv{r},E,\hO)\psi(\vv{r},E,\hO)=\hspace{2cm}\\
\int_0^\infty dE'\int_{4\pi}d\hO'\Sigma_s(\vv{r},E'\rightarrow E,\hO'\rightarrow\hO)\psi(\vv{r},E',\hO')\ +\hspace{1cm}\\
\frac{1}{k}\chi(E,\hO)\int_0^\infty dE'\int_{4\pi}d\hO'\nu(E')\Sigma_f(\vv{r},E'\hO')\psi(\vv{r},E',\hO')
\eeqa

where for simplicity the fission source has been written as the combined prompt and delayed source due to the assumption of time-independence. Note that Eq. \eqref{eq:EigenvalueNTE} is an eigenvalue equation because \(\psi\) appears in every term. The factor \(1/k\) is inserted as the eigenvalue. (Continuous) eigenvalue problems have an infinite number of solutions (eigenfunctions) corresponding to an infinite number of eigenvalues. The objective of solution of a discretized version of Eq. \eqref{eq:EigenvalueNTE} is to find the smallest eigenvalue \(1/k\) because this eigenvalue corresponds to the smallest geometric buckling, and hence the smallest leakage and longest steady time behavior. The goal of reactor design is to obtain systems that self-sustain the critical reaction in the absence of an external source.

\begin{tcolorbox}[breakable]
To illustrate the significance of finding the {\it smallest} eigenvalue \(1/k\) to search for the steady-state solution, consider the one-group diffusion equation in a homogeneous medium as an example:

\beq
\label{eq:1GDiff}
\frac{1}{v}\frac{\partial\phi(\vv{r},t)}{\partial t}-D\nabla^2\phi(\vv{r},t)+\left(\Sigma_a-\nu\Sigma_f\right)\phi(\vv{r},t)=S(\vv{r},t)
\eeq

For a slab reactor in Cartesian coordinates with vacuum conditions at \(x=\pm\tilde{a}/2\) and a symmetric initial condition \(\phi(\vv{r},0)\) in the absence of an external source, a separation of variables technique can be used to seek a solution of the form \(\phi(x,t)=X(x)T(t)\). Separation of variables can in general only be applied to homogeneous problems. Inserting this assumed form into Eq. \eqref{eq:1GDiff} gives, after dividing through by \(X(x)T(t)\):

\beq
\underbrace{\frac{1}{T(t)}\frac{\partial T(t)}{\partial t}}_{-\lambda}+\underbrace{v\left\lbrack-\frac{D}{X(x)}\frac{\partial^2X(x)}{\partial x^2}-\left(\nu\Sigma_f-\Sigma_a\right)\right\rbrack}_{\lambda}=0
\eeq

Because the first term is only a function of time and the second only a function of space, yet the two are equal to one another, both must be equal to a constant \(\pm\lambda\). Therefore, separation of variables has produced two separable differential equations:

\beq
\frac{\partial T(t)}{\partial t}=-\lambda T(t)
\eeq

\beq
\label{eq:Helmholtz}
\frac{\partial^2X(x)}{\partial x^2}+\underbrace{\left(\frac{\nu\Sigma_f-\Sigma_a+\lambda/v}{D}\right)}_{B^2}X(x)=0
\eeq

where \(B^2\), for the time being an arbitrary constant, has been inserted for simplicity of notation. Solution of these two equations subject to the boundary and initial conditions gives:

\beq
T(t)=T(0)e^{-\lambda t}
\eeq

\beq
X(x)=C_n\cos{\left(\frac{n\pi}{\tilde{a}}x\right)}
\eeq

where an infinite number of non-trivial solutions are possible to the spatial Helmholtz-type problem for the discrete points:

\beqa
\label{eq:Eigens}
B_n^2=&\ \frac{n\pi}{\tilde{a}}\\
=&\ \frac{\nu\Sigma_f-\Sigma_a+\lambda_n/v}{D}
\eeqa

for odd \(n\), where \(\lambda\) is now written as \(\lambda_n\) to indicate the infinite number of possible values. Due to the linearity of the \gls{nte} which permits superposition of solutions, the solution can be written as:

\beq
\label{eq:FluxEigenfunctionExp}
\phi(x,t)=\sum_{n=1,\text{odd}}^\infty C_n(x,t)e^{-\lambda_nt}\cos{\left(\frac{n\pi}{\tilde{a}}x\right)}
\eeq

The coefficients \(C_n\) can be determined by application of the orthogonality condition using the initial condition. For \(\lambda_n>0\), the flux decays in time, while for \(\lambda_n<0\), the flux grows in time. Rearranging Eq. \eqref{eq:Eigens} for \(\lambda_n\) gives:

\beq
\label{eq:lambda_n}
\lambda_n=v\left\lbrack D\left(\frac{n\pi}{\tilde{a}}\right)^2+\Sigma_a-\nu\Sigma_f\right\rbrack
\eeq

Note that for both \(\lambda_1>0\) and \(\lambda_1<0\), \(\lambda_{n+1}>\lambda_n\). Therefore, the higher the \(n\) value, the faster that modes decays away in time (for positive \(\lambda\)) or the slower that mode grows in time (for negative \(\lambda\)), so the higher the \(n\), the less presence of that mode for long times. For a sufficiently long time, only the \(n=1\) mode persists; this mode is therefore often referred to as the ``fundamental mode.'' Such a steady-state solution in the absence of an external source represents a critical system. While the separation of variables technique used here does not necessarily imply separability in space and time, for very long times the flux does become separable. An alternative solution technique would have been to expand the flux in the eigenfunctions of the homogeneous system and insert directly into the governing equation to apply orthogonality to obtain a relationship among the coefficients.\newline

Hence, because \(\lambda_{n+1}>\lambda_n\), the objective for performing a criticality calculation is to determine the fundamental mode that remains after a sufficiently long time that the neutron population has reached a steady state. This fundamental mode corresponds to the smallest eigenvalue of the homogeneous problem, \(B_1^2\). The corresponding criticality problem for the general transport equation is to find the smallest eigenvalue \(1/k\). \(B_1^2\) is the smallest eigenvalue of the homogeneous (referring to the form of the differential equation) problem, and is often referred to as the ``geometric buckling.'' This term is proportional to the curvature of the spatial eigenfunction:

\beq
B_n^2=-\frac{1}{X_n(x)}\frac{\partial^2X_n(x)}{\partial x^2}
\eeq

Criticality is therefore obtained for \(lambda_1=0\) in Eq. \eqref{eq:lambda_n}, since all of the higher modes decay the fastest in time. By setting \(\lambda_1=0\), a steady-state solution is obtained:

\beq
B_1^2=\underbrace{\frac{\nu\Sigma_f-\Sigma_a}{D}}_{B_m^2}
\eeq

where \(B_m^2\) is referred to as the ``material buckling.'' If \(\lambda_1>0\), the geometric buckling exceeds the material buckling, giving a subcritical system, while if \(\lambda_1<0\), the material buckling exceeds the geometric buckling, giving a supercritical system. Clearly, criticality is determined by both the system material properties and the geometry.

\end{tcolorbox}

The eigenvalue form is often referred to as the steady-state form of the \gls{nte} because a steady state only exists if the system is critical or if there is an external source in a subcritical system (in which case the \(1/k\) eigenvalue is removed from the steady-state form) \cite{tyobeka}.

While eigenvalue calculations imply an infinite number of solutions, to be considered a valid solution for neutron transport, the eigenfunctions must also be positive. For the one-group diffusion equation in a homogeneous slab, the only positive eigenfunction is the eigenfunction corresponding to \(n=1\), the fundamental mode. Hence, while an infinite number of solutions may in theory be possible, typically only one possible solution exists. The specified problem is likely not a perfectly critical system, hence the ``insertion'' (really, ``un-muting'') of the \(1/k\) factor to ensure a solution can be found. The objective is to solve the eigenvalue form of the \gls{nte} simultaneously for \(\psi\) and \(k\). Due to the placement of \(1/k\) within Eq. \eqref{eq:EigenvalueNTE}, it is clear that \(k\) represents the ratio of fission sources to losses. Section \ref{sec:EigenvalueCalculations} describes how Eq. \eqref{eq:EigenvalueNTE} is numerically solved to obtain the smallest \(1/k\) and the corresponding fundamental mode.

 
 Simple concepts are often introduced to predict how changes in system state affect the criticality by defining the following probabilities:
 
 \beq
 \label{eq:P_NL}
 P_{NL}\equiv\text{probability neutron will {\it not} leak from system before absorption}
 \eeq
 
 \beqa
 \label{eq:fDef}
 f\equiv&\ \frac{\text{probability of absorption in fuel}}{\text{probability of absorption}}\\
 =&\ \frac{\sum_{i=1}^F\Sigma_{a,i}}{\sum_{i=1}^N\Sigma_{a,i}}
 \eeqa
 
 where \(F\) is the total number of fuel isotopes and \(N\) is the total number of isotopes. \(f\) is also known as the ``thermal utilization factor,'' though evaluating the cross sections at fast energies could lend the name ``fast utilization factor.'' For a single energy group, the multiplication factor is simply the multiplication of the average number of neutrons produced per neutron absorbed in fuel (by definition, \(\eta\) from Eq. \eqref{eq:Eta2Def}) with that absorption will occur in fuel (by definition, \(f\) from Eq. \eqref{eq:fDef}), with the probability that absorption occurs within the system rather than leakage (by definition, \(P_{NL}\) from Eq. \eqref{eq:P_NL}):
 
 \beq
 \label{eq:k1}
 k=\eta fP_{NL}
 \eeq
 
When considering multiple neutron energies, Eq. \eqref{eq:k1} can be made more detailed to reflect neutron movement between energy groups by scattering and fission. If the majority of fissions are caused by fast neutrons, \(\eta\) and \(f\) should be evaluated at thermal energies. 

\subsection{The Multigroup Neutron Transport Equation}
\label{sec:MGNTE}

If using a deterministic tool to solve the \gls{nte}, using \gls{ce} cross sections over typical energy ranges of \(10^{-3}-10^7\) eV is impractical due to the very high degree of point sampling required. Therefore, some type of discretization is required for the energy dependence. While the dependence of the angular flux on angle is typically rather weak such that only a few terms in a function expansion could be used for representation, the dependence of angular flux on energy is a very complicated function of the energy range; at high energies, the energy dependence is dominated by the fission spectrum, while at intermediate energies by resonance absorption and at low energies by thermalization effects. A single set of functions to describe all of this behavior would require very many terms, so a discrete ordinates in energy approach is almost universally used. Deterministic transport solvers use a \gls{mg}-in-energy form of the \gls{nte} that is obtained by averaging the \gls{nte} over all energies. For \(G\) energy groups, this produces \(G\) coupled equations, each a function of space, time, and angle, for \(G\) fluxes. Energy group \(g\) is defined over \(E_{g-1}\) to \(E_{g}\), with \(1\leq g\leq G\). Groups are ordered in order of decreasing energy, equivalent to increasing lethargy. This is convenient because energy sweeping methods also typically begin at the highest energies and sweep towards lower energies. The greatest amount of simplification occurs when using a single energy group, which is equivalent to assuming that energy does not change in scattering collisions.

Several definitions are required for the derivation of the \gls{mg} \gls{nte}. Rather than use a collocation-in-energy method where the angular flux is simply sampled at each mesh point \(g\), the angular flux of group \(g\) is defined as the angular flux integrated over the energy bounds for group \(g\):

\beq
\label{eq:GroupwiseQuantity}
\psi_g(\vv{r},\hO,t)\equiv\int_{E_{g-1}}^{E_g}dE\psi\seat
\eeq

In this way, all energies are considered in the discretization, and not simply the pointwise energies representing group boundaries, which permits the use of a relatively small number of energy groups to describe nine orders of magnitude in energy with very rapidly varying cross sections. Groupwise cross sections for reaction \(i\) are defined in a similar manner based on the flux-weighted cross sections:

\beqa
\label{eq:GroupwiseQuantityProduct}
\Sigma_{i,g}(\vv{r},\hO,t)\equiv&\frac{\int_{E_{g-1}}^{E_g}dE\Sigma_{i}\seat\psi\seat}{\int_{E_{g-1}}^{E_g}dE\psi\seat}\\
=&\frac{\int_{E_{g-1}}^{E_g}dE\Sigma_{i}\seat\psi\seat}{\psi_g\sat}\\
\eeqa

where Eq. \eqref{eq:GroupwiseQuantity} has been inserted for the denominator. Note that these group cross sections are dependent on the angular flux, and thereby define an implicit scheme; the alternative, to use a collocation-in-energy method with pointwise cross sections, would require extremely fine energy discretization to obtain accurate results, and hence this implicit dependence is still preferable. 

Using Eqs. \eqref{eq:GroupwiseQuantity} and \eqref{eq:GroupwiseQuantityProduct}, the \gls{mg} \gls{nte} can be derived for group \(g\) by simply integrating the \gls{nte} in Eq. \eqref{eq:nte1} over \(E_{g-1}\leq E\leq E_g\):

\beqa
\label{eq:mg_nte}
&\frac{\partial}{\partial t}\left(\frac{\psi_g(\vv{r},\hO,t)}{v_g}\right)+\hO\cdot\nabla\psi_g\sat+\Sigma_{t,g} \psi_g\sat=\\
&\hspace{1cm}\sum_{g'=1}^G\int_{4\pi}d\hO^'\Sigma_s^{g'\rightarrow g}(\vv{r},\hO'\rightarrow\hO,t)\psi_{g'}(\vv{r},\hO',t)\ +\\
&\hspace{2cm}\chi_{p,g}(\hO)\int_{4\pi}d\hO^'\left\lbrack1-\beta_{g'}\right\rbrack\nu_{g'}\Sigma_{f,g'}(\vv{r},\hO^',t)\psi_{g'}(\vv{r},\hO^',t) +\\
&\hspace{3cm}\sum_{j=1}^J\chi_{d,j,g}\lambda_jC_j(\vv{r},t)+ S_g\sat
\eeqa

where the velocity was moved outside the time differentiation in the time term due to independence on time and \(\hO\cdot\nabla\) was brought outside the energy integral since this operator is not a function of energy. 

The following sections describe the generation of group constants for energies in the fast and thermal ranges. These methods are based on approximations that neglect spatial dependence and use theory to predict the detailed energy spectrum over which the microscopic cross section data can be averaged.

\subsubsection{Selection of Group Boundaries}

The selection of energy boundaries for the energy group structure is not a simple task. Numerical solution of the coupled group equations is greatly simplified if the groups are selected to be sufficiently far separated that scattering out of energy group \(g\) to another energy group is only possible for \(g\rightarrow g-1\) and \(g\rightarrow g+1\). This group structure design is known as a ``directly coupled'' structure, in which case the matrix problem becomes bidiagonal. With heavy moderators such that a very large amount of energy cannot be lost in a single collision, this definition is simpler. However, for light moderators such as hydrogen, neutrons can lose up to all of their energy in an elastic scattering collision, and a directly-coupled group structure can no longer be defined. However, such a structure can usually be approximated by choosing \(E_{g-1}/E_g>150\), which results in less than 1\% of neutrons skipping energy groups in scattering reactions. \(E_{g-1}/E_g\) is also typically selected as constant for other purposes.

In addition, if the thermal groups are selected as sufficiently wide, upscattering can be neglected. This results in a lower triangular matrix structure that permits successive solution of the faster energy groups followed by the more thermal energy groups, as the source in group \(g\) is a linear combination of the fission rate caused by groups \(g-1\leq g\leq 1\). Or, for a very thermal problem, the thermal flux can be described with a Maxwellian distribution. Upscattering can be neglected in \glspl{lwr} and \glspl{htgr} by selecting about 1 and 3 eV, respectively, as the upper boundary of the thermal group \cite{duderstadt}.

\subsubsection{Thermal Spectrum Group Constants}
This section describes the theory behind the generation of group parameters for energies in the thermal range. The generation of thermal group constants is complicated by the complex neutron scattering process at low energies, where the thermal motion of the nuclei cannot be neglected, and neutron energies are typically on the order of binding energies of the nuclei such that interaction tends to occur between a neutron and a group of nuclei simultaneously. Scattering cross sections are strongly dependent on both energy and angle. Further, unlike fast energy cross sections that tend to only depend on the identity of the interacting media, thermal cross sections depend strongly on the state of matter and temperature of the interacting medium.
