\documentclass[10pt]{article}
\usepackage[letterpaper]{geometry}
\geometry{verbose,tmargin=1in,bmargin=1in,lmargin=1in,rmargin=1in}
\usepackage{setspace}
\usepackage{ragged2e}
\usepackage{color}
\usepackage{titlesec}
\usepackage{graphicx}
\usepackage{float}
\usepackage{mathtools}
\usepackage{amsmath}
\usepackage[font=small,labelfont=bf,labelsep=period]{caption}
\usepackage[english]{babel}
\usepackage{indentfirst}
\usepackage{array}
\usepackage{makecell}
\usepackage[usenames,dvipsnames]{xcolor}
\usepackage{multirow}
\usepackage{tabularx}
\usepackage{arydshln}
\usepackage{caption}
\usepackage{subcaption}
\usepackage{xfrac}
\usepackage{etoolbox}
\usepackage{cite}
\usepackage{url}
\usepackage{dcolumn}
\usepackage{hyperref}
\usepackage{courier}
\usepackage{url}
\usepackage{esvect}
\usepackage{commath}
\usepackage{verbatim} % for block comments
\usepackage{enumitem}
\usepackage{hyperref} % for clickable table of contents
\usepackage{braket}
\usepackage{titlesec}
\usepackage{gensymb}
\usepackage{verbatim}

\setcounter{secnumdepth}{4}



\titleclass{\subsubsubsection}{straight}[\subsection]

\newcounter{subsubsubsection}[subsubsection]
\renewcommand\thesubsubsubsection{\thesubsubsection.\arabic{subsubsubsection}}
\renewcommand\theparagraph{\thesubsubsubsection.\arabic{paragraph}} % optional; useful if paragraphs are to be numbered

\titleformat{\subsubsubsection}
  {\normalfont\normalsize\bfseries}{\thesubsubsubsection}{1em}{}
\titlespacing*{\subsubsubsection}
{0pt}{3.25ex plus 1ex minus .2ex}{1.5ex plus .2ex}

\makeatletter
\renewcommand\paragraph{\@startsection{paragraph}{5}{\z@}%
  {3.25ex \@plus1ex \@minus.2ex}%
  {-1em}%
  {\normalfont\normalsize\bfseries}}
\renewcommand\subparagraph{\@startsection{subparagraph}{6}{\parindent}%
  {3.25ex \@plus1ex \@minus .2ex}%
  {-1em}%
  {\normalfont\normalsize\bfseries}}
\def\toclevel@subsubsubsection{4}
\def\toclevel@paragraph{5}
\def\toclevel@paragraph{6}
\def\l@subsubsubsection{\@dottedtocline{4}{7em}{4em}}
\def\l@paragraph{\@dottedtocline{5}{10em}{5em}}
\def\l@subparagraph{\@dottedtocline{6}{14em}{6em}}
\makeatother

\setcounter{secnumdepth}{4}
\setcounter{tocdepth}{4}

% Generate the glossary of acronyms
\usepackage[acronym]{glossaries}
\makeglossaries

\newacronym{aoo}{AOO}{Anticipated Operational Occurrence}
\newacronym{aplhgr}{APLHGR}{Average Planar Linear Heat Generation Rate}
\newacronym{aprm}{APRM}{Average Power Range Monitor}
\newacronym{ari}{ARI}{Alternate Rod Insertion}
\newacronym{bop}{BOP}{Balance of Plant}
\newacronym{bwr}{BWR}{Boiling Water Reactor}
\newacronym{cdt}{CDT}{Channel Distortion Testing}
\newacronym{ceng}{CENG}{Constellation Energy Nuclear Group}
\newacronym{clp}{CLP}{Core Loading Plan}
\newacronym{cmss}{CMSS}{Core Monitoring Software System}
\newacronym{cmr}{CMR}{Cycle Management Report}
\newacronym{colr}{COLR}{Core Operating Limit Report}
\newacronym{cpr}{CPR}{Critical Power Ratio}
\newacronym{crb}{CRB}{Control Rod Blade}
\newacronym{crsp}{CRSP}{Control Rod Sequence Package}
\newacronym{cs}{CS}{Core Spray}
\newacronym{cscs}{CSCS}{Core Standby Cooling System}
\newacronym{dba}{DBA}{Design Basis Accident}
\newacronym{dlo}{DLO}{Dual Loop Operation}
\newacronym{eal}{EAL}{Emergency Action Level}
\newacronym{ec}{EC}{Engineering Change}
\newacronym{efph}{EFPH}{Effective Full Power Hours}
\newacronym{ellla}{ELLLA}{Extended Load Line Limit Analysis}
\newacronym{eoc}{EOC}{End of Cycle}
\newacronym{eor}{EOR}{End of Rated}
\newacronym{epri}{EPRI}{Electric Power Research Institute}
\newacronym{fcl}{FCL}{Flow Control Line}
\newacronym{fcv}{FCV}{Flow Control Valve}
\newacronym{ferc}{FERC}{Federal Electric Regulatory Commission}
\newacronym{ffmt}{FFMT}{Failed Fuel Monitoring Team}
\newacronym{fri}{FRI}{Fuel Reliability Index}
\newacronym{fw}{FW}{Feed Water}
\newacronym{hepa}{HEPA}{High Efficiency Particulate Air}
\newacronym{ica}{ICA}{Item Control Area}
\newacronym{icf}{ICF}{Increased Core Flow}
\newacronym{lhgr}{LHGR}{Linear Heat Generation Rate}
\newacronym{lprm}{LPRM}{Local Power Range Monitor}
\newacronym{inpo}{INPO}{Institute of Nuclear Power Operations}
\newacronym{lfpc}{LFPC}{Loss of Full Power Capability}
\newacronym{loca}{LOCA}{Loss of Coolant Accident}
\newacronym{lpf}{LPF}{Local Peaking Factor}
\newacronym{maprat}{MAPRAT}{Maximum Average Planar Linear Heat Generation Rate Ratio}
\newacronym{mcpr}{MCPR}{Minimum Critical Power Ratio}
\newacronym{mellla}{MELLLA}{Maximum Extended Load Line Limit Analysis}
\newacronym{mflcpr}{MFLCPR}{Maximum Fraction of Limiting Critical Power Ratio}
\newacronym{mflpd}{MFLPD}{Maximum Fraction of Limiting Power Density}
\newacronym{msl}{MSL}{Main Steam Line}
\newacronym{nei}{NEI}{Nuclear Energy Institute}
\newacronym{ni}{NI}{Nuclear Instrumentation}
\newacronym{npsh}{NPSH}{Net Positive Suction Head}
\newacronym{oca}{OCA}{Owner Controlled Area}
\newacronym{ol}{OL}{Operating Limit}
\newacronym{olr}{OLR}{On-Line Risk}
\newacronym{oprm}{OPRM}{Oscillation Power Range Monitor}
\newacronym{pce}{PCE}{Personal Contamination Event}
\newacronym{plr}{PLR}{Part Length Rod}
\newacronym{ppe}{PPE}{Personal Protective Equipment}
\newacronym{rbm}{RBM}{Rod Block Monitor}
\newacronym{rcms}{RCMS}{Rod Control Management System}
\newacronym{rps}{RPS}{Reactor Protection System}
\newacronym{rr}{RR}{Reactor Recirculation}
\newacronym{rwe}{RWE}{Rod Withdrawal Error}
\newacronym{scwe}{SCWE}{Safety Conscious Work Environment}
\newacronym{slo}{SLO}{Single Loop Operation}
\newacronym{tls}{TLS}{Thermal Limit Set}
\newacronym{wano}{WANO}{World Association of Nuclear Operators}


\begin{document}
\tableofcontents
\clearpage

\section{BWR Stability}

\subsection{Oscillation Power Range Monitors}
The \gls{oprm}s are a system that initiates scrams if local instabilities are detected in the neutron flux, which is usually caused by thermal-hydraulic instabilities. The \gls{oprm}s are fed data by groups of \gls{lprm}s (cells) that are better able to detect local instabilities than the core-averaged \gls{aprm} readings (\gls{lprm}s don?t feed directly into the \gls{rps}, but feed in through the \gls{aprm}s, \gls{oprm}s, and \gls{rbm}s). Like the \gls{aprm}s, the \gls{oprm}s are not physical detectors, but rather process data provided by groups of \gls{lprm}s. The grouping of \gls{lprm}s to \gls{oprm}s is different from the grouping of \gls{lprm}s to \gls{aprm}s. With the intention of preventing false triggering of scrams, the \gls{oprm}s are usually off, but will automatically turn on when flow is less than 60\% rated drive flow and for greater than 28.1\% power. These flow and power indications are provided by the \gls{aprm}s (\gls{aprm} flows and power). 
\subsection{Power to Flow Map}
\gls{bwr}s exhibit flow instabilities at high power and low flow conditions. In these conditions, the high powered fuel will transfer a lot of heat to the coolant, causing an increase in the void content and a decrease in density of the single-phase region, both of which add negative reactivity to the core. This negative reactivity addition will cause bundle power to decrease, which will then cause the coolant to revert to a higher density, causing a positive reactivity addition due to the increased moderation. This can lead to power oscillations with periods on the order of 1-2 seconds and power increases from 20\%-140\% power. \gls{oprm}s protect against these power oscillations, while the \gls{aprm}s and \gls{rbm} protect against less non-oscillatory power excursions. The set points of the protective scrams or rod blocks is shown in Table \ref{table:1}.

\begin{table}[h]
\caption{Set points of the protective scrams or rod blocks, where \(W\) is the percent of rated drive flow.}
\centering
\begin{tabular}{l l}
\hline\hline
Scram type & Setpoint\\ [0.5ex]
\hline
APRM high flux 				& \textgreater 120\% power\\
APRM flow-biased scram 	& \(> 0.61W+68.2\%\) and \(< 115.5\%\) power (SLO)\\
APRM flow-biased scram 	& \(> 0.54W+55.9\%\) and \(< 112.3\%\) power (DLO)\\
APRM high flux (mode 2)	& \textgreater 20\% power\\
RBM rod block					& \(> 0.54W+44.7\%\) power (SLO)\\
RBM rod block					& \(> 0.61W+56.9\%\) power (DLO)\\
OPRM turns on					& \textgreater 28.1\% power and \textless 60\% rated drive flow\\
\hline
\end{tabular}
\label{table:1}
\end{table}

Flow has a linear effect on power once outside the natural circulation regime. Knowledge of the rod line is important for understanding reactor power response when decreasing flow. Moving control rods moves vertically on the power-flow map. Adjusting flow does not move you horizontally, however, which is why the power-flow map is important, since the relationship between power and flow depends on the rod pattern. Rod line, also known as the \gls{fcl} or the load line, is essentially a measure of how much of your power control is due to flow as opposed to control rods. The 85\% rod line corresponds to the line that intersects the 85\% power, 100\% flow point. The remainder of the line is traced out by varying flow and observing the effect on power. At the same flowrate, the more control rods that are inserted, the lower the rod line. Hence, higher load lines indicate that a greater portion of your power is controlled by flow rather than by control rods. 

The new 10 \(\times\) 10 GNF-2 fuel actually causes rod line to slightly increase as flow is decreased due to partial length rods and the water box design. This means that, when flow drops to about 70 Mlb/hr, the bundle is more efficient, and power can be increased naturally due to flow design. 

The following sections describe the characteristics of the power-to-flow map.

\subsubsection{Increased Core Flow}
Towards the end of a fuel cycle, most plants are licensed to increase to about 105\% rated core flow to add additional positive reactivity to compensate for quickly decreasing reactivity during the coastdown period. \gls{icf} can also be utilized at the end of a sequence exchange to maintain power without changing rod pattern. This region is still bounded by the 100\% power line and the pump cavitation line. This is why the power-to-flow map goes above 100\% rated flow along the \(x\)-axis.

\subsubsection{MELLLA}
The \gls{mellla} is an investigation that has been performed to develop the power-to-flow map for each power plant. The bounds of this study is the \gls{mellla} line - analysis beyond this power was not performed, and hence from a conservative perspective the plant should not operate outside the domains of the study. The 113.2\% rod line is the \gls{mellla} line beyond which the reactor is not analyzed. This line tapers at 100\% power, since clearly reactor power shouldn?t exceed rated. \gls{mellla} is an upgrade to \gls{ellla} that was needed following power uprates, since without re-analyzing some greater window, the flow window decreases, making operation at an uprated, new full power, much more difficult. The flow window is the horizontal portion of the \gls{mellla} curve (at 100\% power) that represents the allowable flows that can be used to attain full power. Some plants have a fairly large flow window in the range of 90-115 Mlb/hr, but other plants have very small flow windows. 

\subsubsection{Natural Circulation Line}
The natural circulation line represents the path that would be followed if neither \gls{rr} pump were on, a condition that would never be observed in a startup, since in startups, both pumps are in low speed with \gls{fcv}s in the full open position. However, this line could be observed in a slow shutdown with both \gls{rr} pumps tripped. Up to about 20\% power, greater density differences between the core exit and annulus regions promote increased core flow without an adjustment to \gls{fcv} position. At about 20\% power, the two phase pressure drop becomes substantial, and natural circulation can no longer be maintained. 

\subsubsection{Pump Cavitation Line}
The pump cavitation line is a line tangent to about the 54\% rod line that you should not operate below to prevent jet pump and \gls{rr} pump cavitation. The temperature in the \gls{rr} loop is directly dependent on the \gls{fw} flowrate, since higher \gls{fw} flowrates mean that more of the water in the \gls{rr} loop is subcooled, while a smaller fraction is saturated liquid that has not proceeded through the \gls{msl}s. The higher the core flow, the greater fraction of \gls{rr} loop water that is subcooled, and the greater the margin to pump cavitation. A 10.1 \degree F margin to pump cavitation must be maintained. During cavitation, bubbles form in the low-pressure regions around pumps due to the liquid reaching vapor pressure. When these bubbles migrate to higher-pressure regions, they collapse and produce large shock waves. 

\subsubsection{Regions 1 and 2}
Region 1 is the immediate scram region, while Region 2 is the immediate exit region, which can be accomplished by inserting CRAM rods or increasing flow. When attempting to exit the instability region, don?t increase flow beyond 57 Mlb/hr, since you won?t have time to assess the thermal limit impact. The instability region occurs > 80\% rod line and below 45\% flow. 

\subsection{March 1988 Instability Event}
With the unit at full power, an IM technician accidentally caused an erroneous reactor water level high alarm to go off, which initiates an ATWS-RPT, causing both \gls{rr} pumps to trip off. Interlocks prevent the operators from restarting either of the \gls{rr} pumps. Five minutes after the \gls{rr} trips, \gls{lprm} high alarms come in. Several minutes later, the reactor scrams on high flux at 118\%. The power oscillations were fast enough to avoid the \gls{aprm} flow-biased scrams. The \gls{aprm}s had oscillated with 1-2 second periods on a core-wide scale. However, the \gls{lprm} oscillations had much shorter periods and higher magnitudes, due to the cancelling effects of all the \gls{lprm}s contributing to a single \gls{aprm}. It was later shown that MCPR had not been violated. This event was partially caused by GE not having obtained reactor data in the instability region because they did not anticipate entry into this region. The interim solution was to use the power-to-flow map to avoid the instability region before the reliability of \gls{oprm}s could be trusted to not produce spurious scrams. 

\section{Control Rods}

The functionality of the control rods ensures shutdown margin can be met in all conditions. High worth control rods could lead to as low as a 50 second period when withdrawn, and so a great deal of effort is spent ensuring the reliability of the control rod system. This section will describe the specifics associated with BWR and PWR control rods, and their associated support systems. 

Control rods are manufactured by either GE (BWR rods) or Westinghouse (BWR and PWR rods), and the designs of the rods differ substantially between BWRs and PWRs, but the two variations of the BWR control rods are fairly similar. A core will typically consists of a mix of different control blade models as older blades are incrementally replaced as needed. For instance,  has eight different control blade models in their core (a mix of GE and Westinghouse blades of various models). 

The movement and operation of the control rods is performed with the desire of obtaining a haling distribution. This means that the control rods are operated such that the core obtains approximately the same axial power shape throughout the entire cycle. While this is not difficult to obtain for PWRs, this can be challenging to design for BWR rods. 

\subsection{BWR Control Rods}
BWR control blades contain stainless steel tubes in a cruciform shape with \(\textrm{B}_4\)C (natural B, which is 20\% B-10) powder. BWR control rods have cruciform shape in order to obtain a large surface area, which increases their reactivity worth while also helping better distribute the absorber through the fuel matrix. Each blade touches four fuel assemblies. 

All blades contain rollers of some form that touch the channels when the blade is inserted, helping the control rod move evenly up its path. Because the control blades touch the assemblies, structural changes to the channels or blades can have a large impact on the ability to insert reliably.

The control rod density in the core is calcualted as the number of notches inserted (24 notches/rod) divided by the total number of notches. In BOC, the control rod density is about 10-12\%. In the beginning of a fuel cycle, the poisons deplete faster than the fuel is consumed, and so control rods actually have to be inserted to maintain constant power. The maximum control rod density will be on the order of 15-16\% at 1/3 to 1/2 the length of the fuel cycle. In BOC, it is better to have a bottom-peaked core, since this will burn the fuel such that the burnup decreases the risks of power peaking later in the cycle once the burnable poisons begin to deplete and control rods need to be withdrawn. Shallow rods are withdrawn as soon as possible to ensure that withdrawal later in cycle will not suddenly uncover essentially “fresh” fuel. 

\subsubsection{Core Layout}
There are 185 control blades in  core, where each controls four assemblies. There are 764 total assemblies, meaning that there are 24 peripheral bundles that are not part of a control cell.

During refueling, at least two bundles diagonally must be in a control cell to support a control blade and prevent it from falling over. To ensure blade stability, a double- or single-blade-guide can be inserted into the core location. A blade guide essentially looks like a channel box, and simulates a fuel assembly supporting a control blade, since control blades touch channel walls as they are inserted into the core.

\subsubsection{Control Rod Sequences}
BWR control rods are intended to remain inserted during the entire fuel cycle, though with a density between 10-15\%. Control rod sequences describe the procedure by which to change the control rod pattern in the core, either to ensure even burnup by swapping patterns about once every 3 months (sequence exchange), to approach critical, to shutdown, or to perform settling testing. 

Target rod patterns are developed by the fuel vendor. The rod pattern should be designed as close as possible to that provided by the vendor, otherwise the power shape could be too top-peaked, resulting in less Pu-239 buildup from fast neutron absorption in U-238, which could lead to a shorter fuel cycle. Alternatively, if the core is too bottom-peaked, then shallow control rods must be used, which can also shorten cycle length. 

The rules governing allowable control rod movements in a BWR are referred to as the Banked Position Withdrawal Sequence (BPWS). The 185 control rods in the  core are grouped into 10 groups, where there are restrictions on the order and how many groups can be withdrawn at any one time. 50\% of the control rods are in groups 1-4. 

\subsubsubsection{Sequence Exchanges}
Sequence exchanges occur about once every three months, with the purpose of changing the operating control rod core pattern to ensure even burnup. Control rod groups are arranged in checkerboard patterns in the core to keep relatively flat radial profiles as they are withdrawn, while keeping blade worths approximately equal. Spacing between rods in the same group reduces neutronic coupling so that shorter-than-expected periods are not obtained. 

\subsubsubsection{Approach to Critical}
One group of rods is withdrawn all the way out, and another group is notched from 00 to some intermediate notch. Then, nearing criticality, rods are pulled single notches. Criticality is reached when a period of 200-300 seconds is observed. These sequences are designed to keep reactor periods longer rather than shorter, and to avoid reaching criticality on a peripheral rod. Because xenon production is a function of the power profile, the xenon profile in a core is fairly low on the periphery, so withdrawing a peripheral rod can lead to a potentially high worth. Control rods should be inserted if the period drops below 50 seconds. A high notch worth is about 0.0008 \(\Delta\)k/k. 

\subsubsubsection{Shutdown}

\subsubsubsection{Settle Testing}
Settle testing is performed periodically to assess the reliability of control rod insertion. The time for a control rod to fall from one position to another is recorded, and correlated with friction coefficients to determine if the rod is behaving as expected. 

\subsubsection{Replacement Requirements}
Control blades are replaced once they reach an end-of-life criteria that is either based on a neutronic limit due to absorber depletion or a mechanical limit related to structural warping/loss of absorber material due to cracking. The large majority of failures are neutronic failures, both because mechanical failures are typically only discovered if there is chemical evidence and the replacement requirements are structued such that neutronic failure typically occurs before mechanical failure.

Control blades can be shuffled around in the core to optimize life. Blades are to be replaced if they exceed 90\% of their EOL criteria or if they exceed 37 years of in-reactor residence time. At Dresden, an OEM blade that had been in the core for 37 years experienced a bail handle crack in the SFP, prompting the 37 year maximum residence time. Once they have exceeded the EOL criteria, control blades (and channels) are put in the SFP for short periods before they are crushed and sent offsite in special SNM dry casks. 

\subsubsubsection{Neutronic Failure}
Neutronic failure occurs when any three-foot segment experiences a 10\% reduction in relative cold reactivity worth due to B-10 and Hf depletion.  The actual depletion cutoff depends on blade design, but at its lowest is 34\% B-10 depletion for the D-100 blades. This maximum B-10 depletion has steadily improved with advancements in blade design. 

\subsubsubsection{Mechanical Failure}
Most mechanical failures occur due to SCC, which can challenge the integrity of the tubes and their ability to reliably hold poison materials. If boron carbide leaches into the coolant, reactivity will increase. Mechanical limits account for:

\begin{enumerate}
\item Swelling \(\textrm{B}_4\)C
\item Tube pressurization due to the production of He-4 through neutron absorption by B-10 (B-10 + n  Li-7 +\(\alpha\))
\item Manufacturing flaws
\end{enumerate}

Halfnium is often included in blades to reduce the probability of mechanical failure, since Hf absorption does not prouce alpha particles and Hf has a lower swelling tendency. Hence, there are no mechanical limits for Hf blades. Hf blades are particularly advantageous because upon neutron absorption, they transmute to other isotopes of Hf which also have considerable absorption cross sections, allowing longer blade life. 

Like neutronic limits, mechanical limits for blades are expressed in terms of \% B-10 depletion, and at their minimum occur at 55\% B-10 depletion. 

\subsubsection{Manufacturing}

\subsubsubsection{GE Blades}
GE manufactures(d) Duralife, Marathon, and Ultra MD/HD (Medium Duty/High Duty) blades. Each blade has on the order of 84 rods, with 21 rods per 1/4 blade. 

The Duralife blades were the original blades manufactured by GE. These blades consist of four U-shaped stainless steel sheaths that enclose vertical absorber rods. These four sheaths are then welded to a central stainless steel rod. Holes in the sheath allow coolant flow. There are several models of this blade, all of which use hafnium except the D-100 and D-120 blades. Of problematic design are the D-100 blades, which were the original equipment manufacturer (OEM) blades installed at april. This design used stellite, which in radiation conditions results in Co-59 transmuting to Co-60, causing dose concerns. In addition, the D-100 blades have more restrictive mechanical limits because the stainless steel alloy is more susceptible to SCC, and the absence of hafnium results in greater swelling for the same reactivity worth. 

Marathon blades consist of square tubes welded together, without a stainless steel sheath. A pure stainless steel portion exists at the handle. The roller exists in a hole that spans the entire width of the blade. 

\subsubsubsection{Westinghouse Blades}
Westinghouse blades use horizontal absorber tubes within a stainless steel sheath. 


\subsubsection{Associated Systems}

\subsubsubsection{Control Rod Drive Mechanism (CRDM)}
The control rod drive mechanism is responsible for inserting and withdrawing control blades. The CRDM for a single control blade is housed within a guide tube, which is welded to the bottom of the RPV and allows the blade to extend up through the cruciform opening in the fuel support pieces. The control rod drive housing is the gridded lattice in the undervessel region that prevents control rod ejection. About 63 gpm (0.03 Mlb/hr) is used to cool the CRDMs. This water is taken from the Cycled Condensate (CY) tank, and is injected into the bottom of the guide tubes, which then flows up slowly into the reactor coolant system. The CY tank holds a surge volume of radioactive water that is of high enough purity for return to the reactor.

\subsubsubsection{Hydraulic Control Unit (HCU)}
High pressure water and nitrogen are used to insert a control rod. The pressurized nitrogen causes water in the accumulator to go up into the control rod to force the rod upwards. In shutdown conditions when the reactor is depressurized, there exists insufficient pressure to scram the rods, but this is not an issue because you would never be in the depressurized state without all control rods inserted. Under normal conditions, the scram valves are kept closed by the air in the scram air header. To initiate a scram, air is vented from the scram valves to open the scram valves. To insert a control rod, water is inserted from below the velocity delimiter, causing upwards motion, whereas for withdrawal, water is inserted from above, causing downwards motion. 

\subsection{PWR Control Rods}
PWR control rods are often referred to with a special name - Rod Control Cluster Assemblies (RCCAs). PWR control rods are spider-like stainless steel structures that contain an Ag-In-Cd neutron poison. The Ag and In have resonance absorption peaks, while Cd has a large thermal neutron cross section. 

These RCCAs are arranged into four ``banks'' that can be inserted, where the rods are arranged in a checkerboard pattern in the core so that not every assembly accepts an RCCA. This is an important consideration in fuel design, since assemblies to be placed beneath an RCCA must have open guide tubes (no WABA poisons). The shutdown bank is located primarily surrounding the Ring of Fire. The control bank may be used to compensate for various reactivity changes during operation. The RCCAs enter by gravity when power to a magnet holding them up is lost. The RCCAs enter assemblies through guide thimbles. Each fuel assembly contains 25 guide thimbles, where 24 accept an RCCA tube, with the center thimble used for detector instrumentation for calibration activities. 

\section{Diesel Generators}

The diesel generators will turn on to power the safety loads in a LOOP. Diesel fuel is stored in a day tank and a much larger tank underground. The day tank has enough fuel to power the DGs for several hours. The day tank is automatically refilled from the larger tank. This larger tank holds about seven days of fuel. 

\section{Nuclear Instrumentation}

\subsection{BWR Instrumentation}
\subsubsection{Traversing In-Core Probes (TIPs)}

The five TIPs are used to calibrate the LPRMs every 2000 EFPH. All TIPs pass through a common central channel (channel 10), allowing calibration of the TIPs to the heat balance power. The current supplied to the LPRMs is then changed so that their readings match the readings of the TIPs. 

TIPs are inserted into the core only for calibration purposes – they are stored within either a detector room or within the drywell (BWR-6’s only) to prevent depletion when not in use. Many inadvertent high dose accidents (up to 30 rem in some cases) have occurred due to personnel entry into the in-core detector room closely following a detector calibration. 

\section{Thermal Limits}
Thermal limits are chosen so that, in the worst-case \gls{dba}, if the fuel had before-hand been below the thermal limit, gross cladding damage would not occur. Thermal limits generally do not need to be surveyed below 25\% power due to the inherently greater margin. The limiting thermal limits all decrease with exposure, with a switch to a new \gls{tls} at about 5 months from \gls{eoc}. If any thermal limit is violated, it must be restored within 2 hours, or else the plant shutdown in 4 hours. First, several terms must be defined, as they are used in the calculation of thermal limits. The \gls{lhgr} (kW/ft) for a node is computed as:

\begin{equation}
\textrm{LHGR}=\frac{\textrm{power of highest powered fuel rod in a node}}{\textrm{node length}}
\end{equation}

Typical limits on the \gls{lhgr} are shown in Table \ref{table:2}. As can be seen, the \gls{lhgr} is generally required to be lower for assemblies with a greater number of pins. The limiting \gls{lhgr} changes with exposure, and generally initially increases with exposure (becomes less strict) as the pellet swells closer to the cladding, improving gap heat transfer. However, the gross behavior with exposure is a decrease in the limit. 

\begin{table}[h]
\caption{Typical limits on \gls{lhgr} for \gls{bwr} fuel.}
\centering
\begin{tabular}{l l}
\hline\hline
Fuel Type & Limit (kW/ft)\\ [0.5ex]
\hline
9\(\times\) 9, BWR 5/6 & 14.4\\
8\(\times\) 8, BWR 5/6 & 13.4\\
7\(\times\) 7, BWR 4 & 18.5\\
7\(\times\) 7 BWR 2/3 & 17.5\\
\hline
\end{tabular}
\label{table:1}
\end{table}

The \gls{aplhgr} is the \gls{lhgr} averaged over a node, and is equal to the sum of the \gls{lhgr} for each rod divided by the node length. \gls{aplhgr} decreases with exposure due to: 

\begin{itemize}
\item Buildup of gaseous fission products in the pellet-clad gap that decrease thermal conductivity
\item Crud buildup on exterior of assemblies, reducing heat transfer
\item \(ZrH_{1.6}\) cladding embrittlement
\item Buildup of stored decay energy in term of fission products
\end{itemize}

\gls{aplhgr} is also a function of flow conditions - for \gls{slo}, multiply the \gls{dlo} value by 0.78, since at lower flow, the average nodal powers will be lower. The \gls{mcpr} is the minimum of the ratio of the critical power to the actual power:

\begin{equation}
\textrm{MCPR}=\textrm{min}\left(\frac{\textrm{critical power}}{\textrm{actual power}}\right)
\end{equation}

For \gls{dlo} and \gls{slo}, the limiting \gls{mcpr} ratios are 1.13 and 1.15, respectively. The critical power is the assembly power required to reach transition boiling. The limiting \gls{mcpr} is a flow-dependent limit, due to the thermal-hydraulic dependence of the onset of dryout. In \gls{slo}, the core flow is lower, making entry into the instability region more probable, thus requiring a more stringent limit. The limit is also a function of the scram time speeds to notch 39, since the greatest difficulty in insertion occurs in the lower region of the core, and the ability to insert these rods quickly impacts how much margin must be given to \gls{mcpr} (Scram to 39 in < 0.875 seconds or < 0.672 seconds). The limit slightly increases with a pressure increase since the enthalpy of vaporization decreases with increasing pressure, resulting in less of an enthalpy change required to reach critical heat flux and thus allowing higher vaporization fluxes.

GE uses the GEXL correlation to determine the \gls{cpr}. When dryout occurs, the liquid film is separated from the rod surface, which decreases moderation and causes a corresponding power decrease. Due to the power decrease, the liquid will again contact the rod surface, leading to a power increase. GE defines dryout to occur once 25 \degree F oscillations are observed in temperature due to the moderator feedback effect of dryout. 

\subsection{MFLPD}

\gls{mflpd} is a thermal limit that is computed as the maximum ratio of the \gls{lhgr} to the limiting \gls{lhgr}. 

\begin{equation}
\textrm{MFLPD}=\textrm{max}\left(\frac{\textrm{LHGR}}{\textrm{LHGR}_{lim}}\right)
\end{equation}

Keeping this thermal limit below unity ensures that 1) peak cladding temperature is kept below 2200 \degree F to limit spontaneous runaway Zr oxidation, and 2) plastic strain < 1\%. As long as the plastic strain is kept below 1\%, differential thermal expansion of the fuel pellet and cladding will not cause cladding rupture. This is generally the second-most limiting thermal limit for BWRs (\gls{maprat} is the least limiting). 

\subsection{MAPRAT}

\gls{maprat} is computed as:

\begin{equation}
\textrm{MAPRAT}=\textrm{max}\left(\frac{\textrm{APLHGR}}{\textrm{APLHGR}_{lim}}\right)
\end{equation}

In the case of a \gls{loca}, the core is assumed to be essentially void of coolant, and so there is a loss of most convective heat transfer. Without this heat transfer, radiation becomes an important heat transfer mode, and if the bundle were at too high a power pre-\gls{loca}, then radiative heat transfer may not be sufficient to keep cladding temperatures below 2200 \degree F. A planar limit is used because it was found that in a \gls{loca}, cladding temperature primarily depends on the \gls{lhgr} of other pins at the same elevation, and is relatively independent of the actual radial power profile in an assembly due to the need to transfer energy from hotter to colder fuel rods in a \gls{loca}. Center rods in an assembly are most likely to exceed 2200 \degree F in a LOCA, even though the peripheral rods may have been at higher powers, because central rods can't radiate heat by as efficiently as outer rods. This limit is intended to prevent cladding from exceeding 2200 \degree F is the core were to completely void. This limit also ensures that the oxidation shall not exceed 17\% of the total cladding thickness. Highly oxidized cladding is very brittle, and if cooled rapidly, can fracture. 

\gls{lpf}s decrease over a fuel cycle. The higher the \gls{lpf}s, the lower the assembly power necessary to meet other thermal limits early in cycle, and hence once the delta between highest and lowest \gls{lpf}s decreases, the total assembly power can be raised, thus decreasing the ability for all rods to radiate heat in \gls{loca} conditions.

\section{MFLCPR}

The \gls{mflcpr} limit protects against dryout. Dryout is the evaporation of the thin liquid film that cools assembly surfaces past the boiling boundary, and depends strongly on the assembly history upstream of the dryout (of which assembly power is indicative) and less on local conditions such as heat flux. In addition, channel boxes lead all fuel rods in a bundle to have approximately equivalent flow characteristics, and hence if dryout occurs, it likely occurs for multiple rods. This limit is designed so that, in case of a worst-case \gls{dba}, 99.9\% of rods will not experience dryout. This is the most demanding limit for \gls{bwr}s. GNF-2 bundles have been analyzed for up to 60\% flow blockage of an orifice, demonstrating the large margin to \gls{mflcpr}. 




\section{Unfiled}

Average Power Range Monitors (APRMs) [BWR]

The six APRMs each use 21-22 LPRMs? data (129 total in APRMs) to provide representative core-average thermal power readings and trigger rod block and scram set points. 3D Monicore requires 75% of the LPRMs to be functioning, while Powerplex and the WCMS only require 50% operatbility. The APRM readings are continuously calibrated relative to the heat balance power calculated by 3D Monicore. A GAF is assigned to the APRM



This GAF must be within the range 0.98 ? 1.02, or else an APRM calibration must be performed. april has six APRMs, with three assigned to each of the two RPS channels. One APRM in each channel can be bypassed for maintenance. At least 14 LPRMs, with at least two at each level, are required for an APRM to be operable. The APRMs are not the principal power readings at low powers, since otherwise they would require frequent adjustments, since during a startup or shutdown, the reactor is not held in steady state for the heat balance to be very accurate. So, TBV position and motor driven feed pump speed are used as power indications at low power. During steady state operation, the heat balance power is more accurate than the APRM power, since the APRMs are calibrated to the heat balance power. However, in transients, because there is a time delay between heat generated in the pellets and macroscopic thermal-hydraulic changes, the APRMs are more accurate than the heat balance power. The APRMs also provide a reference power to the OPRMs. An APRM can be inoperable for 12 hours. If RPS trip capability can?t be restored within 1 hour, then you must scram. The APRMs are calibrated once every 7 days. 

Flow-Biased Scrams
Each APRM takes a flow reading from each RR loop, and sums these to get the total RR drive flow. This flowrate is then used to determine the location on the power-to-flow map to set the flow-biased scram points. The APRM neutron flux readings are electronically filtered with an RC circuit to reflect the time delay (must be less than 7 seconds) between heat generation in the fuel and transfer to the coolant. This flow-biased scram set point is necessary to trigger scrams such that a slowly-increasing neutron flux does not lead to different thermal-hydraulic conditions/instabilities (prevents against MCPR violation) that would instead be regarded as neutronically safe if only the high flux APRM scram points were used. Instabilities can be caused by thermal-hydraulic changes before the neutron flux becomes excessively high, so a separate flow-biased scram point is used, where the alarm threshold is a function of both power and RR flow. This scram set point occurs before the neutron flux high point only when thermal-hydraulic conditions change slowly. Otherwise, for rapid changes in flux, the neutron flux high alarm will occur first. The flow-biased scram and rod block points are nearly parallel to the MELLLA line, but top off at a maximum power. 

Scrams (RPS)
High neutron flux
> 120% power
Flow biased
> 0.61*W + 68.2% and < 115.5% power (DLO)
> 0.54*W + 55.9% and < 112.3 % power (SLO)

High neutron flux (Mode 2)
> 20% power
Rod Blocks (RCMS)
Flow biased
> 0.61*W + 56.9% power (DLO)
> 0.54*W + 44.7% power (SLO)
Mode 1 downscale
< 3% power
W = percentage rated drive flow

Downscale alarms protect against open circuits, which would cause readings to be lower than actual. The rod block signals have identical slopes to the scram points, but occur at lower powers, so first a rod block is received, and then a scram if power continues to increase. 

If the APRMs are reading higher powers than the heat balance power, then actions can be delayed 12 hours, but only 2 hours if the APRMs are reading lower powers than the heat balance power. 
Boric Acid Reactivity Control [PWR]

It costs about \$63.43/50 lb bag of pure boric acid, which is then injected into the primary coolant system in the form of boric acid, of H3BO3. 


BWR vs. PWR


BWR (april)
PWR
Operating pressure (primary)
1020 psig (1005 actual)
7 MPa
2200 psig
Operating pressure (secondary)
---
1000 psig
Core flowrate
108.5 Mlb/hr, up to 105% for EOL

Rated generator output
1212 MWe

Rated power
3546 MWth (~1180 MWe)

Fuel cycle length
24 months
18 months
Containment
BWR 4, Mark 2


BWRs are safer than PWRs due to their much lower power density and lower required emergency core cooling flowrates ? this results in a core damage frequency about an order of magnitude smaller than that in PWRs. The boiling process in BWRs tends to concentrate more impurities onto the upper assembly area. However, PWRs have slightly better fuel utilization. 
Channels

Channels are about 2.5 mm thick Zircaloy walls attached to the upper tie plate of BWR assemblies with a fastener on one corner. This fastener is used to determine the correct bundle orientation. The channels may be thicker in the corners (interactive channels) and other high-stress regions. Channels were initially included in the BWR design to ensure that high void content could not easily move around the core in transients. Channels direct flow through bundles while allowing flow past the NIs and control blades. In addition, channels act as heat sinks during LOCA scenarios, since they themselves have no heat generation term. There is only about a 1 C temperature gradient across channel walls. 

Channels are made of either Zr-2, Zr-4, or NSF. Channels were initially made of Zr-4, but then Areva, GE, and Westinghouse all switched to Zr-2 due to its better corrosion resistance. However, as fuel cycles became longer and core power densities were increased, bundles needed to be controlled early in life, increasing the likelihood of shadow corrosion. Also, due to the longer fuel cycles, shadow corrosion-induced bowing became even further amplified. So, because Zr-4 has a lower hydrogen pick-up (HPU) rate than Zr-2, channels are now made of Zr-4, even though it has worse corrosion properties than Zr-2. Both Zr-2 and Zr-4 channels can be found in april?s core.  

Zr-2
?	Material traditionally used in BWR cladding
?	Better overall corrosion resistance than Zr-4
?	Higher HPU rate, leading to more zirconium hydrides, and increased shadow corrosion rates.
Zr-4
?	Material traditionally used in PWR cladding
?	Contains less nickel than Zr-2
NSF
?	Niobium, tin, and iron alloy being developed by GE

There is only about 2-3 mm clearance between channels and control blades, so the mechanical form of the channels is important to the reactivity control of the reactor. The rate of deformation is primarily affected by operational decisions that increase burnup and introduce flux gradients (control rod presence and bundle-to-bundle power differences). To mitigate channel deformation, around 50-100 assemblies are rechanneled each outage. There are three primary degradation mechanisms for channels:

1)	Bow: A fast flux gradient, which is commonly found on core peripheries, causes the channel to bow in the axial direction. This gradient leads to differences in energy deposition due to fast neutron interactions. The channel side facing the higher flux region receives more energy from neutrons, causing more atoms to be knocked from the lattice, leading the channel to bow outwards in the direction facing the high-flux region, with the maximum bow occurring at the axial mid-plane. Bowed channels change the coolant flow path, and could lead to higher bypass flow or allowing more flow past a corner rod, which would improve moderation and lead to higher corner peaking factors, potentially to the point that dryout occurs. Rotating the assembly or channel or translating an assembly counter-core during refueling could alleviate this issue. About half of the US BWRs have reported bow-induced interference with control blades. 

2)	Bulge: Channel walls become rounded out in the transverse direction due to differences in water pressure in and outside of the channel. This deformation mechanism is unavoidable, and some will always occur, which can be detrimental when combined with other failure mechanisms. The P between the inner and outer walls of the channel is greatest at the bottom of the assembly. 

3)	Shadow Corrosion: When Zircaloy is in close proximity to the stainless steel in control blades, the rate of ZrH1.6 growth becomes accelerated due to increased HPU rates. The corrosion rate is proportional to the distance between the blade and channel. Because the density of ZrH1.6 is smaller than that of Zircaloy, channel bowing occurs due to differential wall growth as a consequence of shadow corrosion. While shadow corrosion will occur early in life, the bowing will not occur until high burnup is reached. Shadow corrosion is ?activated? by inserting a control blade early in the operating life of a channel (first six months in the core). The channel is less susceptible to picking up hydrogen once a protective zirconium oxide layer, which has lower hydrogen pick up rates, has built up on the exterior surface. Performing a quarterly sequence exchange helps manage shadow corrosion, because then no single bundle is controlled for very long periods of time. 

Deformed channels can:
1)	Interfere with control rod motion and yield increased settle times, slower scram times, and potentially cause an ATWS
2)	Contact LPRM tubes, affecting readings due to the loss of water gap symmetry around the detectors. For neutrons to be detected by the LPRMs, they must be thermalized, so if the moderator symmetry around an LPRM changes, then the reading will be skewed, since neutrons from one bundle will not produce as high a signal as they should. 
3)	Produce different flow paths, leading to different thermal limit margins. Increased bypass flow results in less assembly flow, which can cause a MCPR violation. In addition, corner pins might see higher flowrates, leading to higher peaking factors that may exceed limits. 

If deformed channels are detected during Channel Distortion Testing (CDT), then the control rod must be fully inserted for the remainder of the cycle, which can have a negative economic impact. Sometimes, channel distortion can be reversed by inserting a control rod for some period of time. 
Chemistry Programs [BWR]

The chemistry program protects:
1)	System components ? Corrosion of primary systems is limited to avoid LOCA breaks, while corrosion of BOP systems is limited to protect components and reduce the probability of HELBs, which at one nuclear plant have killed four people after a steam line break. 
2)	Fuel cladding integrity ? Corrosion of cladding is limited.
3)	Out-of-core radiation fields ? Reduce corrosion to limit corrosion products that can become activated as they circulate through the core.

HWC: Hydrogen Water Control
Hydrogen is added to the reactor coolant through the FW lines to maintain reducing conditions, which overall reduces corrosion and SCC. Adding hydrogen also reverses some effects of water radiolysis. 

NMCA: Noble Metal Chemical Addition
NMCA is an additional corrosion control method to the HWC program that injects hydrogen at lower concentrations than HWC with noble metal treated surfaces. Simply adding hydrogen to the coolant will increase dose rates due to additional radioactive byproducts that are produced. So, Noble Chem injects some hydrogen and noble metals such as platinum and rhodium that, by catalytic action, remove oxygen at RPV internals surfaces. In addition, OLNC helps reduce IGSCC, since these noble metals plate out on surfaces. Noble Chem will cause crud redistribution in the core, changing the axial power profile.

The noble metals provide catalytic surfaces for recombination reactions of hydrogen and oxidants. Catalysts are intermediate compounds that have lower activation energies than the intended products, allowing a reaction to occur more quickly if it is statistically more likely that smaller energies be available for transfer (don?t need to ?wait? for a large energy transfer for the interaction to occur ? you can proceed with lower energies, even though the total energy needed to traverse from reactant to final product may be the same). 

Zinc Injection
Zinc is injected to reduce dose rates within the drywell due to RR system piping. Zinc is strongly incorporated into the protective oxide film that forms on stainless steel surfaces. This oxide layer releases cobalt alloys at lower rates, reducing dose, since less Co-59 travels through the core to become activated to Co-60. This can lead to a 2-4x reduction in Co-60 concentrations in the coolant and a savings of 100-200 rem/outage.

This dose-reduction effect of zinc addition was discovered at several BWRs that naturally had much lower dose rates than similar plants due to their inclusion of brass condensers, which release zinc into the coolant. Hope Creek was the first BWR to intentionally add zinc to the coolant, and now most BWRs maintain zinc in 5-15 ppb in the coolant to reduce the amount of Co-60 in the iron oxide corrosion layers in the RR piping.  

Condensate System [BWR]

Reactor water enters the condenser and condenses on the outside of tubes through which lake water (Circulating Water (CW) System) passes. There is one condenser for each turbine (?). The water then leaves the condenser, passes through several condensate booster pumps, and then through FW heaters. Then, FW pumps take the condensate pump outlet 600 psi up to 1005 psi reactor pressure. 

FW Heaters
The heated water (saturated conditions) removed from the heater drain corresponding to the MSRs is used to preheat the reactor water from 120 F at the condenser exit to 423 F for inlet into the reactor annulus region. For the same reactor power, by using FW heating, an additional 1000 MWth can be extracted. The FW heaters are only turned on after the turbine has been synced. 

Loss of Condenser Vacuum
A loss of condenser vacuum pressure could be caused by SJAE failure or a loss of cooling water, in which case pressure in the condenser would increase due to less heat being removed from the primary water. For minor condenser vacuum decreases, there is a decrease in plant efficiency due to the smaller pressure drop over the low-pressure turbines. 

A loss of condenser vacuum will initiate a turbine trip and close the MSIVs and TBVs to prevent all possible paths of coolant to the condenser. The cooling lake inlet temperature cannot exceed 102 F, or else you need to begin shutting down, a process which has occurred at april before. The circulating water inlet is about 80 F during the summer, with an outlet of about 105 F, meaning that the circulating water increases by about 25 F after cooling the reactor water. If the lake should fail, there is still enough water to cool the reactor for 30 days. 

Containment

BWR
The primary containment of a BWR consists of the drywell, which houses the reactor, the RR loops, other systems, and the suppression pool/chamber. The primary containment is designed to withstand all peak temperatures and pressures associated with a DBA, and also fluid jet forces impinging on the walls. Mark 1 designs have a torus that surrounds the base of an upside-down light bulb drywell design. Mark 2 designs have an over-under style, where the suppression pool is located below the drywell.  For Mark 3 designs, the suppression pool is located outside the primary containment, and thus is not technically a part of primary containment. 

Drywell
The drywell is a concrete structure lined with steel that encloses the RPV and RR loops and all other RPV penetration systems. The steel provides the actual air barrier, while the concrete supports it. During operation, the drywell is inerted with nitrogen to limit the likelihood of a hydrogen explosion if an accident were to occur during power operation. Containment inerting is required for all BWRs with Mk 1 and Mk II containments following the TMI accident, since it would only take a small amount of zirconium-steam reaction to produce hydrogen levels above 4% in these volumes due to the small drywell volumes. Nitrogen is used for inerting because air is already primarily nitrogen, so adding more nitrogen easily reduces oxygen concentrations. Nitrogen is also cheaply available. Other gases have been investigated for containment inerting, such as CO2 and halon. Many plants have CO2 fire fighting systems that would not be too difficult to connect to inert the drywell. However, CO2 could potentially decompose into oxygen through radiolysis, leading to an increase in oxygen. Halon is also used in fire suppression systems, though in radiolysis, breaks down into hydrofluoric acid. 

The drywell floor can sustain a maximum pressure of 25 psi above normal and earthquake operating loads. In addition, the drywell can sustain an acting upwards pressure of 5 psi due to suppression pool pressure increases during a LOCA as water flashes to steam. In case of accident conditions, the water can fill the drywell to 1 foot below the refueling floor to permit safe removal of displaced fuel assemblies. There is no single DBA for the containment system, but rather the design criteria are developed from an envelope of accident conditions. 

Biological Shield
A concrete cylindrical shield surrounds the RPV within the drywell, reducing dose and activation of the drywell. 


Control Rods [BWR]


About 60-80% of the neutrons absorbed by the control rods are thermal. The melting point of boron is 2400. 


These more restrictive limits are accounted for by washout and tip depletion adders. These adders are added to the tracked blade depletion and ensure that there is some minimum recorded depletion for the D-100 rods, adding conservatism. The washout adder restricts the possibility of IASCC that would result in B4C leaching from the rod. The cracking threshold is 20% B-10 depletion. The leaching threshold is 25% B-10 depletion. The tip depletion adder accounts for the fact that, even when the D-100 control blades are fully withdrawn, the top node still experiences relatively high radiation flux, and thus depletes and experiences greater mechanical stresses. The hafnium that is included in more advanced blade designs is found in the top node of the blade, which removes the necessity of the tip depletion adder. The magnitude of this adder depends on whether the fuel has a natural uranium blanket on the bottom. 


CRAM Rods
CRAM rods are high-worth control rods that can be rapidly inserted to reduce power in case a transient is approached, such as entry into Region 2 on the power-flow map. These rods do not need to be inserted in any particular order, because if you need to rapidly insert rods, and one does not insert for some reason, you don?t want to be administratively stuck in the sequence until the issue is resolved. 

Rod Worth
The differential rod worth is the notch, which is measured by inserting and withdrawing control rods in single notch increments and measuring the reactor period. The integral rod worth is the  when the rod is inserted up to a particular notch. 

Rod Shadowing
Moving one control rod impacts the worth of another control rod. 

Rod Types
Deep rods typically have a large impact on the overall power of the reactor, but not so much on the actual flux shape. Shallow rods can greatly impact the local flux shape with little impact on the overall core power. 

Haling Distribution
A more modern approach is to purposefully design a bottom-peaked core in BOL to let Pu-239 breed in the top of the core, and then transition to a top-peaked core at EOL. But, operating with a top-peaked core is more difficult because the control rods enter from the bottom. 

Alternate Rod Insertion (ARI)
If the control rods fail to scram, the operator can place the reactor switch in shutdown and then press the manual scram buttons. However, if this also fails, then ARI is initiated, which inserts control rods independent of the RPS. ARI does not vent each scram valve individually, but vents the scram air header. Rods can be individually driven or scrammed into the core using controls that are normally used during scram time testing. If the failure to scram is electrical in nature, the scram pilot solenoids can be de-energized. As a last resort, the HCU overpiston area can be vented to the radwaste drain, and this pressure differential is used to insert the rods. However, this should not be the first course of action, since this can easily spread contamination. 


Core Design

BWR
Fuel in a control cell one cycle does not necessarily move together for the next cycle. Bundles in one quadrant of a control cell are also not restricted to that control cell for their entire lifetimes. 

PANACEA is GE?s core simulator code, which uses three-group diffusion, collapsed to 1.5 group diffusion after assuming that all groups have the same buckling. The node size of control rods  is set to six inches based on the mean free path of fast neutrons. 

PWR
Fuel with lower burnups is typically moved next cycle under RCCA locations to reduce the likelihood of IRI (incomplete rod insertion). 
Diesel Generators (DGs)


BWR: Each unit at april has two DGs, with an additional common DG. During the LOOP event in 2013, the common DG naturally split its load between both units. One DG powers HPCS, while the other powers LPCI, LPCS, and RHR. If the DGs fail, then RCIC can provide cooling for eight hours, being powered by DC batteries. Past those eight hours, however, a meltdown will occur if no additional offsite power is brought in. 

Dose

The contact dose of even small fuel pellet fragments can easily exceed 1000 rem/hour. 

BWRs can significantly reduce dose by vacuuming guide tubes, where debris collects below the velocity limiter, where it becomes activated. 
Electricity Markets

During an electricity auction, the operator of the grid will determine how much electricity it will need to keep a stable grid and meet demand. Then, various power suppliers will bid in until the quota is reached. The suppliers tell the grid operator how much they charge per MWe, and the grid operator purchases power from the lowest bidders. The actual price received by the supplier for their electricity is then determined solely by the averaged cost of electricity of all the bidders with low enough prices to be below the upper quota. All electricity suppliers then receive the same profit rate, which is why natural gas suppliers are driving down electricity prices. Power can be transmitted efficiently up to about 500 miles. 

Exelon produces electricity at about $31/MWhe, where that is divided into about $7/MWhe for fuel, $16/MWe for personnel, and $8/MWhe for capital costs. Part of the reason why Quad Cities and Byron are close to being shutdown is due to wind subsidies in Illinois. Wind providers are guaranteed to receive $25/MWhe, regardless of the actual cost of electricity. So, if there is less demand for electricity, wind providers will still receive $25/MWhe, even though the cost of electricity may be reduced to negative cost. Fuel constitutes about 25% of the cost of a nuclear power plant, while fuel costs reach up to 80-90% of the cost of fossil fueled power production. 

Winter 2014 saw a 22% reduction in the expected MW output to the PJM grid due to the polar vortex and the rerouting of natural gas to be used for electricity production to heating people?s homes. PJM has thus assigned stricter outage penalties that should benefit the nuclear industry. 10 of Exelon?s sites are in PJM (Clinton is in MISO). PJM is responsible for maintaining the stability of the grid by purchasing sufficient power capacity and production at auctions. PJM provides money through capacity auctions and production auctions (?). In capacity auctions, utilities are rewarded simply for their capacity and ability to provide electricity to the grid, regardless of whether or not they actually do provide power. Production auctions then reward utilities for the power actually produced. 

Exelon has been changing their portfolio so that the dividends paid out to consumers used to be 75% generation profits, 25% transmission profits, where this has now been reversed. 

Low Carbon Portfolio Standard
Exelon attempted to pass this measure, which would require all utilities to provide 70% of their electricity production as renewable through the purchase of credits, which would benefit the company and other renewables. Clean coal is included as a green energy source in this legislation. 
Emergency Core Cooling Systems (ECCS) [BWR]

The BWR ECCS systems comprise 1) HPCS, 2) LPCS, 3) LPCI, and 4) ADS. These systems are designed to cool the core in case of an emergency and guarantee that hydrogen generation will not exceed 1% of the hypothetical amount if all the cladding were to react in the Zr oxidation reaction:

Zr + 2H2O  ZrO2 + 2H2 (explosive) + heat (additional cooling concern)

Zr oxidation is only significant above 2200 F, but still occurs at lower rates below this temperature. The ECCS systems will automatically start on low reactor water level or high drywell pressure. 

The ECCS systems are powered by the emergency diesel generators. All systems take suction from the suppression pool and inject into the reactor. Water lost from the reactor in LOCA conditions will drain through downcomers into the suppression pool, creating a closed loop for cooling. Jet pump integrity guarantees that the core can be reflooded to 2/3 height, and the remaining 1/3 of the fuel is cooled by the ECCS systems, as the water level equalizes in the annulus region to that in the core region, and exits through the break in the RR line (worst postulated break). RHR will then cool water in the suppression pool by passing it through a heat exchanger cooled by service water (?). The low-pressure injection systems continue automatically until manually stopped by operators, but the high-pressure systems will cycle on and off based on reactor level. All ECCS components are located near the suppression pool to provide sufficient head, and hence are protected with flood doors. 

HPCS (High Pressure Core Spray)
HPCS is used to maintain reactor water level and reduce pressure slightly for small breaks (< 0.7 ft2) to allow the low-pressure systems to inject. HPCS consists of a single motor-driven pump that can spray water into the core at any pressure, and will even operate at pressures too low for LPCS and LPCI. This pump is located below the suppression pool elevation to provide sufficient injection head from the suppression pool  or the CST (Condensate Storage Tank), the preferred water source. Water is injected from spargers located on a circular ring around the core periphery (128 spray nozzles). The HPCS flowrate is higher the lower the reactor pressure (the greater the break), and at zero reactor pressure, can inject a maximum of 7175 gpm. One DG is devoted to powering HPCS. 

ADS (Automatic Depressurization System)
The ADS consists of 7 of the 13 SRVs, and is used to depressurize the reactor when the coolant demand exceeds what can be supplied by HPCS, but the pressure is still too high to LPCS and LPCI to function. The ADS will initiate only if both HPCS and RCIC fail (which is indicated by low reactor water level or high drywell pressure alarms) and there is indication of sufficient injection head from at least one of the low-pressure systems. Upon initiation of ADS, the 7 SRVs will open and discharge steam from the MSLs directly to the suppression pool until the pressure is low enough for LPCS and LPCI to inject.  In an ATWS event, ADS initiation is inhibited, since this would lead to injection of cold unborated water into the core from safety systems, which would exacerbate reactivity control. 

LPCS (Low Pressure Core Spray)
LPCS consists of a single motor-drive pump that floods the reactor from a circular sparger around the periphery of the core inside the core shroud using water from the suppression pool. LPCS can begin to inject around 400 psig, and operates independently of LPCI. Both LPCS and LPCI provide much larger volumes of water to the core than the high pressure injection systems, since large breaks required large amounts of makeup water. 

LPCI (Low Pressure Coolant Injection)
LPCI is an emergency mode of the RHR system, by which RHR pumps are used to flood the reactor from water taken from the suppression pool. LPCI can inject through either the RR lines or directly to the core.
Emergency Core Cooling Systems (ECCS) [PWR]

Similar to a BWR, the ECCS systems take suction from the containment sump, which is located at the bottom of containment. This structure collects water in case of a large leak. 
Feedwater System (FW) [BWR]

The FW system consists of two turbine-driven reactor feed pumps (TDRFPs) and one motor-driven reactor feed pump (MDRFP) that increase FW pressure before being sent to the reactor through two FW inlets. Each FW inlet discharges water into the reactor at about 423 F (at full power) through three spargers, or perforated stainless steel headers that evenly distribute water without impinging it on the reactor vessel wall. This FW injects just above the fuel line. The total FW flow between the two inlets is 15 Mlb/hr. The FW flowrate is the most important input to the heat balance due to its very high flowrate.

FW flowrates used to be nearly exclusively measured with Venturi meters, which required additional uncertainty margin, since FW flowrate has the greatest impact on the heat balance calculation. Now, most plants have installed LEFMs, flow meters based on ultrasound. This allowed for a narrowing in the uncertainty window due to the more precise instruments, and so many plants that have installed LEFMs have undergone a Measurement Uncertainty Recapture (MUR) Uprate. 

MDRFP
The MDRFP is used to supply FW flow when the steam flow is insufficient to power the TDRFPs, and hence the MDRFP is used in startup and shutdown conditions. This pump has only two speeds, to flow control valves are used to control flowrate. The MDRFP can provide up to 32% rated FW flow (10,300 gpm). The MDRFP and both TDRFPs are equipped with strainers to prevent FME concerns, and if high differential pressure readings are noticed, an alarm will sound. The FW pumps (MDRFPs) are steam-driven by the main turbine because they are not considered safety related.

TDRFPs
The TDRFPs supply variable FW flows by using extracted main steam to power a turbine, which spins a pump. The normal operating speed at full power is about 4500 rpm for each turbine. Each can supply up to 68% of the rated FW flow. The pump speed is controlled by the speed of the steam turbine attached to each pump impeller. 

Level control sensors in the reactor control valves before the TDRFPs. A low reactor water level will manipulate this valve so that a higher FW flow results. During startup, this process is governed by single-element control, meaning that the feedback lags the actual plant conditions, which can lead to oscillations in the FW flow and wide swings in indicated reactor power (a high FW flow simply indicates high power due to its strong impact on the heat balance calculation). Cycling of these control valves for extended periods of time can damage the valves, which is why reactor pressure is not typically held at around 15% power for very long. At higher powers, three-element control is used, meaning that the FW flowrate response is more accurate and predictive, which makes the flow control valve position more steady. 

Loss of FW Heating
A loss of FW heating introduces positive reactivity due to an increase in inlet subcooling. In addition, there will initially be a decrease in water level due to ?shrink? at the now lower FW temperatures. 


Fire Protection Systems

The greatest risk of a fire in a nuclear power plant is ignition of the diesel fuel and damage to cable spreading rooms. Fire barriers can typically hold back fires for 1-3 hours. 

CO2
CO2 displaces oxygen so that combustion can no longer occur. 

Water
A ringheader is arranged around the plant, containing pressurized, stagnant water. Portions of the ringheader can be isolated while others can provide water to localized areas. In areas susceptible to freezing, the pipes are dry, but are filled with water by activating values once a fire is detected. 

Halon 1301
Halon stops the spread of fire by chemically disrupting combustion (displaces oxygen, causing dizziness and confusion). Halon 1301 is a CFC and significantly contributes to ozone depletion, so it hasn?t been manufactured since 1994. Instead, it is recycled and reused. Halon is used to protect electronics and records, since it is a gaseous extinguisher that leaves no residue, and is 2.5 times more effective than CO2. 

Foam
A protein-based foam protects the diesel fuel tanks. 

Fuel Failures

Fuel failures are undesirable primarily from a regulatory perspective, which has a strong economic impact:
1)	Downpower to perform PST
2)	Suppress leaker for remainder of cycle
3)	Extra investigation into the cause of the leak
4)	Shutdown if release rates are exceeded

PWRs are more tolerant of fuel failures from the dose perspective due to the protection offered by their secondary system. BWR coolant is less reduced than that in PWRs, which leads to more secondary degradation issues due to hydriding and oxidation. Only in very extraordinary cases could the increase in internal gas pressure due to fission gas buildup lead to a fuel failure. Overall, most failures are due to debris fretting, vibration, and foreign material. 

The radiation levels due to fuel failures are a combination of recoil, diffusion, and equilibrium activities:
1)	Recoil activity ? results from tramp uranium or washout uranium from previous large fuel defects
2)	Diffusion activity ? the local power level determines the diffusion rate of fission products through the cladding. A greater fraction of long-lived isotopes reach the coolant than short-lived isotopes. For medium-sized defects, there is no direct contact between the cladding and fuel.
3)	Equilibrium activity ? for small pinhole defects, the fission gases must first migrate out of the fuel pellet and then across the gap region to the coolant. A larger fraction of long-lived products reach the coolant than short-lived products. 

Hydriding
H2O decomposes into H2 and O2. The H2 then diffuses from the coolant into the cladding and forms compounds with ZrH1.6 stoichiometry:
Zr + 2H2O  ZrO2 + 4H
 Zirconium hydride is brittle and forms blisters on the fuel rod. ZrH1.6 has a higher specific volume that Zircaloy, so it deforms the cladding. Hydriding also exists as a secondary degradation mechanism from H2 diffusion from water than has seeped into an existing defect. 

PCI: Pellet-Clad Interaction
Stresses, such as rapid power increases, cause differential expansion of the pellet and cladding, causing the pellet to contact the cladding and impose expansion stresses on the cladding. Corrosive fission products such as I and Cd come in direct contact with the cladding, causing SCC. Stresses may be magnified if pellet cracking and shift of small pieces has already occurred within the bundle (such as during shipment or manufacturing). PCI becomes a greater concern at higher exposures due to the decrease in Zircaloy ductility. PCI was the main source of fuel failures in BWRs in the 1975-1985 timeframe, and in particular affects BWRs due to the high local power changes that occur during control rod movements. When PCI failures were first identified, GE released PCIOMR (Preconditioning Interim Operating Management Recommendations) that placed more conservative limits on power ramp rates before a more definitive investigation into reducing PCI could be performed. These fuel conditioning guidelines reflect that there is a low power threshold to PCI, and its occurrence depends on the rate and magnitude of a power change at a particular exposure. While these limits protect against PCI, they are expensive due to the required slow ramp rates during start-ups and sequence exchanges. 

GNF
0.35 kW/ft/hour
Areva
0.25 kW/ft/hour
Optima
0.12 kW/ft/hour

PCI can be reduced by:
1)	Chamfered-edge pellets to reduce stress risers that would tend to fracture off
2)	Following PCIOMR guidelines
3)	Barrier fuel ? the pure Zr inner lining of about 0.003? thickness is more resistant to corrosion. In addition, pure Zr is more ductile than Zr-2, allowing stresses to be absorbed more easily. Vendors initially thought that barrier fuel would solve PCI issues and eliminate the need for PCIOMR limitations. However, PCI failures continued to occur, so preconditioning limits are in place not as a vendor requirement, but as a recommendation. All fuel used at april is barrier fuel. 

CILC: Crud-Induced Local Corrosion
Cladding corrodes due to heavy crud deposits. This crud derives from other system components, such as copper deposits in condensers. CILC primarily affects Gd rods, perhaps because they are cooler, which will precipitate some crud out of the coolant onto the surfaces. Cooler regions attract more ZrH1.6, leading to hydriding corrosion and a reduction in heat transfer through the crud layer. 56 bundles failed at Limerick in 1998 due to crud ? the water had a high copper content. 

Grid-Rod Fretting
The grid spacers wears a hole in the fuel rod due to rod vibration. PWR fuel failures are overwhelmingly due to grid-to-rod fretting due to their higher core flow rates. 

Debris Fretting
Metallic debris becomes stuck in a spacer grid, where it wears a hole in the fuel rod due to rod vibration. The cause of BWR fuel failures tends to be more varied, but debris fretting makes a large contribution. Debris fretting in L1C13 caused a 7 inch axial crack that caused 7 entire fuel pellets to be washed out. Debris fretting used to constitute a much larger portion of fuel failures, but due to better FME practices and debris filters, has been reduced.

Manufacturing Defects
Defects in end plug welds, cladding, etc. typically manifest themselves early in the cycle. 

Handling Damage
BWR assemblies are somewhat protected by channel boxes, but when un-channeled, both PWR and BWR bundles are vulnerable to this damage mechanism. 

Secondary Degradation
Coolant enters a fuel rod through an existing defect. This water then reacts with the fuel and cladding to produce oxidation (U3O8, UO3, ZrO2) and hydriding (ZrH1.6) products. These oxidation and hydriding products are both more brittle and less dense, expanding the cladding while making it weaker. This exacerbates the primary failure and can lead to additional failures. This failure mechanism can be avoided by suppressing primary failures. Five BWRs in recent years have had to shut down due to high offgas due to secondary degradation.

Types of Fission Product Release
Recoil: There is no delay between fission events and release of fission products, since fission products recoil from the edge of the fuel directly through the cladding. Only this release type is directly proportional to power level. This release also characterizes activity from tramp uranium. 

Equilibrium: This type of release results from small pinhole defects, where there is a long delay between fission and release. Coolant and offgas samples will show higher amounts of volatile and long-lived products. This release rate is not directly proportional to the power level. 

Diffusion: Large defects expose the UO2 directly to the coolant. The release rate is usually exponential with power, since diffusion is exponential in temperature. 



Fuel Failure Detection and Remediation Methods

Fuel failures are primarily detected by chemistry samples or special power maneuvers that manipulate rates of fission gas release from a defect. Various ratios of shorter-lived to longer-lived isotopes drop when a fuel failure is present. Before the fuel failure, equilibrium conditions have been reached between the isotopes that can diffuse through the cladding. However, when a fuel failure occurs, it becomes much more likely that the longer-lived isotope escape, since now nuclides at farther distances from the break have enough time to escape before decaying. Increases in the activity of short-lived products indicates that a pellet is in direct contact with coolant. When a fuel failure initially occurs, it is likely a small failure, which leads to a slow depressurization of the rod as the fission gases are vented. After this venting, however, activity indicators may drop, seeming to contradict earlier data, but this may simply be due to fission gas venting completion.

Noble Chem may mask a fuel failure, since injecting hydrogen causes tramp uranium in the coolant system to dislodge and redistribute, momentarily leading to increased offgas activity. 

Sum of 6
Xe-133, Xe-135, Xe-138, Kr-85m, Kr-87, and Kr-88 are gaseous fission products that are continuously detected through the offgas system due to diffusion through cladding. A step change in this activity indicates a fuel failure. A drop in the ratio of Xe-138/Xe-133 (14 min/5.25 days) below 300 indicates a fuel failure. 

Reactor Coolant Samples
Fission products can enter the coolant from tramp uranium, diffusion, and fuel failures. I-131, I-132, I-133, I-134, and I-135 are soluble fission products that can be detected in the coolant. Most other halogens are also soluble in the coolant. Without a fuel defect or significant tramp uranium, I-131 is initially nearly undetectable. If a fuel failure is known to exist, then concentrations of Cs-137 and Cs-137 in the coolant can be used to gauge the exposure of the failed fuel, since the concentrations of these two nuclides depends on time. Np-239 is also detectable in coolant samples. An increase in Np-239 activity implies that the coolant is in direct contact with the pellets. However, hydrogen addition to the coolant can affect Np solubility, making this indicator not as reliable as others. Sr-92 activity also indicates coolant contact with fuel, but the solubility of this nuclide is not affected by hydrogen addition. 

FRI: Fuel Reliability Indicator
The FRI is a uniform measure of fuel performance. The FRI measures the steady state off-gas activity rate at the SJAE outlet for the sum of 6 normalized to a common LHGR and corrected for 1) decay back to time of origin, 2) tramp uranium, and 3) power level. A FRI > 300 Ci/sec indicates a fuel failure.

Failed Fuel Action Plan
Action levels are categorized into three levels. Level 1 signifies one or more failures, with little release. Level 2 signifies one or more failures with substantial release rates that could cause dose issues. Level 3, the most severe, places the unit operation in jeopardy. Operating at Level 3 for greater than 14 days requires Plant Operating Review Committee approval. 


Normal
Level 1
Level 2
Level 3
I-131 or Xe-133 spiking
< 2x steady state value
> 2x steady state value
---
---
Xe-138/Xe-133
 300
 100
---
---
Sum of 6 (Ci/s)
 1000
> 1000
 10,000
 50,000
FRI
 300
> 300
---
---
DEI
---
---
> 0.001
> 20% TS
Noble gases (Ci/s/MWth)
---
---
---
> 20% TS

PST: Power Suppression Testing
Control rods are inserted and withdrawn to determine the locations of failed fuel by detecting fission gas release. PST is performed at about 50-65% power to be within the safe fuel conditioning (PCIOMR) envelope, but high enough to ensure an offgas response. After a control rod is inserted, power locally decreases, allowing water to seep into the rod and flash to steam, releasing fission products (if there is a leak). After each motion, 10 minutes are waited for a chemistry offgas sample to be taken, and then the process is repeated until the leaker is discovered. 

Fuel Sipping
Assemblies are pulled vertically upwards, producing a pressure differential that allows fission gasses to bubble out of the fuel rods. The sipping device attaches to the top of the channel, forming an isolated water column. 

Ultrasonic Cleaning
Burned fuel can be ultrasonically cleaned during an outage to remove the crud layer that has built up. 

Fuel Reconstitutions
Stainless steel dummy rods may replace damaged fuel rods to avoid changes in coolant flow for damaged assemblies that are to be reinserted. 
\section{Heat Balance}

The reactor heat balance provides a means by which to calculate the thermal power by hand in case that other methods are not available and as a comparison against APRM readings, 3D Monicore, the FW temperature, turbine first stage pressure, and potentially IRM readings. This procedure also allows you to identify a single break in the calculation chain that 3D Monicore might mask. This procedure is especially important below 25\% power because the 3D Monicore FW flow readings may cut out and be inaccurate due to the nature of these sensors, but because violation of thermal limits below 25\% power is essentially impossible, this is not a cause for concern. If the heat balance differs substantially (greater than 2\%) from the APRM readings, calibrate the APRMs. This manual heat balance is performed once a month. 







The pressure term in the work expression reflects that unless there is mass transfer, pressure acting on a CV only represents a force, not work. We assume the reactor is in steady state and that no work is done by the reactor, but only in the turbine and pump. Potential and kinetic energy effects are negligible, so that the above equation becomes, around a loop such that 


This equation is applied for each flow path that passes through the reactor core, and then the CTP is taken as . 







Because there is greater difficulty in measuring gas rather than liquid flowrates, we replace the steam flowrate with the sum of the FW and CRD flowrates. 



We have also assumed 100% quality in the steam. The CRD inlet temperature is assumed to be 80 F. This equation is equivalent to tracking three different flows of water through the reactor ? the FW, CRD, and RWCU paths ? and two losses that are independent of mass flowrates. The radiative losses equal 4.1 MW (fixed) due to heat loss from the RPV and other materials. The RR pumps add energy to the coolant in a manner that is proportional to their motor current multiplied by their voltage (P=IV). Electrical energy is converted to mechanical energy with 94% efficiency. 



The FW flowrate is determined by a Venturi meter, so a density correction factor is applied to convert volumetric to mass flowrates. This CTP indication is then compared against several other values. 

1.	TBV positions ? Once rated pressure is reached during startup, steam is allowed to exit the reactor, where it is diverted past the turbine through the TBVs and dumped directly to the condenser. The TBV position correlates to a % CTP for a certain FW temperature. Each TBV is rated to about 5% CTP steam flow when fully open. During startup, the turbine is synced once 1.5 TBV are full open (about 11% RTP). Then, the flow is switched from the TBVs to the TCVs.
2.	Generator MWe ? but, efficiency may not be known, and decreases at lower powers
3.	Turbine 1st stage pressure ? scales linearly with CTP
4.	Final FW temperature ? scales quadratically with CTP
5.	SRMs, IRMs, LPRMs

Enthalpy increases with temperature, so higher than normal temperatures may cause different CTPs. If pressure increases, then CTP will actually decrease, since steam enthalpy decreases slightly with pressure. If the heat balance in 3D Monicore is incorrect, then the APRMs may be miscalibrated (since they?re calibrated to the 3D Monicore heat balance), and the APRM scram and rod block points may not be set correctly. If the APRMs are miscalibrated in the conservative direction, this will lead to an overestimation of fuel depletion, which could cause SDM issues in future cycles. 



HVAC Systems

HVAC systems circulate air within secondary containment. Air is drawn in from outside, and is expelled by exhaust fans to a stack. The containment is kept at slightly lower pressure than the outside to draw in contamination in case of a leak. Exhaust fans expel more air than is drawn in. In the case of an accident, the fans will trip, and not allow air to flow in or out of the containment. 

Standby Gas Treatment
This system restores containment negative pressure in case the HVAC system isolates due to high radiation levels. 
Hydrogen Generation

The only possible source of hydrogen on a large scale is from a severely degraded core. At TMI, about 9 hours into the accident, a pressure spike of 28 psig was observed in the primary coolant system, but dismissed ? this was later determined to be hydrogen gas ignition. Containment spray was initiated to contain the burn and minimize damage. No simulations since TMI have correctly predicted the amount of hydrogen gas actually generated. In 2005, april abandoned the capability of the hydrogen recombiners to remove hydrogen from the containment atmosphere, but kept their capability of mixing the containment atmosphere to reduce local pockets of high hydrogen to oxygen concentrations. This was allowed following analysis that showed no significant hydrogen generation during core uncovering, and that the significant generation would occur in the radiolytic decomposition of water, and not the zirconium-hydrogen interaction. (And the radiolytic production is small). This long time delay for radiolytic products to accumulate would give operators enough time to prepare for the event. 10 CFR 50.44 removed the design-basis LOCA hydrogen release, and eliminated requirements for a hydrogen control system, and so several years later, the recombiners were removed from the Tech Specs. This code still requires a mixed atmosphere. 

The 10CFR requirement regarding hydrogen production is that the maximum hydrogen generation shall not exceed 1% of the hypothetical amount that would be generated if all metal in the cladding tubes adjacent to fuel pellets (i.e. neglecting the plenum cladding material) were to instantaneously react. Hydrogen gas behaves differently in several environments common to reactor systems. In air, hydrogen combusts by:
2H2 + O2  2H2O
Combustion of hydrogen in air can occur as a burn (deflagration) or explosion (detonation). A deflagration is a slow burn where the flame front moves between 2 and 50 ft/sec. A deflagration in the primary containment would require an ignition source to occur. In a detonation, shock wave compressions heat the gas temperature above the auto-ignition temperature of 910 F. Hydrogen is only flammable in air above a certain volume %. Deflagrations can occur between 4-18% and 58-76%. A mixture with less than 4% by volume hydrogen gas will be too lean to burn, while above 76%, the mixture does not have enough oxygen to sustain burning. When the volume percent is less than 12%, a complete burn will not occur ? only a small percent will burn. Detonations will occur in the remaining window of 18-58%. 

In other materials, flammability limits change due to chemical interactions and heat quenching of available energy to be transmitted to unburned regions of the gas. The high heat capacity of water will result in a steam introduction reducing the likelihood of combustion. Only the hydrogen produced in the core region is significant in accident conditions ? any hydrogen produced from radiolysis from transported fission products in the suppression pool is minimal in strength, and due to ECCS activation, strong currents are initiated in the suppression pool, promoting good mixing and dispersion of any hydrogen produced. In the containment, likewise, strong currents provide good mixing, except in the drywell head area and in the CRD undervessel area. The hydrogen recombiners (mixers) are located such that they can circulate gases within these two regions to prevent combustible mixtures exceeding 4% hydrogen. 

Drywells are inerted with nitrogen to limit the oxygen concentration so that hydrogen combustion is limited. The drywell and containment can also be purged to release hydrogen before it reaches explosive levels. 


Zirconium-Steam Reaction


Zirconium oxidation was the main source of hydrogen gas in TMI. The fuel cladding and structural materials in a BWR are both composed of Zircaloy-2, while channels are composed of Zircaloy-4; all of these materials would need to be replaced to eliminate the major risk of hydrogen generation. The actual reaction above takes place in several stages, the first being the decomposition of water into hydrogen and oxygen. This decomposition is an endothermic reaction, and requires a heat source (or radiation interaction) to occur. 

Temperature (F)
Time to produce 100 kg hydrogen (s)
980
6.5e7 (2 years)
1700
1.5e4 (4 hours)
2420
216
3140
17

The zirconium-steam reaction results in zirconium oxidation (which embrittles the cladding), hydrogen gas, and extra heat generation (which places stress on the ECCS systems). ZrO2 to some extent helps mitigate hydrogen production, since it shields the fresh Zr from hydrogen, but the embrittlement of the cladding, especially in accident scenarios when cold ECCS water injection leads to thermal shocks, can cause rapid fuel failures. The hydrogen produced will then accumulate in the drywell, wetwell, or both. Hydrogen within the primary coolant system can cause corrosive conditions elsewhere in the loop (increase pH). Hydrogen in the core can also blanket the fuel and itself protect against additional hydrogen production. Because heat is needed to initiate the reaction, but heat is released in the reaction, given a high enough energy level in the core, the reaction can become self-sustaining. 

Steel-Steam Reaction
Given high-enough temperatures, usually in excess of 2200 F, oxidation of steel components can also lead to hydrogen production. 280 BTUs are produced per pound of steel reacted. Steel oxidation is not the primary cause of hydrogen generation in accident conditions, since the steel isn?t directly in contact with the heat source, it would be difficult for these components to reach 2200 F or higher. However, if the steel reaches its melting point of about 2500-2750 F, then steel oxidation could lead to more hydrogen generation than zirconium. 

Containment Materials
Zinc, a common material in paints and galvanized materials, can oxidize and release hydrogen through a zinc-steam reaction. This source is less likely to contribute significant quantities than the zirconium-steam or steel-steam reactions because most of the zinc is located outside the RPV. Aluminum oxidizes to produce Al2O3 and H2 in corrosion reactions, and hence could also lead to hydrogen production. 

Corium-Concrete Interactions
Corium, or the molten fuel, cladding, control blades, and structural materials, is produced during a core melt accident. This corium will melt through the RPV and fall onto the concrete undervessel floor, where it will interact with the concrete to produce significant amounts of hydrogen gas. The heat from the corium will lead to thermal decomposition of the concrete into gaseous products. These gases pass upwards due to buoyancy forces through the melt, where they undergo chemical interactions to produce hydrogen gas. Because concrete contains a lot of water, this water is released due to the corium heat, which then interacts with the corium to form hydrogen and CO. UO2, Cr2O3, and Mo all interact with water to form hydrogen. 

UO2 + H2O  UO3 + H2
Cr2O3 + 3H2O  2CrO3 + 3H2
Mo + 3H2O  MoO3 +3H2

Radiolytic Decomposition of Water
2H2O + radiation  2H2 + O2

Other products of radiolysis include OH-, H+, HO2, and H2O2. In PWRs, equilibrium is reached between the radiolysis of water molecules and its recombination, since the products will react with each other to re-form regular water molecules. However, in BWRs, the offgas system continually removes radiolysis products, so no such equilibrium is reached. For every 104 eV of radiation energy, about 22 H2O molecules can be decomposed. This radiolysis is due to the gamma rays that directly deposit their energy in the coolant. Decay energies are the driving force for radiolysis, so radiolysis rates are typically proportional to reactor power level. Radiolysis is the only significant production pathway for oxygen inside the containment. This method is relatively slow at producing hydrogen when compared to the Zircaloy-steam reaction. Radiolytic production is only sufficient to initiate flammable conditions after 30 days. 




Intermediate Range Monitors (IRMs) [BWR]

IRMs are ionization chamber fission chamber detectors that are used in startup, power ascension, and heat-up. Eight detectors are located within the core, and are inserted/retracted as needed to prevent detector depletion. Because there is a much higher neutron population in the IRM range, these detectors don?t need to be as sensitive as the SRMs, and are operated at lower voltages and with argon pressures 10x less than those in the SRMs. In addition, the IRMs only have about 25% the enrichment as the SRMs, because otherwise the detector would heat up so much from the increased fissions that damage could occur. The IRMs can read up to 40% power, but are typically only used up to 10% power. 
Local Power Range Monitors (LPRMs)

LPRMs are fission chamber detectors operated in the ionization regime located between every four assemblies in BWR cores that provide local flux readings to the:
1)	APRMs for core-average power determination
2)	LPRM Group A or B for additional local flux monitoring
3)	OPRMs for local power oscillation detection
4)	RBM for local rod blocks
5)	3D Monicore for thermal limit calculation (at least 75% needed)

Each LPRM contains about 0.00027 g U-235. Four LPRM detectors are located on one common string except in BWR-6?s, which can replace only a single LPRM at a time. All detectors used in the industry are the regenerative sensors (GE NA-300), which include additional U-234 to breed U-235 over time. This allows the neutron sensitivity to remain approximately constant or even increase until about 3.0 snvts (1 snvt = 1021 nvt) exposure, which improves the life of the detector. Without regenerative sensors, the sensitivity would steadily decrease upon first insertion into the core. 


Calibration Factor (C Lattice)
Electronic EOL (A)
GE NA-200
0.137
50.0
GE NA-250
0.164
50.0
GE NA-300
0.092
50.0
Reuter Stokes 200
0.120
50.0
Reuter Stokes 212
0.120
50.0
KWU MBK-61
---
40.2
IST WLK-24261
0.199
65.0

Detection Principles
The thin layer of enriched uranium interacts with neutrons that enter the detector, causing fission to occur. Thus, LPRMs are primarily thermal neutron detectors. A fission will occur in the lining, and produce two or more fission products. These fission products move into the gas region. Because they are highly charged, they ionize the argon gas, producing many ionization electrons. These electrons are collected at the anode, where they produce a signal. The signal received at the anode is produced due to fission product ionization of the fill gas and electrons produced from Compton scattering of gamma rays within the fill gas. LPRMs are operated in current mode, rather than pulse mode as the SRMs, so that a lower level discriminator cannot be used, and so the signal consists of neutron and gamma interactions. If an LLD were used, then there would be a very high dead time due to the high neutron flux. Once criticality has been reached, the neutron population is theoretically stable, and hence the readings become proportional to the neutron population, even with a significant portion of the reading arising from gamma interactions. Below criticality, the gamma background must be filtered out by pulse mode so that small changes in neutron population can be detected to monitor the approach to criticality. 

Plant Arrangement
LPRMs are contained in dry tubes that are welded to the bottom of the vessel head. These tubes have small holes drilled in them to allow cooling of the detectors by bypass flow. Four detectors are welded together onto a single string or are placed individually (BWR-6 only) within these tubes. Each dry tube contains four LPRMs, four signal cables, and one TIP tube. LPRMs are labeled as A, B, C, or D, depending on their elevation in the core ? the A detectors are 18? from the fuel bottom, the D detectors 18? from the fuel top, with all detectors equally spaced at 36? from each other. The 172 LPRMs (in 43 dry tubes) in april?s core are assigned to one of six APRMs or to either LPRM Group A or B. One LPRM in each dry tube is assigned to either LPRM Group A or B. 

The detectors convert neutron flux to percent power by the (approximate) equation:



Calibration
LPRMs are calibrated to read 80 when the surrounding fuel?s APLHGR is 12.2 kW/ft. Because the APLHGRlim changes with exposure, however, there is no direct correlation between LPRM reading and thermal limits. During operation, a detector is PANACEA rejected if 3D Monicore notices a large difference between the LPRM reading and what 3D Monicore would expect. Because it is more likely that the detector be inaccurate than the core simulator code, the detector reading is not included in the thermal limit calculations. These detectors are calibrated every 2000 EFPH by changing the calibration current supplied to the detectors so that their readings will match TIP readings that are collected during the calibration process. This accounts for typical sensitivity losses with U-235 depletion. LPRMs must have GAFs within the range of 0.9-1.1, so detectors with GAFs outside the range of 0.95-1.05 are selected for calibration to leave sufficient margin. 

IV Plots
Sweep through a range of voltages in the LPRMs to help redeposit any lumps/hills of U3O8 coating. The uranium might redeposit over time due to flux-induced effects, leading to tendrils of uranium that can lead to spiking LPRM power readings. Removing these tendrils reduces the spiking effects. Spiking is more commonly observed following an outage, since new LPRMs might have unstable uranium coatings. 

Detector Failure
Once the U-234 has been bred into U-235, and there is no longer net production of U-235 from U-234 neutron absorption, sensitivity drops exponentially. LPRMs decrease in sensitivity over time due to:
1)	Chamber gas release	
2)	U-235 depletion, which is measured in terms of both the decrease in the initial calibration current and the signal/noise ratio. The gamma sensitivity of the detector remains constant over time, since the sensitivity is really only a function of the chamber construction. As U-235 depletes, the ratio of the gamma signal to the neutron signal increases, which affects the transient performance of the detector. During a transient, the gamma signal will lag the neutron signal, since 40% of the gammas produced are from the decay of fission products. Because LPRMs provide trip signals in transients, they must be fast responders in all conditions, and so the signal/noise ratio must be less than 5. Once this ratio exceeds 5, more than 17% of the signal may be due to gamma interactions. The second measure of U-235 depletion arises from the decrease in the calibration current ? once the value decreases below about 115 A, the detector must be bypassed, since it?s current output is too low for reliability. The minimum calibration current is calculated by multiplying 0.092, or the detector-specific constant, by the initial calibration current, which is usually about 1250 A. The initial calibration current is the current initially supplied to the detector upon installation to give a full power signal. While the actual initial calibration current for regenerative detectors is about 850 A, extrapolation is made back to the y-axis with a linear fit, giving the 1250 A used. 
3)	Electronic effects - The calibration current decreases to the point where, even when the amplifier is set on full gain, the output current is too low for a proper reading to be established. Below about 50 A, the current level, and hence sensitivity, cannot be compensated for by an increase in system gain. 

A detector is bypassed once its calibration current decreases below the allowed limit. During each TIP set, data is taken for the detector exposure and calibration current, leading to plots on semi-log axes of calibration current vs. exposure. Sudden changes in the calibration current may indicate chamber gas release. A decrease in the gas pressure does not change the neutron/gamma ratio, but will affect overall detector sensitivity. Detectors are bypassed, i.e. routed to ground so that they give zero reading, before reaching their neutronic or electronic EOLs. 

Fit adaptive ? use 3D Monicore predictions
LPRM adaptive ? use LPRM readings
TIP adaptive ? use TIP readings

Main Generator

The main generator consists of copper windings aligned on a rotating shaft that rotates at the same speed as the turbine. Surrounding this copper rotor are stator windings, which are hollow copper tubes cooled by water. A magnetic field is moved across a conductor, causing electrons to move at right angles to the motion, producing an electric current. The direction reverses when the opposite pole faces the same direction. The main turbine speed is set at 1800 rpm so that the generator outputs current at a frequency of 60 Hz. The rated main generator output is 1212 MWe. 

Once leaving the generator, a transformer is used to step up the 22 kV generator output voltage to the 220 kV transmission voltage (more windings on the secondary side). This allows for less losses because the power can be transmitted at a lower current. 

H2 Cooling
The generator is filled with hydrogen for cooling during power operation, since the high-speed spinning rotor will produce heat that must be removed. H2 is better than air for heat transfer, while also producing less heat from friction. Pure hydrogen will not cause an explosion (rather, an opportune combination of hydrogen and oxygen will) when maintained at 90% or greater purity. A sealing system is located between the turbine and the generator to prevent hydrogen leakage, which could cause an explosion. When entering an outage, the generator is purged with CO2, and then air. When re-entering an outage, the generator is again purged with CO2, and then returned to hydrogen fill. This limits the likelihood of an accidental hydrogen explosion due to lingering oxygen. 

Stator Cooling
The stator is liquid-cooled to allow for maximum power output. The hollow copper stator tubes are cooled by water flowing through them. This water cannot be allowed to conduct electricity, so it must be kept very pure by demineralizers, since water will only begin to conduct electricity as it accumulates impurities. 
Main Steam System [BWR]

Dry steam leaves the reactor and flows out four MSLs located asymmetrically around the reactor vessel to allow for more room for the steam separator and dryer to fit. The steam passes through the four MSLs (which each contain SRVs, a flow restrictor, and two MSIVs) through the Main Steam Tunnel to an equalizing header before passing to the main turbine. This equalizing header branches into A) 5 TBVs or B) 4 TSVs followed by an intermediate header, followed by 4 TCVs. Four MSLs are used to:
1)	Allow TSV and MSIV testing without inducing large pressure transients. This testing is necessary because the closure times of these valves has a large impact on the thermal limit response. 
2)	Provide limited inventory loss in case of a LOCA
3)	Limit differential pressure on vessel internals in a LOCA
4)	Allow continued operation at reduced power with one MSL isolated

Rated steam flow is 15.435 Mlb/hr. This reading should ideally be equal to the sum of the FW flow and CRD flow, but due to the greater difficulty in measuring the flowrate of a compressible fluid, the indicated steam flow is typically around 15.1 Mlb/hr, while FW flow is around 15.3 Mlb/hr. 

SRV (Safety/Relief Valve)
SRVs are 13 valves located on the MSLs within the drywell that discharge steam directly to the suppression pool in case of emergency situations when HPCS and RCIC fail to provide sufficient cooling to the reactor. This depressurization allows the low-pressure systems to inject. The SRVs will also open to relieve pressure when the ECCS systems are not injecting ? in these cases, the high pressure will open a disk to automatically open the valve. The 13 SRVs open at different high pressures (not all will open at once, but rather more open the higher the reactor pressure). No more than one SRV may be inoperable. SRVs are located on the MSLs to reduce penetrations in the vessel and allow for easier refueling. 

SRVs are designed only for vapor conditions, and can become damaged to the point that they cannot function if water level in the reactor rises high enough and either decreases vessel exit quality substantially or even spills over into the MSLs. If reactor water level increases, the SRVs must discharge to the suppression pool to rapidly decrease reactor pressure so that some water flashes to steam, reducing level in the reactor to the point that the functionality of the SRVs is no longer challenged. The SRV functionality must be maintained if they are to meet their safety function. 

Flow Restrictor
Flow restrictors, located on the MSLs, constrict flow in case of a LOCA to reduce inventory loss before the MSIVs can close. 

MSIV (Main Steam Isolation Valve)
MSIVs are valves located on the MSLs located both inboard and outboard of the drywell, with both placed as close as possible to the drywell. These valves are used to isolate the reactor in 3-5 seconds in case there is a break to both prevent LOCA inventory loss and reduce radiation release. Because closure of just one of these valves will completely isolate a MSL, closing an MSIV leads to a large pressure transient in the reactor. Closure of all four MSIVs is the worst postulated pressure transient, since this isolates the reactor from the ultimate heat sink, meaning the reactor should scram. In 2014, april scrammed from closure of only one MSIV ? the pressure increase cause a scram, though the operators expected that the pressure transient would not be large enough to warrant a scram.

TBV (Turbine Bypass Valve)
The five TBVs can bypass steam directly from the MSLs to the condenser. The five valves have a flow capacity of 25% of the rated steam flow. During a startup, before syncing to the turbine, steam is sent through the TBVs straight to the condenser until about 11% power is reached, at which point steam is directed instead through the TSVs and TCVs to the turbine. TBV failure represents one of the most severe pressure transients in BWRs. 

TSV (Turbine Stop Valve)
A TSV closes off steam flow to the turbine in case of a scram. 

TCV (Turbine Control Valve)
The four TCVs are pressure regulators. Changing the position of these valves changes the pressure downstream of the valve, which then becomes the new primary system pressure. If pressure is high, these valves will open to reduce flow resistance. These valves are not actively controlled, but are automatically controlled by pressure sensors. These valves are typically between 55%-60% open. 

EHC: Electro-Hydraulic Control
The EHC system provides the high-pressure fluids necessary to control valves going to the turbine. A synthetic fire-retardant oil is used. 

Reactor Head Vent
A vent on the top of the reactor vessel continuously vents non-condensable gases that pool at the top of the reactor to the A MSL. 
Main Turbine

The turbine extracts mechanical energy from steam at a full rated speed of 1800 rpm. This speed is kept constant so that the generator always produces electricity with a 60 Hz frequency. Steam enters the high pressure turbine, then passes through one of two Moisture Separator-Reheaters (MSRs), and then passes through cross-around pipes to enter one of the low-pressure turbines. The low-pressure turbines are much larger than the high pressure turbines to allow greater expansion area for the steam that had already expanded through the high pressure turbine. Exhaust steam from the low-pressure turbines is sent though hoods into the main condenser, while steam is extracted at various locations for heating purposes. april?s rated thermal power is 3546 MWth, while Clinton?s is 3473 MWth.

Main Turbine
The main turbine is a 14-stage steam turbine, with seven stages in the high pressure turbine and the remaining seven stages shared in the low pressure turbines. Steam enters at the center of the high pressure turbine shaft and expands in both directions to minimize axial thrust forces. The stationary blades (diaphragms) direct steam to the rotating blades (buckets), which are connected to the shaft. Each set of diaphragms and buckets constitutes a stage. The flow then passes through one of two MSRs, and then into one of three low pressure turbines. The low pressure turbines are lined in parallel, but are all on the same shaft. 

About 90% of the plant?s electricity comes from the high pressure turbine. The turbine 1st stage pressure indication is taken from a pressure sensor located between the 1st stage bucket and the 2nd stage diaphragm. This pressure, of about 675 psig at full power, scales linearly with reactor power. Steam exits the high pressure turbine at 185 psig. 

MSR: Moisture Separator and Reheater
The MSRs dry and superheat the high pressure turbine exhaust before re-entry into the low pressure turbines to increase efficiency and the lifetime of the low pressure turbines. The steam is forced to make rapid directional changes, separating out any liquid droplets. Liquid droplets are sent to heater drains. Then, the steam is heated further by steam extracted from other turbine stages, further drying the steam before it is sent to the low pressure turbines. This reheated steam exits the MSRs at about 520 F and 185 psig. The principal reason to reheat between turbines is to reduce moisture damage to turbine blades. A secondary benefit is the increased average temperature of heat addition, which improves efficiency. 

Heater Drains
The energy extracted in the form of condensed water from the MSRs is sent to the heater drains, where this energy is used for FW heating (?). If the heater drain functionality is lost, then moisture will build up in the MSRs, which will cause water to build up in the turbines, which will damage the blades. 

Turbine Trip
A turbine trip causes the TSVs to close and the TBVs to open, which dump steam to the condenser. However, if the TBVs fail to open, there will be a large pressure transient. 

Nuclear Fuel [BWR]

UO2 or UO2 + Gd2O3 (4-8% by weight Gd2O3) powder is pressed, sintered to ceramic form, and ground into cylindrical pellets with chamfered edges. Ceramic fuel, as opposed to metal fuel, is used due to the much higher melting point (2800 C vs. 1100 C) and stability in water-steam environments, even though metal fuels have significantly higher thermal conductivities (better response in thermal transients). The pellets are loaded into rods by the use of shaker tables that slowly vibrate the pellets into the cladding. A gap between the pellet and cladding is included to allow for differential thermal expansion. Before the rod is sealed, it is backfilled with helium to about 6-10 atm, which has good heat transfer and neutronic characteristics (low ). 

Eight fuel rods in each bundle are tie rods, or rods with threaded end plugs that screw into the upper and lower tie plates to hold the assembly together. Tie rods are typically located on the bundle periphery. Standard fuel rods slip fit into the upper tie plate via end plugs. While the active fuel length is about 12 feet, a fuel rod is about 13 feet long due to the 10? plenum (with a spring to stop movement) at the top of the rod to allow for fission gas collection to prevent rod overpressurization. Each fuel rod has cladding composed of 0.66 mm Zircaloy (Zircaloy-2 for BWRs, Zircaloy-4 for PWRs) due to its corrosion resistance and low  0.18 b (for pure Zr). Expansion springs are located at the tops of fuel rods to allow for axial thermal expansion of the cladding due to heat and irradiation effects. 

Seven to eight spacer grids are located about every 1.5 feet, though they may be unevenly spaced to improve thermal limits. Spacers introduce turbulence that improves heat transfer. The upper tie plate contains the fuel bundle handle and carries the weight of the assembly during fuel moves. The channel box slides over the assembly and rests on posts on the upper tie plate, where it is screwed in by a spring clip fastener and screw. 

The lower tie plate supports the weight while sitting. A fuel bundle weighs about 600 lb, while the channel only weight 80 lb. The nose piece sits in a fuel support piece. Two holes are located on the sides of the nose piece that are not adjacent to a control blade. These holes allow bypass flow into the region outside the channel boxes. About 10% of the core flow is bypass flow. Between the nose piece and the remainder of the assembly is a debris filter. The remaining 90% of the flow passes through the debris filter and up through the inside of each channel. 

A fuel cell consists of four fuel bundles, a control rod, and the removable fuel support piece (circular structure with four circular openings to accept the nose pieces of fuel bundles) that supports them. Spacer tabs separate the four assemblies in a control cell, leaving room for control blade entry. The 24 bundles not in fuel cells are supported by peripheral support pieces that are permanently attached to the lower core plate. Orifices in the fuel support piece direct all flow upwards (after it makes a 90 degree turn in the lower plenum region), which then diverges into bypass flow (after making another turn) through the sideways hole or fuel flow through the debris filter. 

april?s fuel reaches a discharge burnup of about 30-35 GWd/sT. About 23 MWD/sT is accumulated each day at full power. Annual cycles were implemented at Clinton because the ability to order lower-enriched fuel allowed fuel to be preferentially loaded towards the center of the core, allowing large fuel savings.

Design
Lower-enriched pins are placed on bundle edges that will be adjacent to control rods. When the control rod is inserted, while the blade has a very high cross section for thermal neutrons, there is relatively little impact on fast neutrons, which produces a hard neutron spectrum near control rods. Fast neutrons preferentially lead to the build up of Pu-239 due to the epithermal resonance absorption of U-238, which, if the fuel were highly enriched and then uncovered from the blade, could lead to undesirable power peaking. The top and bottom 6-8? contains natural uranium blankets. 

BWRs us Gd2O3 (natural Gd) neutron poison in order to reach the target burnup of 40,000-50,000 MWd/T. Gadolinium decreases the thermal conductivity of the fuel, increasing the centerline temperature. So, the avoid hot spots, the enrichment in Gd rods is lower than in un-poisoned rods. More Gd is typically loaded in the bottom halves of assemblies so that the use of shallow rods is less necessary and there is a more uniform axial flux shape. While Gd isotopes absorb neutrons via capture reactions, boron absorbs via (n,) reactions. Higher order lattices (n x n) lower the LHGR. Over a core life, the cycle begins at an intermediate  (not the maximum in the cycle), and decreases linearly within the first 25 days as fisson product poisons build in to equilibrium levels. Then, as the burnable poisons begin to deplete,  again rises in a quadratic fashion until about 1/3 to 1/2 of the fuel cycle, or when the burnable poisons have become depleted. This depletion then allows the fuel to be burned, causing  to decrease quadratically until EOL.  

BWR bundles have axially-varying water rods to improve moderation (allowing enrichment reduction), since water in these rods typically does not boil. These water rods have physical walls that do not allow horizontal mixing. Assemblies with asymmetric water boxes are designed in that way simply because they cannot be made symmetric with 3x3 water boxes in an 8x8 rod design. 

As long as the fuel isn?t overheated, if there is a cladding breach, then over 90% of the fission products are maintained within the fuel. This allowed fuel design to use the same axial rod height, but a reduced plenum spacing, allowing active fuel lengths to increase from 144 to 150 inches. Upon building an assembly, the vendor will use a detector to confirm its enrichment. 

Part length rods (1/3 and 2/3 height of full fuel rods) are used because:
1)	Less fuel is included in the top of the core, where it would be underutilized
2)	Improvement to thermal limit margins in the top of the core
3)	Reduce the 2-phase pressure drop at the top of the core, which leads to better core and channel stability and allows for an increase in the cladding diameter to maximize the fuel weight for an overall P. For a limited P, this pressure drop can be shifted to the debris filter and core plate, which leads to better stability and allowable debris filter designs. The greater portion of the allowable assembly axial pressure drop that can be shifted to the core plate, the smaller the holes in the debris filter. There is about an 18 psi pressure drop over the lower core plate. 
4)	Improved moderator/fuel ratio to improve SDM and efficiency

Areva Fuel
ATRIUM-10 is the now-standard Areva fuel, first manufactured in 1992. This fuel contains an asymmetric water channel (3x3), and includes eight part-length rods. CASMO is AREVA?s lattice code, while MICROBURN is their core simulator. 

GNF Fuel
GNF-2 fuel consists of 10x10 lattices, with 8 partial length rods, 6 short rods, and two 2x2 square water channels. GNF-2 fuel was introduced in 2008. 

Westinghouse Fuel
Westinghouse produces BWR fuel (SVEA, Optima), which consist of four sub-bundles and a water cross within the channel box. The unique water channel offers better support and reduces channel distortion and bowing. The four assemblies sit in a basket-like structure. This design is particularly difficult to rechannel. 

Westinghouse is the only nuclear fuel company to make PWR, BWR, AGR, and VVER fuel. Phoenix is Westinghouse?s core simulator code, while ANC is used to create the cross sections. 

Fuel Handling/Inspections
New fuel arrives on site un-channeled to allow for visual inspections. The bundles are then channeled on site by sliding the channel over the bundle and fastening it with a screw in one corner of the bundle. 

Core Monitoring Codes
The three principal BWR fuel vendors supply a core monitoring code that is used to make predictions, perform heat balances, and print out plant reports. Whenever a plant switches fuel vendors, they must also switch core monitoring codes. GE uses 3D Monicore/Acumen, Areva uses Powerplex, and Westinghouse uses the Westinghouse Core Monitoring System (WCMS). When a new core monitoring system is adopted, it is run in parallel with the more familiar system before full implementation. 
Nuclear Fuel [PWR]

The most common PWR fuel vendor is Westinghouse, whose RFA fuel design consists of 17x17 fuel pins with 24 guide tubes and a central instrumentation tube for in-core detectors. The diameter of the guide tubes decreases as you move further down the assembly in order to act as a dashpot to reduce the rod drop velocity (similar to the velocity delimiter in BWR control blades). The diameter of the instrumentation tube, however, is constant. 

Several types of burnable poisons are used in PWR fuel. Initially, PWRs didn?t use burnable poisons, and the cycle lengths were much shorter. Wet Annular Burnable Absorbers (WABAs) exist in the form of spider assemblies that are inserted into the guide tube locations (some or all) from above, as long as the assembly is not under an RCCA location. Each rod is annular, with a zircaloy cladding surrounding Al2O3-B4C poison, with an air gap between the poison and each cladding surface to allow for swelling and collection of helium. Coolant flows through the center of each WABA pin, and coolant also passes between the WABA and the guide tube walls. All WABAs are removed after one cycle to help reduce neutronic penalties due to both residual poison and reduced water flow through an assembly. Then, the spider assemblies are chopped up for disposal. Fresh WABAs can be reinserted into once-burned assemblies. 

Integral Fuel Burnable Absorbers (IFBAs) are ZrB2 coatings on the outside of pellets that burns out faster than WABAs. This complete depletion leads to no residual penalty at EOL. However, because this coating is directly on the fuel pellets themselves, the  reaction leads to additional gas pressure within these rods. 

Gd2O3 is used in some PWRs, but not the 4-loop Westinghouse designs. Er2O3 is an integral fuel burnable absorber (mixed into the fuel) used in CE PWRs, though most plants have converted to ZrB2. Borosilicate glass was one of the first poisons developed by Westinghouse, and consists of glass clad in stainless steel, with a void center. This poison was first tested at Ginna. 

Nuclear Instrumentation [PWR]

The neutron monitoring system consists of both in-core and ex-core detectors, though the in-core detectors do not reside permanently in the vessel as the LPRMs in BWRs, but rather perform similar to TIPs. Excore detectors detect neutron leakage from the core, which is proportional to power. Excore detectors are used because they are easier to operate and maintain than in-core detectors, and detailed flux profiles are not needed for PWRs. PWRs contain 2 SRMs, 2 IRMs, and 8 power range monitors. The power range monitors are located on the ?angled? corners of the ?cylindrical? core, with one above another in each of the four locations. On two of the flat faces, are located one SRM and one IRM, with the SRM beneath the IRM. The other two flat faces contain spare locations that do not have detectors. All of these excore detectors are located outside the RPV. 

Source Range Monitors
The SRMS are BF3 counters operated in pulse mode so that gamma pulses can be discriminated with an LLD, similar to the BWR SRMs. The boron is enriched in B-10 to about 96%, and neutrons are detected through the  reaction, since the alpha particle and Li-7 then ionize the gas, leading to detectable charge. The pulse size depends on whether the reaction produced the ground or excited state of Li-7. And, due to the wall effect, either the Li-7 or alpha particle will cause ionization. 

Intermediate Range Monitors
The IRMs are compensated B-10 ion chambers. HPDE is used as a moderator and insulator. Each B-10 ion chamber really consists of two detectors ? one that is coated in enriched B-10, and another that is uncoated, allowing the gamma signal to be subtracted from the neutron signal. Signal processing subtracts out the gamma signal. These are located closer to the core midplane than the SRMs. 

Power Range Detectors
The APRMs are also B-10 ion chambers (uncompensated), and each if 6 feet long. These detectors are uncompensated because at full power, the gamma release is proportional to the power. Ideally, the top detector would indicate the power in the top half of the core, but because neutrons don?t leak only perpendicular to the fuel, this is not the case. 
Offgas System

The steam jet air ejectors (SJAEs) maintain vacuum pressures in the condenser to:
1)	Improve plant efficiency ? The higher the P over the turbine, the higher the plant efficiency. Because there is a greater pressure drop, the steam will expand more through the last turbine, allowing more work to be extracted. 



When the lake temperature outside increases, the condenser is less effective at removing heat from the coolant, which leads to a smaller P over the turbine, which reduces plant efficiency. 

2)	Prevent moisture droplet damage to the turbine blades ? Due to the vacuum in the condenser, steam is pulled into the condenser before it is allowed to condense on the turbine blades. 

3)	Remove non-condensable gases ? Air leaks into the condenser, while fission product gases, H2 and O2 produced from radiation-induced hydrolysis of water (largest contribution to non-condensable gases), and N-16 formed from activation of O and produced in the primary coolant. If these non-condensable gases were not removed, then they would blanket the condenser tubes, impeding heat transfer. 

The SJAEs are similar in construction to jet pumps ? the very low pressure, high velocity flow through the SJAEs sucks in any non-condensable gases, which are then processed. Main steam is sent through the SJAEs to provide the drive flow. In a BWR, these gases must be processed through the offgas system before being vented, but because radiation levels are much lower in PWRs, the offgas can be directly vented to the environment. 

The SJAE exhaust is then sent to the offgas system, which has the following flow steps:
1)	A condenser removes some steam that has become entrained with non-condensable gases. 
2)	A platinum recombiner recombines O with H to produce H2O. This water is then passed through a condenser to cool it down, and then this water is returned to the plant. 
3)	The remaining gases are passed through a quarter-mile long pipe (holdup volume) to allow for the gases to decay before release. 
4)	Reheated for humidity control.
5)	An adsorption bed containing charcoal selectively absorbs and delays Xe and Kr from the mostly-air offgas, permitting them longer time to decay. 
6)	Discharge to main stack.

Pressurizer

The pressurizer controls pressure in PWRs in the range of 1700-2500 psig. There is one pressurizer per reactor, which is connected to one of the loops at the bottom of the pressurizer. The water in the pressurizer is essentially stagnant, as coolant does not flow through the pressurizer, but rather reaches an equilibrium height based on system pressure. Due to the near incompressibility of water, the pressurizer is able to maintain constant pressure in the entire primary coolant loop by either heating or condensing the coolant. The pressurizer is located high relative to the core so that if a LOCA is detected based on instrumentation in the pressurizer, it is likely that the core is still flooded.

The pressurizer should not be allowed to ?go solid,? or fill entirely with water. The function of the pressurizer relies on both liquid and vapor being present in the pressurizer, and if completely filled with water, then if there is a sudden pressure change in the primary system, water hammer can occur. 

With a pressure increase or decrease, it is not necessary to always either turn on the electrical heaters or the cooling spray ? for small pressure changes, the pressure can be maintained constant by allowing heat to be transferred between the liquid and vapor in the pressurizer. For a small pressure decrease, more liquid will flash to steam in the pressurizer, naturally producing a compressive force. For a small pressure increase, the vapor space is compressed, causing the vapor to become superheated. This superheat is transferred back to the primary coolant and pressurizer walls, thus condensing itself. However, some pressurizers always have the heater on to make up for fixed losses such as radiative losses. In this case, to decrease pressure, the heaters may simply turned off for some period of time. 

Increase Pressure
To increase pressure, the water in the pressurizer is electrically heated by coils at the bottom of the pressurizer. Some coolant will then expand upon heating, thus exerting a pressurizing force on the system. The heaters engage once the water level in the pressurizer reaches a certain height. The extent of the heating is not controlled by heater intensity, but rather by how long the heater is on. 

Because the pressure in the primary system is held constant by the pressurizer, the water level in the pressurizer then depends only on the temperature of the RCS.  

Decrease Pressure
To decrease pressure, cold water from the RCS cold leg is sprayed through a nozzle on the top of the pressurizer, condensing water and reducing the pressure due to the vapor in the pressurizer top. If system pressure is very high, the PORVs on top of the pressurizer open to release steam to tanks where it can be condensed. Once these tanks become full, rupture disks will open, allowing primary coolant to spill onto the containment floor into sumps. 


Reactivity Anomaly (RA)

The reactivity anomaly surveillance compares actual and predicted hot critical reactivity parameters to ensure that the safety analyses performed prior to the fuel cycle still apply based on current plant conditions as indicated by the hot critical eigenvalue. It is possible for the reactor to have drifted in its operating conditions from the baseline values used in the safety analyses, so the applicability of the analyses must be periodically assessed. 

Because SDM compares reactivity parameters in the cold condition, the RA surveillance only needs to be performed every 1000 MWD/MT at full power, and 24 hours after reaching criticality following a vessel geometry change. This 24 hour period is included to allow xenon to reach steady state levels. If the RA is outside the allowable  1% k/k, then is must be restored within 72 hours, or else the reactor shut down. Extreme model fidelity is not required, since conservative design is expected to always be able to shut down the reactor. The RA is calculated by comparing hot critical eigenvalues from 3D Monicore and from the Cycle Management Report predictions made by PANACEA. An alternate method uses the control rod density in the core, compared with the predicted rod density. 

The predicted  is calculated by a core simulator code (PANACEA) as a function of burnup using the set initial operational conditions, while the measured  is determined by 3D Monicore using actual plant conditions. If the two values are reasonably similar at BOC, then is normalized to  to reduce the code-to-code differences and make operational differences clearer.
  1% k/k = 0.01 k/k, or 1000 pcm

  1% k/k = 0.01 k/k, or 1000 pcm

A large reactivity anomaly might indicate:

1)	The use of incorrect values in the predictions ? the plant has operated for a long time or has drifted from the baseline conditions used in the safety analyses
2)	Drifting plant sensors simply indicating different plant conditions. If flow rates or power were continually measured as incorrect, then the heat balance may have recorded over/under estimations of fuel depletion. 
3)	Control rods out of sequence or with unexpected worths
4)	Core geometry or composition error, such as incorrectly-loaded fuel, incorrect fuel received from the vendor, or channel bowing causing flow-redistribution-induced power shifts. The Hatch OPEX demonstrates that the inability to identify an incorrectly-placed control blade led to high RAs for a long period of time. 
5)	
Reactor Core Isolation Cooling (RCIC)

RCIC provides reactor cooling in case the reactor is isolated from the condenser/FW systems during high power operation (RHR is used for low power or shutdown conditions). When the condenser/FW systems are lost, steam flow to the TDRFPs is lost, and so RCIC is essentially a substitute for a TDRFP. 

The RCIC system consists of a steam-driven turbine that powers the RCIC pump, which takes suction from either the Cycled Condensate (CY) storage tank or the suppression pool. The preferred source of water is from the CY tank, since the water quality in the suppression pool is much poorer, since basically any water escaping a leak in the reactor pressure boundary may have touched the floor, etc. This water is then injected into the reactor, and the discharge from the turbine is sent to the suppression pool. The steam used to power the RCIC turbine is taken from the MSL before the SRVs (and hence before the MSIVs and the TDRFP suctions), so reactor pressure must be at least 66 psig to provide enough head to power the RCIC pump. The RCIC turbine is not very efficient, but is chosen due to its high reliability. 

During a normal reactor shutdown, steam is sent through the TBVs directly to the condenser, and the FW system returns this water. However, if the FW system is unavailable, RCIC supplies makeup water to the reactor. 

Not all BWRs have a RCIC system ? some have isolation condensers, where the coolant is sent through a heat exchanger, and is returned to the core. 


Reactor Operation

The reactor is operated in five modes:
1.	Power operations (> 5% power)
2.	Startup ( 5% power)
3.	Hot shutdown (T > 350 F)
4.	Cold shutdown (200 F < T < 350 F)
5.	Refueling

BWRs used to load follow, dropping several hundred MW at night and on weekends. 

ReMas are either emergent flow drop to decrease power, or shutdown. 
Reactor Pressure Vessel (RPV) [BWR]

The RPV comprises one of the important barriers to fission product release. The RPV is made of carbon steel with a ?? inner stainless steel lining for corrosion resistance. The RPV has an upper limit design temperature of 575 F and pressure of 1250 psig, with a nominal operating pressure of 1005 psig. The RPV houses the core, the steam separator, and the steam dryer. A stainless steel core shroud separates the core region from the downcomer region, allowing only a liquid-vapor mixture to enter the steam separator. This steam separator is attached to the upper core shroud so that all water exiting the core must pass through the steam separator. The core exit void fraction is only about 14%, but after passing through the steam separator, is about 90%, and upon exiting the steam dryer, is 99.9%. 

Natural circulation will occur as long as the water level in the RPV is above the steam separator drains. Otherwise, flow stratification will occur. 

Coupons are mounted in the RPV to measure radiation-induced transitions in the DBTT, since the RPV is the component most susceptible to brittle failure. These coupons hang at the core mid-plane and are removed once every 10 years to measure material degradation. 

Steam Separator
A steam separator is located at the top of each standpipe connected to the shroud head. The core exit liquid-vapor mixture passes through turning vanes, which by centrifugal forces, fling water droplets outwards, causing them to fall back down into the downcomer region. One of the reasons that coolant flows upwards is to allow the saturated liquid exiting the steam separators to fall back down to the annulus region. 

Steam Dryer
Steam is forced to make rapid directional changes, causing water to fling outwards and be collected back in the downcomer region. 
Reactor Protection System (RPS)

The RPS system initiates power and flow-biased scram signals based on input from the APRMs. The RPS consists of two independent trip systems (A and B), each with two logic channels (A and C APRMs or B and D APRMs). For a full scram to occur, one-out-of-two twice logic is required within the four APRMs. A half scram occurs if only one APRM trips. 
Reactor Recirculation (RR) System

The RR system is used to control the core flow rate, and can adjust power in the 65%-100% range simply by changing flow. Forced convective cooling provides for better heat transfer and allows for another method of reactivity control, since increasing RR flow rate sweeps voids out of the core, increasing moderation without disturbing control rod patterns. The RR system also provides forced circulation cooling in shutdown conditions, coupled with the RHR system taking suction from the RR A loop and passing the flow through a heat exchanger. 

The RR system consists of two independent recirculation loops that each require about 6.5 MWe power (about 1% of the produced power goes towards powering the RR pumps). Suction water is taken from the very bottom of the annulus region (below the core plate) simultaneously into both RR loops (A and B). This flow then passes through a flow elbow that houses the APRM and RBM flow converters and RR flow control instrumentation. The flow then passes through a two-speed motor-driven RR Pump, through a FCV which controls the flowrate, and then through a discharge valve to an equalizing header outside the RPV. The equalizing header connects to five inlets to the RPV, each of which directs flow from the RR loop to a Ram?s Head. 

The flow entering the RPV flows upwards very rapidly through the jet pump risers and then diverges into two different jet pumps. There are 10 jet pumps for each RR loop, and 20 total in the annulus region of the reactor. During this process, the high-pressure, low-velocity flow is accelerated so that its velocity increases (and pressure decreases) through the convergent nozzles of the jet pumps. The low-pressure, high-velocity flow then enters the mixing region, which has five mixing vanes open to the water in the annulus downcomer region. This mixing region is located at 2/3 active fuel height. Because the flow within the jet pump is at low pressure, water is sucked in from the annulus region and accelerated, encouraging mixing of the driving and driven flows. This mixed water then passes through a diffuser permanently attached to the baffle plate with a divergent cross-sectional area, causing the pressure to increase and the velocity to decrease. The coolant then flows out the jet pump outlet, located below the lower core plate, into the core inlet plenum and upwards through the core. All flow must pass through a jet pump prior to entering the core ? the baffle plate separates the annulus water region from the inlet plenum. The jet pumps thus provide the motive force to push water from the annulus region to the diffusers. 

The jet pump outlet forces a mixture of saturated liquid that has not proceeded through the steam line (~ 545 F downcomer) and incoming subcooled feedwater flow (~ 423 F) through the core, causing the actual core inlet temperature to be about 523 F (the inlet temperature of the RWCU system). Therefore the RR system also facilitates preheating of the FW before core entry. The rated drive flow is 33.85 Mlb/hr, which is the RR drive flow necessary to achieve 108.5 Mlb/hr core flow. 

Jet Pump Safety Significance
All 20 jet pumps must be operable for the LOCA analyses to apply, since their operation is an implicit assumption in the ability of the core to reflood to 2/3 height following a LOCA. If the jet pumps are damaged, their suction height may become lower, placing the plant in an unanalyzed condition. If a jet pump is found inoperable, the plant must be shut down. 

RR Loop Safety Significance
The RR system can either have both loops operating (dual-loop operation), or a single loop operating (single-loop operation). LOCA analysis has been performed for DLO with the break occurring in the higher-flow loop, and in SLO for the break occurring in the running loop. Safety can be maintained as long as the appropriate thermal limit set was in place before the incident. If a LOCA occurs in DLO, then the broken loop will nearly instantaneously stop providing flow, while the other running loop will coastdown until the jet pump suction is uncovered. 

When in DLO, if the loops have significantly different flows, reverse flow could occur, which can cause intense vibrations and damage to the jet pumps, and potentially cause mixer displacement. Reverse flow is assumed to occur when the active loop (when in SLO) is at or above 21,000 gpm. Even if both loops can operate, if there is a large flow mismatch, the lower-flow loop should be tripped to enter the now-safer SLO. The probability of reverse flow increases with flow mismatch between the loops. If both loops are OOS, then the reactor must be shut down. 

There are several additional safety mechanisms involving the RR pumps to add negative reactivity at EOC and in ATWS conditions. The ATWS-RPT signal will trip both RR pumps to low speed in case of an ATWS to add extra negative reactivity. 

M-ratio


The M-ratio is an indication of the degree of flow amplification through the jet pumps. Drive flow correlates to flow exiting the RR pumps. Driven flow correlate to the suction flow that mixes with the drive flow. Driven flow consists of the downcomer saturated liquid and feedwater flows. For the 54.25 Mlb/hr passing through set of 10 jet pumps, about 2/3 of this flow is suction flow, while 1/3 is drive flow.  

For a fixed RR speed, a higher M-ratio correlates to higher core flow. The higher the core flow resistance, the lower the M-ratio, since although the drive flow may be constant, there is greater impedance to suction flow. Core flow resistance increases the greater the bundle dP or 2-phase pressure drop. During coastdown, the 2-phase pressure drop decreases, and so there is naturally a small increase in core flow, adding positive reactivity when the reactor is being put in a lower power condition. When control rods are withdrawn, the power increase causes the 2-phase pressure drop to increase, leading to an increased core flow resistance, which then decreases the M-ratio and actually adds additional negative reactivity due to reduced core flow. However, when control rods are inserted, the power decrease decreases the 2-phase pressure drop, which decreases core resistance, which slightly increases core flow, leading to a small positive reactivity insertion. New equilibriums are reached due to 1) reactivity coefficient feedback and 2) core flow resistance feedback.

Bi-stable flow
Bi-stable flow naturally occurs in BWRs due to flow vortices in the RR loop and natural differences in jet pump flows due to the external header not distributing flow evenly between the five jet pump risers. Bi-stable flow causes flow swings of 0.1-1.5% with frequencies of 1-200 cycles/hour. In the low probability of bi-stable flow occurring simultaneously in both loops, reactor power can increase by up to 2%. 

april Jet Pump Plug Seal Loss
During the outage prior to L2C16, to perform maintenance on the RR lines, jet pump plugs were installed on all jet pumps for isolation. These plugs are installed in the five mixing vanes in the mixing region of the jet pumps. Once the seals are in place, the RR lines are vented and drained. An improper vacuum was placed, and three of these seals were lost into the reactor. The Unit started up and maintained power at about 30% for an extended period of time while analytical work was performed by ORNL, GE, and Westinghouse. The seals are large enough to block the flow orifices on the peripheral fuel bundles, and it had to be demonstrated that the seals had sufficiently degraded to the point where they would be sufficiently ductile to pass through the orifices and not cause flow-blockage-induced dryout. 

GE?s analysis showed that peripheral flow could keep peripheral bundles cooled if they were kept at a certain low power level. ORNL demonstrated radiation-assisted weakening of the seals to the point where they could pass through the orifices. Westinghouse used a newly-built experimental facility to analyze flow characteristics associated with flow blockage. As a result of the analysis, a MCPR penalty was enforced, and the rod pattern adjusted to depress peripheral bundle powers. With enough time, it was shown that the seals would degrade to safe conditions, and the reactor was brought up to full power after extended low-power operations for about 3 weeks.

Lowering Reactor Water Level
In case of an ATWS, lowering the reactor water level can reduce a threat to containment integrity, since lowering reactor water level promotes voiding. Lowering the water level reduces the natural circulation driving head, lowering the core flow rate. In addition, it provides increased carryunder from the dryers and separators, decreasing the core inlet subcooling. Uncovering the FW spargers preheats the FW with steam. However, lower water level commonly leads to level oscillations. 
Reactor Water Cleanup (RWCU) [BWR]

The RWCU system has functions similar to the chemistry program, in that it is designed to remove impurities in order to limit:
1)	Corrosion of plant components and fuel cladding
2)	Plating out of corrosion products on heat transfer surface areas
3)	Activated corrosion product flow throughout the plant

The RWCU system takes suction of about 0.11 Mlb/hr from the bottom of the RPV head and from the RR line to clean the reactor water. Suction is taken at the bottom head due to the high crud buildup in that location. The inlet to the RWCU system is about 530 F (a mixture of 423 F feedwater and 545 F saturated liquid), so this flow is first passed through a heat exchanger for cooling to less than 140 F to avoid damaging the filters/demineralizers, then through a cleanup process. The water is then returned to the reactor via the FW line after being reheated to about 450 F. The energy for reheat comes from the extracted RWCU line (exiting the vessel). 

Level Control
The RWCU system also provides level control in shutdown and startup conditions when the MDRFPs may not be running and some means is needed to account for natural water expansion/shrink during heatup/cooldown. The RWCU system can send steam directly to the condenser or Radwaste systems. The RWCU system is needed for this because the reactor is not generating steam through the normal path of MSLs to condenser, and hence the MDRFPs cannot be used to control level. 

In cold shutdown, the valves that control FW flowrate need a nonzero flowrate to function, so the RWCU passes reactor water directly to the condenser or Radwaste processing so that a net FW flow can be developed in shutdown conditions without causing a level increase in the reactor. During power operation, level is controlled by a sensor that sends signals to valves upstream of the TDRFPs. 

SBLC Injection
RWCU can be used as an alternate way to inject boric acid in case the SBLC line is inoperable. 

Residual Heat Removal System (RHR) [BWR]

The RHR system consists of three loops. The A and B loops both have heat exchangers, while the C loop does not. The RHR system has various safety and non-safety related functions. Some important functions include:
1)	Shutdown cooling ? RHR maintains the reactor in cold shutdown once the reactor has already been depressurized below 135 psig (~ 350 F) with all control rods inserted. When in shutdown conditions, RHR maintains the coolant temperature below 125 F indefinitely to remove decay heat and prevent flow stratification, since the MDRFPs cannot. The RHR A system takes suction from the A RR loop, while the RHR B system takes suction from the B RR loop. Only one of these is operated at a time. This flow is then pumped outside of containment through RHR pumps, through a heat exchanger, and is then returned to the vessel via the RR lines. If this normal return path is not available, the water can be returned through one of the ECCS pumps. If this cooling is sufficient, then no boiling will occur, and water exiting the core top will all fall down back into the downcomer region. 
2)	Suppression pool cooling ? The suppression pool water is passed through a heat exchanger cooled by lake water, and is then returned to the pool. This mode is utilized when the ECCS system are injecting (since their hot water is returned to the suppression pool via the LOCA break) or when the SRVs have actuated. 
3)	Suppression chamber/drywell spray ? In a LOCA blowdown, when steam blows from the reactor into the drywell and suppression pool, not all the steam may condense, so RHR can provide additional sprays of cool water. 
4)	LPCI ? Suction is taken from the suppression pool, is cooled through heat exchangers, and is then injected into the core. 

Shutdown [BWR]

1)	Drop flow to about 70 Mlb/hr.
2)	Insert control rods in the reverse order of the startup sequence. 

Coastdown
At the beginning of the coastdown period, the reactor is maxed on flow. For about 20 days, the power is then decreased at about 2% per week, and at the end of this period the reactor is shut down. In the past, FW heaters would be turned off near EOC to provide additional positive reactivity to continue operation for longer periods of time. 
3)	
Shutdown Margin (SDM)

SDM is the amount by which the core could be made subcritical with all control rods inserted except the rod with the highest worth assuming the core is xenon-free and at 68. SDM must be maintained  0.38% k/k (380 pcm) with the highest worth rod withdrawn, i.e. the core must be at or below a cold critical eigenvalue of 0.9962. If SDM is calculated experimentally, which is never done due to the intensive control rod movements required, then SDM need be  0.28% k/k. Having SDM within the appropriate limits ensures that the reactor can be shut down when at power and maintained shut down. SDM is calculated in the cold critical condition, since this is the most reactive state due to the absence of xenon (at least 72 hours in cold shutdown), high moderator and fuel temperatures, and a moderator void content.  

If SDM is outside acceptable limits, then it must be restored within 6 hours, or else the plant must be shut down (Mode 3) in 12 hours. If in hot shutdown you still aren?t at the appropriate SDM, insert all control rods immediately, and prepare to secure secondary containment and gas treatment within 1 hour. If in refueling, insert all control rods that are adjacent to at least 1 assembly, since control rods not adjacent to an assembly have little neutronic impact. SDM must be re-evaluated 1) 4 hours after reaching criticality following a geometric change, 2) within 72 hours of noting that a control rod is stuck partway out, and 3) after every geometry change in refueling. 

SDM changes with time due to fuel and control rod depletion. When measuring SDm at BOC, you must account for the fact that SDM might decrease over the cycle. To measure SDM during startup, pull to critical (when there is a reactor period of 200-300 seconds), which is actually a supercritical reactor. Compare this eigenvalue with that predicted by 3D Monicore. A period correction is applied since the SDm equation assumes the reactor has an infinite period. To experimentally measure SDM during refueling, the head does not need to be fully tensioned, and special procedures allow bypassing this requirement to pull control rods. For Unit 1, the minimum cycle SDM is 0.0177  (1770 pcm). 

Control rod interlocks prevent more than one control rod from being removed at a time, helping ensure shutdown margin is met in refueling conditions. Clinton did not analyze the SDM impact of putting a fuel assembly in the core to hold up a control blade during a fuel move. 

If a shutdown is incomplete due to control rod failures, analysis has shown that the core is shutdown if all control rods are inserted to at least position 02, except one, which can be full out. 
Source Range Monitor (SRM)

SRMs are used to measure power in low-power regimes and refueling periods. Four SRMs are inserted/retracted into the core when needed to limit detector depletion ? they are withdrawn when criticality is reached. Criticality is reached once you reach a period of about 280 seconds, though you don?t necessarily consider yourself critical if you measure a shorter period, since locally you might go critical from pulling a rod adjacent to one of the SRMs. When you are close to criticality, the periods on all four SRMs become similar and stable. For a BOC startup, you reach criticality around 185. Starting up after an outage, temperature will increase at about 20/hour. The heatup rate is limited to 100/hour. 

The SRMs are proportional counter fission chamber detectors made of titanium with a thin layer of U3O8 on the interior. These detectors are pressurized with argon as the fill gas. Fission in the enriched layer results in highly-charged fission products that ionize the argon, producing electrons that are collected at the anode. SRMs are operated in pulse mode. Because SRMs are typically used in very low power operations and refueling, the fission rate is typically very low, while the gamma activity of the fission products is very high. Operating in pulse mode allows a lower level voltage discrimination to be set so that the low-amplitude pulses due to single events such as Compton scattering can be ignored, while the high-amplitude pulses due to fission product ionization are amplified.  

In refueling, there must be a continuous face-adjacent path between bundles and an SRM to avoid local criticality without detection by an SRM. Sources, which look something like fuel assemblies, can be placed in the core temporarily during refueling if there are insufficient counts. For a fresh core or a for a reactor that has been shut down for a long time, sources are inserted in order to obtain statistically significant neutron counts in the SRMs, not to obtain the neutron source needed to reach criticality. If you were to reach criticality without your knowledge, and then withdraw past prompt criticality, the core would suffer serious damage. An alternative to using a startup source would be to wait a long enough time between rod withdrawals for subcritical multiplication to lead to a detectable neutron level. 

Spent Fuel Pool (SFP)

Each SFP contains about 4000 assemblies. When fully flooded with water, the SFP must have keff < 0.95.  The pool temperature is maintained with 68 and 125 F. A bundle is considered ?cold? after five years in the SFP. The enrichment of spent fuel is above that of natural uranium. 

A 1-year inventory visual inspection of the SFP is required by the NRC, and a 6-year video camera inspection is simply a best practice in the industry. Any material containing greater than 1 gram of U-235 or plutonium is considered SNM (special nuclear material), and must be reported. All quantities less than 1 gram must still be tracked, but don?t need to be reported. 

During refueling, the SFP is allowed to connect directly to the reactor and steam separator/dryer pit. New fuel is located in the SFP. PWRs undergo a full-core offload during refueling. The refueling DBA is the dropping of a fuel assembly on top of another, causing it to release its fission products. 

Coupons
Coupon trees (heights of assemblies) are included to determine the severity of accumulated IASCC. Hot bundles freshly removed from a core following an outage are purposefully placed surrounding a coupon tree for conservatism. Otherwise, hot bundles cannot be put adjacent to each other in the pool. 

Failed Fuel Canisters
Badly failed fuel can be put in failed fuel canisters, which are essentially large rectangular boxes that surround a fuel assembly. april has a few of these canisters, but after damaging failed fuel even further after placing it in a canister, the practice was abandoned. 

Neutron Poisons
Boroflex (rubber + boron) inserts snap into place within each rack cell. However, being rubber, these components degrade quickly, hence the importance of the coupon tree surveillances. One of april?s pools uses Boroflex, while the other uses racks made of aluminum + boron, which has reduced irradiation-related degradation mechanisms. 

NFV: New Fuel Vault
Before loading into the SFP immediately before an outage, fuel is stored in the new fuel vault, which for april is located next to the SFP. april has four assemblies in the NFV because they are testing out new GNF fuel (lead test assemblies), and in case something goes wrong with those four assemblies, there are backups for quick replacement. The fastest Westinghouse has made a fuel bundle was over a weekend. 
Standby Liquid Control (SBLC)

SBLC (sometimes pronounced ?slick?) injects a sodium pentaborate (NaB5) mixture towards the bottom of the core inlet plenum area to cease fission in the event that the control rods fail to stop the chain reaction. Suction is taken from a storage tank containing heated NaB5. The mixture is kept heated to 75-85 to ensure that the poison remains dissolved in the liquid water. If the storage tank drops below 60, sodium pentaborate will precipitate from solution, forming a crystalline substance that is difficult to get back into solution should the temperature be again increased. Additional chemicals would need to be added to the tank to ensure dissolution. Borax and boric acid are mixed together at  in a separate tank, and an air stirring system is used to mix the solution. After being thoroughly mixed, the sodium pentaborate is added to the storage tank. The heated tank is often surrounded by short walls to prevent corrosion of other components in the event that there is a leak in the tank. This system is used for emergencies only, and has only ever been implemented on accident. 

The added solution has enough negative reactivity to overcome a decrease in voids to 0%, a decrease in fuel and moderator temperature to 68, xenon decay to zero concentration, and all control rods full out, all while also meeting shutdown margin. This correlates to a concentration of 660 ppm. 522 ppm will maintain the reactor shutdown in hot shutdown conditions. It takes about 15-20 minutes to fully inject the SBLC solution. During this time, the reactor requires a heat sink such as the main turbine-condenser or suppression pool. 

Boron will plate out in the core at hot spots and water/vapor interfaces. This likely would not impact the ability to shut down, since the design analysis accounts for 25% imperfect mixing. 


Startup [BWR]
1)	Begin with all rods in and RR pumps at low speed with flow control valves full open. The reactor is in the depressurized condition with cold coolant (less than 212 F). 
2)	Pull rods. When the reactor is still far from critical, usually entire groups of rods are pulled. As you pull rods, you begin to add nuclear heat to the coolant and see the heat rise above the decay heat level (Point of Adding Heat (POAH)). [If a transient occurs below the POAH, there is no temperature feedback effect]. Because the MSIVs are closed, the reactor slowly begins to increase in pressure and temperature, since no coolant is allowed to escape (boiling kettle). In this regime, natural circulation helps to increase flow without manually increasing flow, since the density in the downcomer is much greater than the density at the core exit. 
a.	Because you can be critical at any neutron population, ?critical? is defined when you reach a period of 300 seconds or you begin to add nuclear heat. 
3)	At about 11% power, and once you are at rated pressure, close the TBVs, which were dumping steam straight to the condenser, so that you can sync the turbine-generator. 
4)	At about 32% power, you cannot continue to increase power by rods or else you would enter the instability region. Increase flow to increase power.
5)	At about 50% flow, stop increasing power by flow, and switch to increasing power by pulling rods. When a control rod is removed, the 2-phase pressure drop in the core increases, with produces a flow resistance that  then actually causes flow to slightly decrease. 
6)	Continue to increase power by flow until reaching 100% power. 

During a startup, the reactor is maintained supercritical with a period of about 200 seconds. Periods less than 50 seconds are of concern from an operability standpoint. 

Fresh Core Startup
When there is burned fuel in the core, neutron decay leads to a steady-state neutron source population that can be manipulated to get a power-producing critical core. In a fresh core, there are not sufficient neutrons to reach criticality, so neutron sources are placed in fresh cores. Eight sources may be placed in the dry tubes that are removed after one or two fuel cycles. 

Syncing the Turbine-Generator
Syncing the generator to the grid requires than you manually match the grid phase to the generator phase so that there is no phase difference between the generator and the grid. Two light bulbs are use to gauge the proximity to the two electrical systems being in-phase ? when lit together, the generator has been properly synced to the grid. For DG performance tests, the DGs must be synced to the grid to demonstrate their power-producing abilities. During one DG test, the DG output electricity was not properly synced to the grid, which when connected sent huge amounts of energy to the DG, causing it to upend, twist, and fail. 

Steam Generators (SGs) [PWR]

SGs are elevated above the reactor so that in the case of a pump failure, coolant will maintain natural circulation to provide cooling. The SGs are enclosed in shield walls to minimize dose. SGs are either U-tube or once-through designs, both of which transfer heat from the primary coolant to the secondary coolant. 

The pressure in the primary loop is about 2200 psi, while the pressure in the secondary loop is about 1000 psi. Each SG has one hot inlet and one cold inlet for the primary coolant. Each SG typically has 4,000-6,000 tubes, each of which can be plugged to remove them from service. 

The shell side of the SG refers to the side containing the secondary coolant. Most PWRs have replaced one or more SGs. Russian VVER SGs are horizontal, which are less susceptible to tube damage. 

U-Tube Steam Generators
U-Tube SGs are manufactured by Westinghouse and CE. The heated primary coolant enters at the very bottom of the SG, and then travels upwards through thousands of tubes. These tubes then invert, and the cooled primary coolant exits the SG at the very bottom. A divider plate prevents the hot and cold primary coolant from mixing at the bottom. The top of the SG is tapered outwards to allow the steam separators and dryer to fit. Secondary coolant enters above the top of the inverted tubes. This preheated FW mixes with saturated liquid that has fallen down from the swirl vane steam separators and steam dryer. This FW plus downcomer flow is forced into a downcomer region and then upwards through the tube region. A plate ensures that all coolant exiting the ?core? region passed up through the moisture separators, from which it must be collected into saturated liquid until it can mix with the FW. The quality at the ?exit? of the tube region is about 25%. The heated secondary coolant then exits at the very top of the SG to the turbine. Horizontal plates with holes punched in them support the tubes, and anti-vibration bars support the bent parts of the tubes. 

Carryunder refers to significant quality in the downcomer region. A high quality in the downcomer region decreases the head available to send the secondary coolant up through the SGs, which can result in imperfect separation of steam once the mixture gets to the steam separators. 

A circulation ratio is defined as:


For a PWR, the circulation ratio is about 3, while for a BWR, is about 7, due to the fact that the RR system makes a closed loop forced circulation that does not necessarily exit the vessel directly as all the core flow in a PWR must. The circulation ratio is desirable to have as high as possible so that impurities are not concentrated in the bottoms of the SGs. 

Once-Through (Counterflow) Steam Generators (OTSG)
Once-through SGs are manufactured by B\&W, and are commonly found in 2-loop PWRs such as Davis Besse and TMI. Primary coolant enters at the very top of the SG, passes through vertical tubes, and exits the bottom, making one path through the SG. FW enters at the side of the SG, and passes through a downcomer region where it is preheated to saturation by some steam that is allowed to pass through a port in the shroud from the ?core? region to the annulus region. The FW then enters the inlet plenum area, travels upwards through the tubes, and then turns around and exits at an elevation only slightly higher than the incoming FW location. The exit FW is heated to a minimum of 35 F superheat. 

Some once-through SGs contain an economizer, which is essentially a different method by which the secondary coolant becomes saturated liquid before entering the tube region. Instead of a port in the shroud allowing the mixing of steam with incoming subcooled FW, the FW is heated to the saturated state by the economizer, which uses superheated steam extracted from the SG outlet to heat the incoming FW. These IEOTSGs (Integral Economizer Once-Through SGs) typically have FW and steam ports located closer to the bottom of the SG, since there does not need to be as much vertical height to allow preheating of the FW based on ports in the shroud. 


Transients

ATWS: Anticipated Transient Without Scram
In 2006, april?s generator had just been disconnected form the grid for a refueling outage, when the reactor scrammed. Three control rods didn?t go full in, and indications didn?t tell the operators exactly where they were. A site general emergency was declared due to the potential that SDM might not be met once the reactor went re-critical following cool down. The control rods did not enter fully due to channel distortion. 

CRDA: Control Rod Drop Accident
In a BWR, it is possible for a control rod to become decoupled from its drive mechanism, causing it to stick in the core as its drive mechanism is withdrawn for normal control rod movements. The control rod may then become unstuck, and fall out of the core, where it can both insert positive reactivity very rapidly and mechanically damage the drive mechanism as it falls. The velocity limiter restricts the control rod drop velocity to less than 3.11 ft/s, a speed set to limit fuel damage due to large power changes. Fuel damage can occur at a fuel enthalpy of 280 cal/g. 

The CRDA is most severe at low powers, when the moderator temperature may be lower, leading to shorter thermal neutron migration lengths. The shorter the average length traveled by the thermal neutrons, the less likely the neutrons that were initially controlled by the rod to be controlled by a nearby control rod. 

LOOP: Loss of Offsite Power
All safety systems that require electricity are powered by offsite power, since these emergency systems could still be needed when the plant is in shutdown conditions (not Modes 1 or 2) and cannot produce its own electricity. LOOPs contribute about 70% to the risk of nuclear power plants. LOOP events typically initiate in the switchyard.

BWR: The generator is essentially cut off from the grid, causing a generator load reject, which then causes a turbine trip to prevent additional energy from being sent to the generator, where it could not be dissipated. Once the turbine trips, the TSVs close, which causes a pressure increase in the reactor, which causes a scram. Following the scram, the TBVs open to allow steam to pass directly to the condenser. 

LOCA: Loss of Coolant Accident
A LOCA accident is assumed to initiate with the instantaneous circumferential rupture of a 24-inch RR line or the 26-inch main steam line. Maximum drywell and suppression chamber pressures will likely occur towards the end of the blowdown phase of the accident (blowdown, refill, reflood). The same peak pressure will occur for either break of a RR or MSL line. The most severe drywell temperature increase would result from a small break above the water level in the reactor, since the enthalpy of steam is about twice that of water. 




Xenon
Peak xenon following a power change is reached at about the square root of the change in % power. 







\end{document}