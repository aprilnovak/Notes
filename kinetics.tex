\section{Time-Dependent Solutions}
\label{sec:Kinetics}

This section describes how time-dependent solutions to the \gls{nte}, or to simplifications of the \gls{nte} such as the diffusion equation, are performed. These solutions can either be obtained through solution of the time-dependent equations or through simplifications that do not track the spatial evolution in time, but use simplified feedback models to compute the changes in power. The first of these methods is described in Section \ref{sec:TimeDependence}, while the second is described in Section \ref{sec:PseudoSteadyState}. Both methods typically use an \gls{ic} determined from a steady state eigenvalue calculation that corresponds to the initial steady state of the system. Both methods also rely on several kinetics parameters described in this section that are frequently computed by performing adjoint calculations and weighting against the adjoint flux \cite{tyobeka}.

The time response of a system is dependent on the extent to which delayed neutrons influence the behavior. Significant neutron hold-up may occur in large reflectors, which is commonly approximated either with an additional fictitious delayed neutron group \cite{xin_wang_thesis} or with a modified neutron lifetime. The delayed neutron fraction is highest for Th-232 and U-238, and lowest for Pu-239 and U-233. In \glspl{lwr}, as U-238 depletes and converts to Pu-239, a greater fraction of fission occurs due to PU-239, so \(\beta\) generally decreases with burnup. A typical range in \(\beta\) for \glspl{lwr} is from 0.007 to 0.005. In CANDU reactors where the uranium is at natural enrichment levels, the greater \(\beta\) due to U-238 dominates, giving one of the largest \(\beta\) for modern reactor designs. Due to the greater fraction of Pu-239 in fast systems, typical \(\beta\) values are on the order of 0.0036 for heterogeneous designs, approximately half that in \glspl{lwr}. In heterogeneous designs where U-238 is placed in a lower-flux region for transmutation to Pu-239, the effective \(\beta\) is somewhat lower than in homogeneous designs due to the lower importance of U-238, a high-\(\beta\) nuclide. In systems where transmutation of Pu-239 to Pu-241 occurs over core life, \(\beta\) tends to increase with burnup, though this can be overcompensated by the buildup of U-233.

If the reactor is critical, this criticality is sustained by both prompt and delayed neutrons. If the reactivity exceeds the delayed neutron fraction \(\beta\), criticality can be sustained by prompt neutrons alone, effectively decreasing the system time constant to the time constant characterizing the prompt neutrons. Because this time constant is much smaller than the time constant characterizing both prompt and delayed neutrons, this situation is often referred to as ``prompt criticality,'' and will generally lead to very rapid increases in power unless there are sufficiently negative, fast-acting, feedback mechanisms. 

The reactivity is the fractional change in the neutron population relative to the current generation:

\beq
\label{eq:ReactivityDef}
\rho=\frac{n_ok-n_o}{n_ok}=\frac{k-1}{k}
\eeq

The analysis of reactor kinetics is typically simplified so that \(\rho\) is a known quantity, though in reality there is a strong interrelationship with operating conditions such as temperature and density. The remainder of this section presents various methods for accounting for the transient behavior in Eq. \eqref{eq:nte1}.

\subsection{Time-Dependent Transport Calculations}
\label{sec:TimeDependence}

A direct time-dependent solution of Eq. \eqref{eq:nte1} can be performed by discretization of the time derivative. Using an implicit Euler discretization for both Eq. \eqref{eq:nte1} and the coupled delayed neutron precursor concentration equations rearrangement shows that the solution at each time step is equivalent to a steady-state fixed-source problem with slightly modified fission and external source terms \cite{pautz}. Therefore, adaptation of a steady-state fixed source solver to transient problems can be easily performed. For example, implicit Euler discretization of both Eqs. \eqref{eq:nte1} and \eqref{eq:DelayedNeutrons}, and substitution of Eq. \eqref{eq:DelayedNeutrons} into Eq. \eqref{eq:nte1} for \(C_j\) at the current time step gives:

\beqa
\label{eq:TimeNTE}
\hO\cdot\psi^{(n+1)}(\vv{r},E,\hO)+\left\lbrack\Sigma_t^{(n+1)}(\vv{r},E,\hO)+\frac{1}{v(E)\Delta t}\right\rbrack\psi^{(n+1)}(\vv{r},E,\hO)=\hspace{2cm}\\
\int_0^\infty dE'\int_{4\pi}d\hO\Sigma_s^{(n+1)}(\vv{r},E'\rightarrow E,\hO)\psi^{(n+1)}(\vv{r},E',\hO')+\hspace{1cm}\\
\left\lbrack\chi_p(E,\hO)\left(1-\beta\right)+\sum_{j=1}^J\frac{\chi_{d,j}(E,\hO)\lambda_j\beta_j}{1+\Delta t\lambda_j}\right\rbrack\int_0^\infty dE'\int_{4\pi}d\hO'\nu(E')\Sigma_f^{(n+1)}(\vv{r},E',\hO')\psi^{(n+1)}(\vv{r},E,\hO)+\\
\sum_{j=1}^J\frac{\chi_{d,j}(E,\hO)\lambda_j}{1+\Delta t\lambda_j}C_j^{(n)}(\vv{r})
\eeqa

where \((n+1)\) and \((n)\) superscripts indicate values at the \((n+1)\)-th and \(n\)-th time steps and \(\Delta t\) is the time step size. A smaller time step is frequently used to ensure that Eq. \eqref{eq:TimeNTE} is more diagonally-dominant, improving convergence \cite{tyobeka}.

During the linear system solution, it is customary to use the solution from the previous time step as the initial guess for the iterative solver. This can be even further improved by extrapolating the solution from the previous time step using computed reactor periods based on the ratio in the flux between the previous two time steps \cite{pautz}:

\beq
\label{eq:FluxExtrapolation}
\psi^{(n+1,0)}=\psi^{n}\exp{\left\lbrack\frac{1}{\Delta t_n}\ln{\left(\frac{\psi^{(n)}}{\psi^{(n-1)}}\right)}\Delta t_{n=1}\right\rbrack}
\eeq

Eq. \eqref{eq:FluxExtrapolation}, though simple in concept, has shown excellent acceleration capabilities \cite{tyobeka}.

\subsection{Pseudo Steady-State Calculations}
\label{sec:PseudoSteadyState}

A pseudo steady-state transient can be analyzed using a succession of criticality calculations coupled with a simple model for reactor kinetics such as the \gls{pke} in Eq. \eqref{eq:PKE}. These approximations assume that the flux spatial distribution varies slowly in time. For example, at time step \(i\), a criticality calculation is performed using the eigenvalue form of the \gls{nte} in Eq. \eqref{eq:EigenvalueNTE} with coupling to a \gls{th} module via Picard iteration based on the power at time step \(i\). This calculation computes the fundamental mode flux and the corresponding eigenvalue. This eigenvalue is then used in a simple model for reactor kinetics, such as the \gls{pke}, which requires as input the reactivity defined in Eq. \eqref{eq:ReactivityDef}. The result of the simple kinetics model is to predict the power change over time step \(i\) by evolving the \gls{pke} solution from time \(i\) to \(i+1\). The power at step \(i+1\) is then used as an input to the \gls{th} module to compute a new flux distribution, and the iterative process is repeated until the end of the transient. These methods permit the flux spatial distribution to vary in time, but because the simpler kinetics methods are commonly based on diffusion theory, there are many layers of approximation present. 

Alternatively, the simplest method only uses a simple kinetics approximation to advance an initial fundamental spatial mode in time. Feedback may be included through the use of lumped \gls{th} models. 

\subsection{Time Eigenfunction Approximation}

For the one-group diffusion approximation, an eigenfunction analysis using separation of variables for a bare reactor provided the solution in Eq. \eqref{eq:FluxEigenfunctionExp}. One technique for approximating the time dependence of the one-group diffusion equation is to assume the flux can be given simply by the leading \(n=1\) term. From Eq. \eqref{eq:lambda_n}, \(\lambda_{n+1}\) is given in terms of \(\lambda_n\) as:

\beq
\lambda_{n+1}=\lambda_n+2vD(n+1)\left(\frac{\pi}{\tilde{a}}\right)^2
\eeq

For typical values characterizing thermal systems, \(\lambda_{n}\), \(n\neq1\) are several orders of magnitude larger than \(\lambda_1\), and decay away very quickly in time. Therefore, a reasonable approximation for the time dependence is the use of only the leading \(n=1\) term in Eq. \eqref{eq:FluxEigenfunctionExp}:

\beq
\phi(x,t)=C_1(x,t)e^{-\lambda_1t}\cos{\left(\frac{\pi}{\tilde{a}}x\right)}
\eeq

where \(\lambda_1\) can be written in terms of \(k=pfP_{NL}\) as:

\beqa
\lambda_1=&\ v\left\lbrack D\left(\frac{\pi}{\tilde{a}}\right)^2+\Sigma_a-\nu\Sigma_f\right\rbrack\\
=&\ \frac{k-1}{l}
\eeqa

where \(l\) is the prompt neutron lifetime in a finite reactor accounting for leakage:

\beq
l\equiv l_\infty P_{NL}
\eeq

where \(l_\infty\) is the prompt neutron lifetime in an infinite reactor, simply taken as the mean lifetime from birth to absorption:

\beq
l_\infty\equiv\frac{1}{v\Sigma_a}
\eeq

\(l\) is on the order of 10$^{-4}$ seconds for typical thermal reactors, and 10$^{-7}$ seconds for typical fast reactors. This simple eigenfunction analysis of the time dependence predicts a time dependence of the neutron population as:

\beq
\label{eq:EigenfunctionNt}
n(t)=n_0\exp{\left(\frac{k-1}{l}t\right)}
\eeq

The reactor period in this case is extremely small, and precludes any realistic reactor control. Because \(l\leq l_\infty\), the reactor period will be smaller in systems with significant leakage. Eq. \eqref{eq:EigenfunctionNt} can be slightly improved in accuracy by replacing \(l\) by an ad hoc weighted \(l\) that accounts for the effects of both prompt and delayed neutrons:

\beq
\label{eq:AverageL}
\langle l\rangle\equiv(1-\beta)l+\sum_{j=1}^J\beta_j\left(l+\frac{1}{\lambda_j}\right)
\eeq

\(\langle l\rangle\) is on the order of 10$^{-1}$ seconds for thermal systems, approximately three orders of magnitude larger than the prompt neutron lifetime \(l\). However, this simple approximation applies a single time constant to describe the neutron population, when in reality a complicated interrelationship between time constants characterizing the prompt neutrons and each of the \(J\) delayed neutron groups exists. Eq. \eqref{eq:EigenfunctionNt} with \(l\) given by Eq. \eqref{eq:AverageL} should only be used for very small reactivity changes. Section \ref{sec:PKE} describes an improvement to this model.

\subsection{The Point Reactor Kinetics Equations}
\label{sec:PKE}

The \gls{pke} are an approximation to the time dependence in Eq. \eqref{eq:nte1} that assume 1)~a one-group diffusion theory model, and 2)~a scalar flux that is separable in space and time. This approximation therefore implies that the flux profile is independent of time, and during a transient only changes by multiplication by a constant, clearly an oversimplification of very spatially-dependent events such as withdrawal of a control rod. For a transient initiated by only a small change in a control rod position, the use of the \gls{pke} showed an excellent agreement with both diffusion theory and S$_N$ for the \gls{pbmr} \cite{tyobeka}. The \gls{pke} is more applicable sufficiently long after an initial transient when effects have permeated through the entire system such that the system responds with the same time response in all locations.

The assumption of no energy dependence neglects the fact that delayed neutrons are born at considerably lower energies than prompt neutrons. When these factors are combined, it is clear that a more sophisticated analysis would use different spatial distributions to describe the group fluxes since they have considerably different spatial dependence in thermal reactors. The tighter neutron transport coupling in fast reactors results in the \gls{pke} being a better approximation than in loosely-coupled thermal systems. 

The time rate of change of delayed neutron precursors is given by Eq. \eqref{eq:DelayedNeutrons}, with the one-group diffusion approximation resulting in \(\nu\Sigma_f\) and \(\psi\) being replaced by the appropriate group constants and scalar flux,

\beq
\label{eq:Precursor1}
\frac{\partial C_i(\vv{r},t)}{\partial t}=\beta_i\nu\Sigma_f\phi(\vv{r},t)-\lambda_iC_i(\vv{r},t)
\eeq

The flux is described by the one-group diffusion equation:

\beq
\label{eq:Power1}
\frac{\partial \phi(\vv{r},t)}{\partial t}-\nabla\cdot\left\lbrack D\nabla\phi(\vv{r},t)\right\rbrack+\Sigma_a\phi(\vv{r},t)=(1-\beta)\nu\Sigma_f\phi(\vv{r},t)+\sum_{j=1}^J\lambda_jC_j(\vv{r},t)
\eeq

Assuming both the neutron and precursor concentrations are separable in space and time and that both are described by the fundamental mode obtained from a solution to the homogeneous Helmholtz problem, insert the following assumed forms into Eqs. \eqref{eq:Precursor1} and \eqref{eq:Power1}:

\beq
C_j(\vv{r},t)=C_j(t)\psi(\vv{r})
\eeq

\beq
\phi(\vv{r}t)=vn(t)\psi(\vv{r})
\eeq

which gives:

\begin{subequations}
\label{eq:PKE}
\begin{eqnarray}
\frac{\partial n(t)}{\partial t}&=&\frac{\rho(t)-\beta}{\Lambda}n(t)+\sum_{j=1}^J\lambda_jC_j(t)\\
\frac{\partial C_j(t)}{\partial t}&=&\frac{\beta_j}{\Lambda}n(t)-\lambda_jC_j(t)
\end{eqnarray}
\end{subequations}

where Eq. \eqref{eq:Helmholtz} was used to relate \(\nabla^2\psi(\vv{r})\) to \(B_n^2\psi(\vv{r})\). Note that the assumption of the same spatial dependence for \(\phi\) and \(C_i\) does not capture the fact that \(C_i\) is actually proportional to \(\Sigma_f\phi\), and could be characterized by a very different profile from the scalar flux. In addition, the use of the fundamental mode implies that the system is in the critical state, which does not apply for the time-dependent investigations that are the subject of this section. So, the \gls{pke} are only valid for systems slightly perturbed from the critical state.

Eq. \eqref{eq:PKE} are known as the \gls{pke}, and are commonly written in terms of power, rather than neutron concentration, by multiplying both equations by \(w_f\Sigma_f\), which is typically bundled into the \(C_j(t)\). This substitution neglects any delay between the fission rate and thermal power production due to conduction through the fuel and cladding. \(\Lambda\) is defined as the mean neutron lifetime from birth in fission to absorption in thermal fission:

\beqa
\label{eq:LambdaDef}
\Lambda\equiv&\frac{l}{k}\\
=&\frac{1}{v\nu\Sigma_f}
\eeqa

If \(k>1\), \(\Lambda<l\), indicating that the lifetime is shortened by enhanced absorption fission rates. The \gls{pke} can be derived with the one-group and space-time separability assumptions relaxed, which gives the same set of equations in Eq. \eqref{eq:PKE} with slightly different selections for \(\rho\), \(\beta\), and \(\Lambda\) \cite{duderstadt}. Alternatively, if the transient is sufficiently slow to evolve, a series of pseudo steady-state eigenvalue calculations can be used to compute approximate \(\psi(\vv{r},t)\) in an approximation known as the ``adiabatic approximation.''

Eq. \eqref{eq:PKE} are a system of coupled \glspl{ode} that are generally very stiff due to the wide range in time scales, from \(l\) to the longest-lived delayed neutron precursor group, which may have a half life on the order of 80 seconds. If \(\rho(t)\) is a prescribed function of time, the \gls{pke} are a coupled set of linear equations. However, in general \(\rho\) also depends on the neutron concentration, yielding a coupled set of nonlinear equations. Only for bare reactor systems modeled by the one-group diffusion equation with homogeneous properties can \(\rho\) be {\it computed} based on a change in material properties and/or composition. In other cases, \(\rho\) must be a specified function of the \gls{th} state or computed from eigenvalue calculations. 

The remainder of this section describes several simplifications to Eq. \eqref{eq:PKE} that allow more physical intuition when \(\rho\) is a known function of time. %However, because many reactor period measurements are taken at close to zero power, feedback models are typically not required for their analysis, and the \gls{pke} with a constant \(\rho_0\) can often be applied.

\begin{comment}
The multiplication factor \(k\) is the ratio of the rate of neutron production to the rate of neutron loss:

\beqa
\label{eq:KDef}
k\equiv&\frac{\text{rate of neutron production}}{\text{rate of neutron loss}}\\
\equiv&\frac{P(t)}{L(t)}
\eeqa

where \(P\) is the rate at which neutrons are produced in the system and \(L\) is the rate at which neutrons are lost from the system by absorption and leakage. If the neutron lifetime were constant for all neutrons, i.e. ignoring that in some scenarios a neutron may induce fission as their first reaction after birth, but in others scatter 100 times until absorption in non-fissile material, then \(k\) could easily be defined as:

\beq
\label{eq:KDef2}
k\equiv\frac{\text{number of neutrons in one generation}}{\text{number of neutrons in previous generation}}
\eeq

However, because the neutron lifetime \(l\) is not constant and identical for all neutrons, Eq. \eqref{eq:KDef} is a more useful operational definition for \(k\) than Eq. \eqref{eq:KDef2}.
\end{comment}

\subsubsection{One Delayed Neutron Group}
\label{sec:OneDelayedGroup}
A common simplification to Eq. \eqref{eq:PKE} is the assumption of a single delayed neutron group. In this case, the single group is characterized by the total \(\beta\) defined in Eq. \eqref{eq:BetaDef} and a weighted delayed neutron decay constant based on a simplification to Eq. \eqref{eq:AverageL}:

\beq
\langle\lambda\rangle=\left(\frac{1}{\beta}\sum_{j=1}^J\frac{\beta_j}{\lambda_j}\right)^{-1}
\eeq

Eq. \eqref{eq:PKE} then simplify to the following:

\begin{subequations}
\label{eq:PKE2}
\begin{eqnarray}
\frac{\partial n(t)}{\partial t}&=&\frac{\rho(t)-\beta}{\Lambda}n(t)+\langle\lambda\rangle C(t)\\
\frac{\partial C(t)}{\partial t}&=&\frac{\beta}{\Lambda}n(t)-\langle\lambda\rangle C(t)
\end{eqnarray}
\end{subequations}

If the reactivity is a constant \(\rho_0\) independent of time, a solution to Eq. \eqref{eq:PKE2} exists of the form:

\beq
\label{eq:n1}
n(t)=n_0e^{\omega t}
\eeq

\beq
\label{eq:C1}
C(t)=C_0e^{\omega t}
\eeq

Inserting these assumptions into Eq. \eqref{eq:PKE2} provides a quadratic for \(\omega\) that must be satisfied to obtain a non-trivial solution:

\beq
\label{eq:OmegaQuadratic}
0=\Lambda\omega^2+\left(\lambda\Lambda+\beta-\rho_0\right)s-\rho_0\lambda
\eeq

Solution of this quadratic in Eq. \eqref{eq:OmegaQuadratic} gives two possible values for \(\omega\):

\beq
\omega=\frac{-(\lambda\Lambda+\beta-\rho_0)\pm\sqrt{(\lambda\Lambda+\beta-\rho_0)^2+4\Lambda\lambda\rho_0}}{2\Lambda}
\eeq

These two possible solutions requires modification of Eqs. \eqref{eq:n1} and \eqref{eq:C1} to the following general solutions:

\beq
\label{eq:n2}
n(t)=n_1e^{\omega_1t}+n_2e^{\omega_2t}
\eeq

\beq
\label{eq:C2}
C(t)=C_1e^{\omega_1t}+C_2e^{\omega_2t}
\eeq

Therefore, inserting the assumed solutions in Eqs. \eqref{eq:n2} and \eqref{eq:C2} into Eq. \eqref{eq:PKE2},

\begin{subequations}
\label{eq:PKE3}
\begin{eqnarray}
n_1\omega_1e^{\omega_1t}+n_2\omega_2e^{\omega_2t}&=&\frac{\rho(t)-\beta}{\Lambda}\left(n_1e^{\omega_1t}+n_2e^{\omega_2t}\right)+\langle\lambda\rangle \left(C_1e^{\omega_1t}+C_2e^{\omega_2t}\right)\\
C_1\omega_1e^{\omega_1t}+C_2\omega_2e^{\omega_2t}&=&\frac{\beta}{\Lambda}\left(n_1e^{\omega_1t}+n_2e^{\omega_2t}\right)-\langle\lambda\rangle \left(C_1e^{\omega_1t}+C_2e^{\omega_2t}\right)
\end{eqnarray}
\end{subequations}

and application of initial conditions under the assumption of zero time rate of change at \(t=0\) (i.e. the system is initially in steady state), 

\beq
n_0=n_1+n_2
\eeq

\beq
\frac{\beta}{\Lambda\langle\lambda\rangle}n_0=C_0+C_1
\eeq

gives four conditions for the solution of \(n_1\), \(n_2\), \(C_1\), and \(C_2\). This solution is not shown here; in the limit that \((\Lambda\lambda+\beta-\rho_0)^2 \gg 4\Lambda\lambda\rho_0\) and \(\abs{\rho_0}\ll\beta\) (i.e. a small reactivity insertion), the roots are approximately given as:

\beqa
\label{eq:omegaPKE2}
\omega_1\approx&\frac{\langle\lambda\rangle\rho_0}{\beta-\rho_0}\\
\omega_2\approx&-\frac{\beta-\rho_0}{\Lambda}\\
\eeqa 

and the neutron population is given as:

\begin{equation}
\label{eq:PowerPKE2}
n(t)=n_0\left\lbrack\frac{\beta}{\beta-\rho_0}e^{\omega_1t} - \frac{\rho_0}{\beta-\rho_0}e^{\omega_2t}\right\rbrack
\end{equation}

The transient behavior will now be investigated for example reactivity insertions of \(\rho_0=\pm0.001\) for \(\beta=0.006\), \(\Lambda=10^{-4}\) seconds, and \(\langle\lambda\rangle=0.08\) s$^{-1}$. For a positive and negative reactivity insertion, Eq. \eqref{eq:PowerPKE2} becomes:

\beq
n(t)=\begin{cases}n_0\left\lbrack1.20e^{0.016t}-0.20e^{-5t}\right\rbrack & \rho_0>0\\
n_0\left\lbrack0.85e^{-0.011t}+0.14e^{-7t}\right\rbrack & \rho_0<0\end{cases}
\eeq

Provided \(\rho_0<\beta\), \(\omega_2\) is always negative, while \(\omega_1\) is positive for positive insertions and negative for negative insertions. For a reactivity change of less than one dollar, a prompt positive/negative jump in response to a positive/negative insertions, occurs due to nearly instantaneous changes in multiplication of the prompt neutrons. This prompt jump is governed by \(\omega_2\). The long-term response is governed by multiplication of the delayed neutrons, and hence is governed by \(\omega_1\). Both the short-term and long-term response are exponential. Because \(l\) only appears in \(\omega_2\), the long-term behavior of thermal and fast system is approximately the same for sufficiently small reactivity insertions.

The reactor response is undefined for \(\rho_0=\beta\). For \(|\rho_0|>\beta\), very different behavior exists depending on whether the insertion is positive or negative. In the limit of an infinitely large negative insertion, \(\omega_1=-\langle\lambda\rangle\) and \(\omega_2\rightarrow-\infty\) such that the long-term reactor period is simply equal to \(1/\langle\lambda\rangle\). Therefore, no matter how negative an insertion, the {\it time rate of change} of the neutron population is limited by the delayed neutron time constant because some time must be allowed to permit delayed neutron precursors to decay. Note that the amount of negative reactivity will still determine the initial prompt drop in power, and the larger the negative insertion, the more substantial this drop. In the limit of an infinitely large positive insertion, \(\omega_1=-\langle\lambda\rangle\) and \(\omega_2\rightarrow\infty\), leading to an infinitely fast rate of increase in the power. For negative insertions, the rate of change is limited by the delayed neutron precursors, but no such limit holds for positive insertions.

\subsubsection{The Inhour Equation}
\label{sec:Inhour}

The single delayed neutron group analysis shown in Section \ref{sec:OneDelayedGroup} assumed a single delayed neutron group. A simple ad hoc extension to multiple delayed neutron groups can be performed by rearranging Eq. \eqref{eq:OmegaQuadratic} using the definitions in Eqs. \eqref{eq:ReactivityDef} and \eqref{eq:LambdaDef}:

\beq
\rho_0=\frac{\omega l}{\omega l+1}+\frac{\beta s}{(\omega l+1)(\omega+\lambda)}
\eeq

An extension to multiple delayed neutron groups can be achieved by simply replacing the second term using an average over the delayed neutron groups:

\beq
\label{eq:Inhour}
\rho_0=\frac{\omega l}{\omega l+1}+\frac{1}{\omega l+1}\sum_{j=1}^J\frac{\beta_j\omega}{s
\omega+\lambda_j}
\eeq

Eq. \eqref{eq:Inhour} is known as the ``inhour equation'' due to early attempts to solve Eq. \eqref{eq:Inhour} to determine the reactivity required to obtain a reactor period equal to one hour. Eq. \eqref{eq:Inhour} does not permit a solution for \(\rho_0>1\). The neutron concentration is now described as an exponential function of the \(J+1\) \(\omega\) values that are the roots to Eq. \eqref{eq:Inhour},

\beq
n(t)=\sum_{j=1}^{J+1}n_je^{\omega_jt}
\eeq

Only one root will always be positive; all other roots die out quickly in time, and hence the reactor period is governed by the largest root to Eq. \eqref{eq:Inhour}.

\subsubsection{The Prompt Jump Approximation}

As was seen in Sections \ref{sec:OneDelayedGroup} and \ref{sec:Inhour}, the time response of a nuclear system is typically characterized by several very small time constants and one large time constant that governs the long-term behavior. A simplification to the \gls{pke} in Eq. \eqref{eq:PKE} can be made by neglecting the presence of prompt neutrons entirely, i.e. \(l=0\). With this simplification, Eq. \eqref{eq:PKE}a becomes, after recognizing that the delayed neutron precursor concentration changes much slower than the neutron concentration,

\beq
\label{eq:PromptJump}
\frac{n_2}{n_1}=\frac{\beta-\rho_1}{\beta-\rho_2}
\eeq

where ``1'' refers to the initial state and ``2'' to the final state. Eq. \eqref{eq:PromptJump} only indicates the power immediately following the insertion, i.e. after the prompt jump has occurred. The power will continue to vary exponentially in time, and this approximation cannot capture this behavior. The prompt jump approximation is valid to within about 1\% for small reactivity insertions.

The \gls{sdm} is the negative reactivity inserted into the reactor with the highest worth control rod fully ejected; assuming the entirety of the \gls{sdm} is inserted into a system with zero excess reactivity, Eq. \eqref{eq:PromptJump} predicts that the power immediately following the negative insertion is directly related to the \gls{sdm}:

\beq
\frac{P_2}{P_1}=\frac{1}{1+\rho_{sdm}/\beta}
\eeq

where \(\rho_{sdm}\) is the \gls{sdm}. A higher \gls{sdm} permits a more rapid prompt power reduction following a scram, though the long-term shutdown behavior is still governed by the delayed neutron precursors.
