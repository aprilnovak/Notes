\section{The Diffusion Equation}
\label{sec:Diffusion}

The diffusion equation is the simplest form of the \gls{nte} that is derived by assuming a \(P_1\) expansion for the angular dependence of the angular flux according to the form in Eq. \eqref{eq:ScatteringLegendre}:

\beqa
\label{eq:FluxLegendreP1}
\psi\seat\approx&\sum_{l=0}^1\frac{2l+1}{4\pi}\psi_l\sset P_l(\hO)\\
=&\frac{1}{4\pi}\psi_0\sset+\frac{3}{4\pi}\psi_1\sset\hO\\
\eeqa

where \(\psi_l\) is the \(l\)-th moment of the angular flux. Based on the definitions of the first few Legendre polynomials in Eq. \eqref{eqn:LegendrePolynomials_P0P1P2}, a \(P_1\) approximation is equivalent to assuming linear anisotropy in the angular flux. Using a linear approximation for anisotropy is reasonably accurate if the angular flux is only weakly dependent on angle. To interpret the physical meaning of the flux moments \(\psi_0\) and \(\psi_1\), integrate Eq. \eqref{eq:FluxLegendreP1} over angle:

\beqa
\label{eq:FluxLegendreP1_AngleIntegration}
\int_{4\pi}^{} d\hO\psi\seat=& \frac{1}{4\pi}\int_{4\pi}^{}d\hO\psi_0\sset +\frac{3}{4\pi}\int_{4\pi}^{} d\hO \psi_1\sset\hO\\
\phi\spas = &\ \psi_0 + \frac{3}{4\pi}\psi_1\int_{4\pi}^{} d\hO  \hO\\
   = &\ \psi_0
\eeqa

where Eqs. \eqref{eq:SolidAngleIntegration} and \eqref{eq:OmegaCartesianIntegration} have been used. \(\phi\sset\) is the scalar flux, defined as the angular flux integrated over solid angle:

\beq
\label{eq:ScalarFluxDef}
\phi\sset\equiv\int_{4\pi}d\hO\psi\seat
\eeq

Eq. \eqref{eq:FluxLegendreP1_AngleIntegration}  shows that the zero-th moment of the angular flux is equivalent to the scalar flux. To interpret \(\psi_1\), multiply Eq. \eqref{eq:FluxLegendreP1} by \(\hO\) and then integrate over solid angle to utilize Eq. \eqref{eq:OmegaCartesianIntegration} to cancel the first term instead of the second:

\beqa
\label{eq:FluxLegendreP1_AngleIntegration2}
\int_{4\pi}^{} d\hO   \hO  \psi\spa  =& \frac{1}{4\pi}\int_{4\pi}^{} d\hO   \hO  \psi_0\sset + \frac{3}{4\pi}\int_{4\pi}^{} d\hO   \hO   \hO  \psi_1\sset\\
J\sset = &\ \psi_1\sset\\
\eeqa

where Eqs. \eqref{eq:OmegaCartesianIntegration} and \eqref{eq:4PiOmegaOmega} have been used. \(\vv{J}\sset\) is the current, defined as the integral of the angular current over solid angle:

\beq
\label{eq:Current}
\vv{J}(\vv{r},E,t)\equiv\int_{4\pi}^{}d\hO  \vv{j}\seat
\eeq

where \(\vv{j}\) is defined in Eq. \eqref{eq:AngularCurrent}. Eq. \eqref{eq:FluxLegendreP1_AngleIntegration2} shows that the first moment of the angular flux is equivalent to the current. Integrating the \gls{nte} in Eq. \eqref{eq:nte1} over angle, and using the definitions of scalar flux and current in Eqs. \eqref{eq:ScalarFluxDef} and \eqref{eq:Current} gives:

\beqa
\label{eq:NeutronContinuityEquation}
\frac{\partial}{\partial t}\left(\frac{\phi\sset}{v(E)}\right)+\nabla\cdot\vv{J}\sset+\Sigma_t\sset\phi\sset=\\
\int_{4\pi}d\hO\inscatteringsource\psi\seatout+\\
\int_{4\pi}d\hO\promptfissionsource\psi(\vv{r},E',\hO',t)+\\
\int_{4\pi}d\hO\delayedfissionsource)+Q\sset
\eeqa

where the streaming term was expressed as \(\nabla\cdot(nv\hO)\) as in Eq. \eqref{eq:Streaming} to obtain \(\nabla\cdot\vv{J}\). Eq. \eqref{eq:NeutronContinuityEquation} is sometimes referred to as the ``neutron continuity'' equation. A different version of the neutron continuity equation can be obtained by first multiplying Eq. \eqref{eq:nte1} by \(\hO\), and again integrating over solid angle:

\beqa
\label{eq:TEAngleAngleIntegrated2}
\frac{\partial}{\partial t}\left(\frac{\vv{J}\sset}{v(E)}\right)+\nabla\cdot\int_{4\pi}d\hO\psi\seat\hO\hO+\Sigma_t\sset\vv{J}\sset=\\
\int_{4\pi}d\hO\inscatteringsource\psi(\vv{r},E',\hO',t)\hO+\\
\int_{4\pi}d\hO\promptfissionsource\psi(\vv{r},E',\hO',t)\hO+\\
\int_{4\pi}d\hO\hO\delayedfissionsource+\int_{4\pi}d\hO Q\seat\hO
\eeqa

where the gradient was brought outside the angle integral (written this time as \(\hO\cdot\nabla\psi\) instead of as \(\nabla\cdot(nv\hO)\) as in Eq. \eqref{eq:Streaming}. The last three terms in Eq. \eqref{eq:TEAngleAngleIntegrated2} represent the first moments of the prompt fission source, the delayed fission source, and the external source. If the fission sources and their energy and angle distributions are assumed isotropic, as is common, both of the fission integrals are zero. Inserting the expansion in Eq. \eqref{eq:FluxLegendreP1} into Eq. \eqref{eq:TEAngleAngleIntegrated2} for \(\psi\), with Eqs. \eqref{eq:FluxLegendreP1_AngleIntegration} and \eqref{eq:FluxLegendreP1_AngleIntegration2} utilized, gives:

\beqa
\label{eq:P1a}
\frac{\partial}{\partial t}\left(\frac{\vv{J}\sset}{v(E)}\right)+\frac{1}{3}\nabla\phi\sset+\Sigma_t\sset\vv{J}\sset=\\
\int_{4\pi}d\hO\inscatteringsource\left\lbrack\frac{1}{4\pi}\phi(\vv{r},E',t)+\frac{3}{4\pi}\vv{J}(\vv{r},E',t)\hO'\right\rbrack\hO+\\
\int_{4\pi}d\hO\promptfissionsource\left\lbrack\frac{1}{4\pi}\phi(\vv{r},E',t)+\frac{3}{4\pi}\vv{J}(\vv{r},E',t)\hO'\right\rbrack\hO+\\
\int_{4\pi}d\hO\hO\delayedfissionsource+\int_{4\pi}d\hO Q\seat\hO
\eeqa

where Eqs. \eqref{eq:4PiOmegaOmega} and \eqref{eq:4PiOmegaOmegaOmega} have been used and \(\Sigma_{s1}\) is the first angular moment of the scattering cross section.

In addition, including the results from Eq. \ref{eq:4PiOmegaOmega}, the streaming term becomes:

\begin{equation}
\label{eq:P1b}
\begin{aligned}
\frac{1}{v(E)} \frac{\partial\vv{J}\spas}{\partial t} +
 \frac{1}{3}\nabla\phi + 
 \Sigma_t(\vv{r},E)\vv{J}\spas = \\
 \int_{0}^{\infty}dE' \int_{4\pi}^{ } d\hO  ' \Sigma_{s1}(\vv{r}, E'\rightarrow E)\vv{J}(\vv{r}, E', t) + \vv{S_1}\spas
\end{aligned}
\end{equation}

Bundling all source terms Eq. \ref{eq:NeutronContinuityEquation} in \(S\), this equation becomes:

\begin{equation}
\label{eq:NeutronContinuityEquation_BundledSource}
\begin{aligned}
\frac{1}{v(E)} \frac{\partial\phi\spas}{\partial t} +
 \nabla\cdot J\spas + 
 \Sigma_t(\vv{r},E)\phi\spas = \\
 \int_{0}^{\infty}dE' \Sigma_s(\vv{r}, E'\rightarrow E)\phi(\vv{r}, E', t) + S\spas\\
\end{aligned}
\end{equation}

Eqs. \ref{eq:P1b} and \ref{eq:NeutronContinuityEquation_BundledSource} are the energy-dependent \(P_1\) equations. From Eq. \ref{eq:ScatteringMomentsLegendre}, the scattering term in Eq. \ref{eq:P1b} can be simplified if we first define the average of the isotropic scattering angle:

\begin{equation}
\label{eq:IsotropicScatteringAngle}
\bar{\mu_o} = \frac{\int_{4\pi}^{} d\hO  \Sigma_s(\mu_o)\mu_o}{\Sigma_s} = \frac{\Sigma_{s1}}{\Sigma_s}
\end{equation}

Then, if we further make the one-speed approximation so that the integrals over energy disappear from Eqs. \ref{eq:P1b} and \ref{eq:NeutronContinuityEquation_BundledSource}, we obtain the 1-group \(P_1\) equations. 

\begin{equation}
\label{eq:P1b_1group}
\frac{1}{v} \frac{\partial\vv{J}(\vv{r}, t)}{\partial t} +
 \frac{1}{3}\nabla\phi + \Sigma_t(\vv{r}\vv{J}(\vv{r},t) = \bar{\mu_o}\Sigma_s\vv{J}(\vv{r}, t) + \vv{S_1}(\vv{r}, t)
\end{equation}

\begin{equation}
\label{eq:NeutronContinuityEquation_BundledSource_1group}
\frac{1}{v} \frac{\partial\phi(\vv{r}, t)}{\partial t} +
 \nabla\cdot J(\vv{r}, t) + 
 \Sigma_t(\vv{r})\phi(\vv{r}, t) = \Sigma_s(\vv{r})\phi(\vv{r}, t) + S(\vv{r}, t)
\end{equation}

It is common to define a transport cross section as:

\begin{equation}
\label{eq:TransportCrossSection}
\Sigma_tr=\Sigma_t-\bar{\mu_o}\Sigma_s
\end{equation}

and then to introduce this into Eq. \ref{eq:P1b_1group}. Further approximations will lead to the derivation of the diffusion coefficient. By assuming that the isotropic source \(\vv{S_1}\) is zero, and rearranging Eq. \ref{eq:P1b_1group} with the definition for the transport cross section inserted:

\begin{equation}
\label{eq:P1b_CurrentApproximation}
\frac{1}{\abs{{\vv{J}}}} \frac{\partial\abs{{\vv{J}}}(\vv{r}, t)}{\partial t} = - v\frac{1}{3}\nabla\phi - v\Sigma_{tr}(\vv{r})\vv{J}(\vv{r},t) \approx 0
\end{equation}

We can assume from this equation that the time rate of change of the current is nearly zero in comparison to the total reaction frequency, since \(v\Sigma_{tr}\) is typically on the order of \(10^5\) per second or higher. The current does not tend to change extremely rapidly. Therefore, with this assumption, we can write:

\begin{equation}
\label{eq:P1b_CurrentApproximation_2}
\vv{J}(\vv{r},t) =  - \frac{1}{3\Sigma_{tr}(\vv{r})}\nabla\phi = -D\nabla\phi
\end{equation}

This defines the diffusion coefficient. The transport mean free path, or \(1/\Sigma_{tr}\), is essentially a corrected mfp that accounts for anisotropies in the elastic scattering process.  Because neutrons are biased towards forward scattering, the transport mfp is larger than the actual mfp. So, in this derivation of the diffusion equation, along with the derivation of \(D\), the following assumptions were made: 1) the angular flux is linearly anisotropic (\(P_1\) approximation), 2) the source is isotropic, and 3) the rate of change of the current is small. Only the assumption of linearly anisotropic angular flux is crucial to this derivation, since the other assumptions can be relaxed by working with the \(P_1\) equations instead of the 1-group diffusion equation. Weak dependence on angle is violated near boundaries or where material properties change rapidly over distances comparable to a mean free path, near localized sources, and in strongly absorbing media. When the angular flux has a strong angular dependence, it typically also has a strong spatial dependence. 

The energy-dependent diffusion equation can be derived from the \(P_1\) equations by again assuming that the rate of change of the current is small with respect to the total interaction frequency and the source is isotropic. So, making these assumptions in Eq. \ref{eq:P1b}, we can derive the energy-dependent version of Fick's law. 

\begin{equation}
\label{eq:P1b_energyDepdendent}
\vv{J}\spas = - \frac{1}{\Sigma_t(\vv{r},E)}\left(\frac{1}{3}\nabla\phi + \int_{0}^{\infty}dE' \int_{4\pi}^{ } \Sigma_{s1}(\vv{r}, E'\rightarrow E)\vv{J}(\vv{r}, E', t)\right)
\end{equation}

This does not give a very nice-looking diffusion coefficient... But, if we assume the scattering is isotropic in the lab frame (which is highly unrealistic – scattering is only approximately isotropic in the center of mass frame), then \(\Sigma_{s1}=0\) because there would be no angular dependence, which would reduce to the same version of Fick’s law as was obtained in the energy-independent case. Alternatively, bceause that approximation is rather bad, using Eq. \ref{eq:IsotropicScatteringAngle} to express the scattering term, and applying a delta function to the scattering integral so that the integral drops out, the energy-dependent diffusion coefficient can be defined as:

\begin{equation}
\label{eq:P1b_energyDepdendent2}
\vv{J}\spas = -\frac{1}{3\left(\Sigma_{tr}(\vv{r},E)\right)}\nabla\phi
\end{equation}

This diffusion coefficient assumed no anisotropic contribution to the energy transfer in a scattering collision. More rigorous derivations are possible with knowledge of neutron thermalization theory. The energy-dependent diffusion equation is therefore simply one of the energy-dependent \(P_1\) equations, Eq. \ref{eq:NeutronContinuityEquation_BundledSource}, with the energy-dependent Fick's law in Eq. \ref{eq:P1b_energyDepdendent2} inserted for \(J\). The fission source, once ``un-bundled'' from \(S\), would have the form:

\begin{equation}
\label{eq:FissionSource}
S_{fission}=\chi(E)\int_{0}^{\infty}dE'\nu(E')\Sigma_f(\vv{r}, E', t)\phi(\vv{r}, E', t)
\end{equation}

Diffusion theory in and of itself is not extremely accurate near boundaries, but through successful modification to boundary conditions and constants used, its application can yield sufficiently accurate results. For instance, the diffusion coefficient can be altered to account for anisotropic scattering. 

To derive the boundary condition for zero incoming/outgoing current, introduce the \(P_1\) approximation from Eq. \ref{eq:FluxLegendreP1} into Eq. \ref{eq:PartialCurrent}, where we have required that either the incoming or outgoing current be equal to zero:

\begin{equation}
\label{eq:PartialCurrent_BC1}
J_\pm \spas = \int_{2\pi\pm}^{} d\hO   \left(\frac{\phi_0\spas}{4\pi} + \frac{3}{4\pi} \hO  \phi_1\spas\right)\hO  \cdot\hat{n}
\end{equation}

Using Eqs. \ref{eq:OmegaDotOmega} and \ref{eq:P1b_CurrentApproximation_2}, the integral over half the solid angle becomes an integral over \(\theta\) and \(\phi\), and recalling that the integral of any two components of solid angle over \(4\pi\) equals \(4\pi/3\):

\begin{equation}
\label{eq:PartialCurrent_BC2}
J_\pm \spas = \frac{\phi}{4} \mp \frac{D}{2}\hat{n}\cdot\nabla\phi
\end{equation}

Applying this condition in 1-D geometry implies that if we linearly extrapolated the flux beyond the boundary, it would vanish at the extrapolated distance \(x = x_{surface} + 2D\):

\begin{equation}
\label{eq:PartialCurrent_BC2}
\frac{1}{\phi}\frac{d\phi}{dx}=-\frac{1}{2D}
\end{equation}

This motivates replacing the zero re-entrant current condition with the extrapolated boundary condition, which introduces some error, though diffusion theory is already somewhat inaccurate near boundaries. More advanced transport calculations suggest using the condition \(x = x_{surface} + 0.7104\lambda_{tr}\). This suffices unless the radius of curvature of the boundary is smaller than a mfp, in which case more complicated extrapolated boundary conditions could be derived. Because the transport mfp is often only several centimeters, and reactor cores are several meters in diameter, we can often assume that the flux goes to zero on the boundary, ignoring the extrapolated distance entirely.

It is common to define the diffusion length \(L\), which is essentially a measure of how far neutrons will diffuse from its birth to its death:

\begin{equation}
\label{eq:DiffusionLength}
L^2=\frac{D}{\Sigma_a}
\end{equation}

We can relate the diffusion length to the average distance traveled by the neutron, which will depend on which coordinate system we choose. In general, the average traveled distance is:

\begin{equation}
\label{eq:AverageDistance}
\bar{x}=\int_{0}^{\infty}xp(x)dx
\end{equation}

where \(p(x)\) is the probability of absorption. The probability of absorption is essentially the ratio of the absorption rate to the rate at which neutrons ``started-out'' moving in that direction. The neutron source rate in Cartesian geometries is \(S/2\) for planes, and in spherical geometries is \(S\). For a Cartesian system:

\begin{equation}
\label{eq:AbsorptionProbability_Cartesian}
p(x)=\frac{\Sigma_a\phi dx}{S/2}
\end{equation}

Once you have expressed \(p(x)\), conduct a diffusion analysis to determine the flux from the appropriate source, and substitute this in for \(\phi\) in Eq. \ref{eq:AbsorptionProbability_Cartesian}. Again for a Cartesian geometry, this becomes:

\begin{equation}
\label{AverageDistance_Cartesian}
\bar{x}=\int_{0}^{\infty}x\frac{\Sigma_a\left(\frac{SL}{2D}\exp(-x/L)\right)dx}{S/2}=L
\end{equation}

And hence in Cartesian geometry, the diffusion length is exactly the average distance traveled by a neutron from birth to death.
