\section{The Spectral Angular Method}
\label{sec:PN}

An alternative angular discretization strategy to the discrete ordinates equations described in Section \ref{sec:SN} is to expand the angular dependence of the angular flux in a finite series of orthogonal functions. In general 3-D geometries, these functions are typically selected as the spherical harmonics functions described in Section \ref{sec:SH}. The spherical harmonics functions reduce to the Legendre polynomials in 1-D, so for 1-D geometries, these functions are typically selected as the Legendre polynomials described in Section \ref{sec:LegendrePolynomials}. After the flux has been expanded as a finite sum of functions and inserted into the \gls{nte}, the entire equation is multiplied by a function of different order and properties of orthogonality used to derived a set of coupled equations for the coefficients of the expansion. While the discrete ordinates method requires interpolation between the allowable directions of motion \(\hO_n\), the \(P_N\) method yields a continuous angular dependence.

Expanding the angular flux as a finite series of orthogonal functions is known as the \(P_N\) method. This method provides reasonably-correct spatial solutions, but errors are often present in magnitude that can be difficult to detect.

