\documentclass[10pt]{article}
\usepackage[letterpaper]{geometry}
\geometry{verbose,tmargin=1in,bmargin=1in,lmargin=1in,rmargin=1in}
\usepackage{setspace}
\usepackage{ragged2e}
\usepackage{color}
\usepackage{titlesec}
\usepackage{graphicx}
\usepackage{float}
\usepackage{mathtools}
\usepackage{amsmath}
\usepackage[font=small,labelfont=bf,labelsep=period]{caption}
\usepackage[english]{babel}
\usepackage{indentfirst}
\usepackage{array}
\usepackage{makecell}
\usepackage[usenames,dvipsnames]{xcolor}
\usepackage{multirow}
\usepackage{tabularx}
\usepackage{arydshln}
\usepackage{caption}
\usepackage{subcaption}
\usepackage{xfrac}
\usepackage{etoolbox}
\usepackage{cite}
\usepackage{url}
\usepackage{dcolumn}
\usepackage{hyperref}
\usepackage{courier}
\usepackage{url}
\usepackage{esvect}
\usepackage{commath}
\usepackage{verbatim} % for block comments
\usepackage{enumitem}
\usepackage{hyperref} % for clickable table of contents
\usepackage{braket}
\usepackage{titlesec}
\usepackage{booktabs}
\usepackage{gensymb}
\usepackage{longtable}
\usepackage{soul} % for striking out text
\usepackage{tcolorbox} % for colored boxes
\tcbuselibrary{breakable} % to allow colored boxed to extend over multiple pages
\usepackage[makeroom]{cancel}	% to cancel out text
\usepackage{breqn}
\usepackage[mathscr]{euscript}
\usepackage{listings}
\lstset{
	basicstyle=\ttfamily\small,
    frame=single,
    breaklines=true,
    postbreak=\raisebox{0ex}[0ex][0ex]{\ensuremath{\color{red}\hookrightarrow\space}}
}

% new commands for shorter writing of equations
\newcommand{\beq}{\begin{equation}}
\newcommand{\eeq}{\end{equation}}
\newcommand{\beqa}{\begin{equation}\begin{aligned}}
\newcommand{\eeqa}{\end{aligned}\end{equation}}
\newcommand{\la}{\langle}
\newcommand{\ra}{\rangle}

\titleclass{\subsubsubsection}{straight}[\subsection]

% define new command for triple sub sections
\newcounter{subsubsubsection}[subsubsection]
\renewcommand\thesubsubsubsection{\thesubsubsection.\arabic{subsubsubsection}}
\renewcommand\theparagraph{\thesubsubsubsection.\arabic{paragraph}} % optional; useful if paragraphs are to be numbered

\titleformat{\subsubsubsection}
  {\normalfont\normalsize\bfseries}{\thesubsubsubsection}{1em}{}
\titlespacing*{\subsubsubsection}
{0pt}{3.25ex plus 1ex minus .2ex}{1.5ex plus .2ex}

\makeatletter
\renewcommand\paragraph{\@startsection{paragraph}{5}{\z@}%
  {3.25ex \@plus1ex \@minus.2ex}%
  {-1em}%
  {\normalfont\normalsize\bfseries}}
\renewcommand\subparagraph{\@startsection{subparagraph}{6}{\parindent}%
  {3.25ex \@plus1ex \@minus .2ex}%
  {-1em}%
  {\normalfont\normalsize\bfseries}}
\def\toclevel@subsubsubsection{4}
\def\toclevel@paragraph{5}
\def\toclevel@paragraph{6}
\def\l@subsubsubsection{\@dottedtocline{4}{7em}{4em}}
\def\l@paragraph{\@dottedtocline{5}{10em}{5em}}
\def\l@subparagraph{\@dottedtocline{6}{14em}{6em}}
\makeatother

\newcommand{\volume}{\mathop{\ooalign{\hfil$V$\hfil\cr\kern0.08em--\hfil\cr}}\nolimits}

\setcounter{secnumdepth}{4}
\setcounter{tocdepth}{4}

% Generate the glossary of acronyms
\usepackage[acronym]{glossaries}
\makeglossaries

\newacronym{RANS}{rans}{Reynolds-Averaged Navier Stokes}

\begin{document}

\begin{centering}
\large Fluid Dynamics\\
\end{centering}

\tableofcontents
\clearpage

\section{Introduction}
\begin{flushleft}\justify

This document contains my course notes for fluids, though a large portion of the theory is included in the PRONGHORN manual, and isn't be repeated here.

\section{Turbulence}

In the study of turbulence, many mathematical models separate the turbulent field into the sum of the mean field and a fluctuation, where a mean is denoted with angled brackets \(\la(.)\ra\) and the fluctuation with a prime. For instance, the velocity is decomposed into:

\beq
\vv{V}(\vv{x},t)=\la\vv{V}(x,t)\rangle+\vv{V}(x,t)'
\eeq

This decomposition is sometimes referred to as a ``Reynolds decomposition.'' Likewise, density and energy can also be decomposed. 

\begin{tcolorbox}[breakable]
Some important statistical results are needed before carrying out the derivation of the averaged Navier Stokes equations. The results presented here will be often in the derivation of turbulence models. A mean is defined as:

\beq
\la U\ra\equiv\int_{0}^{\infty}Vf(V)dV
\eeq

The mean of the mean is simply equal to the mean, since the integral of \(f(V)\) from 0 to \(\infty\) is by definition unity. The mean of the fluctuation is zero:

\beqa
\la U'\ra\equiv&\int_{0}^{\infty}(V-\la V\ra)f(V)dV\\
\equiv& \int_{0}^{\infty}Vf(V)dV - \int_{0}^{\infty}\la V\ra f(V)dV\\
\equiv&\la U\ra - \la U\ra\\
\equiv&\ 0\\
\eeqa

The mean of the product of a fluctuation and a mean is:

\beqa
\la U'\la\rho\ra\ra\equiv&\int_{0}^{\infty}(V-\la V\ra)\la p\ra f(V)dV\\
\equiv&\int_{0}^{\infty}\left(V\la p\ra-\la V\ra\la p\ra\right)f(V)dV\\
\equiv&\la V\ra\la\rho\ra-\la V\ra\la\rho\ra\\
\equiv&\ 0\\
\eeqa

\end{tcolorbox}


The mean flow equations are derived by substituting these decompositions into the Navier-Stokes equations and then taking the mean of the entire equation. The continuity equation becomes:

\beqa
\frac{\partial\la \left(\la\rho\ra+\rho'\right)}{\partial t}+\nabla\cdot\la \left((\la\rho\ra+\rho')(\la\vv{V}\ra+\vv{V}')\right)\ra=&0\\
\frac{\partial\la\rho\ra}{\partial t}+\frac{\partial\rho'}{\partial t}+\nabla\cdot\la \left(\la\rho\ra\la\vv{V}\ra+\la\rho\ra\vv{V}'+\la\vv{V}\ra\rho'+\rho'\vv{V}'\right)\ra=&0\\
\eeqa

Taking the mean of the equation, and recognizing that taking the mean and differentiation with respect to both time and space commute:

\beqa
\frac{\partial\la\la\rho\ra\ra}{\partial t}+\cancel{\frac{\partial\la\rho'\ra}{\partial t}}+\nabla\cdot\left(\la\la\rho\ra\la\vv{V}\ra\ra+\la\la\rho\ra\vv{V}'\ra+\la\la\vv{V}\ra\rho'\ra+\la\rho'\vv{V}'\ra\right)=&0\\
\frac{\partial\la\rho\ra}{\partial t}+\nabla\cdot\left(\la\rho\ra\la\vv{V}\ra+\cancel{\la\la\rho\ra\vv{V}'\ra}+\cancel{\la\la\vv{V}\ra\rho'\ra}+\la\rho'\vv{V}'\ra\right)=&0\\
\frac{\partial\la\rho\ra}{\partial t}+\nabla\cdot\left(\la\rho\ra\la\vv{V}\ra+\la\rho'\vv{V}'\ra\right)=&0\\
\eeqa

For incompressible flow, the continuity equation simplifies to:

\beq
\label{eq:1}
\nabla\cdot\vv{V}=0
\eeq

Taking the average of this equation:

\beq
\label{eq:2}
\nabla\cdot\la\vv{V}\ra=0
\eeq

Alternatively, plugging in the Reynolds decomposition to Eq. \eqref{eq:1} gives:

\beq
\label{eq:3}
\nabla\cdot\la\vv{V}\ra+\nabla\cdot\vv{V}'=0
\eeq

Subtracting Eq. \eqref{eq:2} from Eq. \eqref{eq:3} shows that, for incompressible flow, both the mean of the velocity and the fluctuation in velocity are divergence-free. 



\end{flushleft}
\end{document}
